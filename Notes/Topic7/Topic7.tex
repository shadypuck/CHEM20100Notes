\documentclass[../notes.tex]{subfiles}

\pagestyle{main}
\renewcommand{\chaptermark}[1]{\markboth{\chaptername\ \thechapter\ (#1)}{}}
\renewcommand{\thechapter}{\Roman{chapter}}
\setcounter{chapter}{6}

\begin{document}




\chapter{Band Theory in Solids}
\section{Module 21: Electronic Structure of Solids (1D Solids)}
\begin{itemize}
    \item \marginnote{2/5:}Solid silicon's symmetry space group would be $Fd\overline{3}m$.
    \item Suggested reading: \textcite{bib:bandTheory}.
    \item To consider solids, let's first consider an infinite chain of hydrogen atoms.
    \begin{itemize}
        \item This should separate into \ce{H2} molecules (\textbf{Peierl's instability}).
        \item However, other substances can have chains of $p_z$ orbitals, such as platinum atoms.
    \end{itemize}
    \item An imaginary zoo of hydrogen molecules (we use the limit of a cycle of hydrogen atoms to approximate an infinitely long chain):
    \begin{itemize}
        \item \ce{H2} has a bonding and antibonding MO.
        \item Cyclic \ce{H3+} is the most abundant ion in the universe (recently discovered by UChicago). One bonding and two antibonding orbitals.
        \item We can keep adding hydrogen atoms to our rings.
        \item For an infinitely long cycle of hydrogen atoms, we will have an infinite number of states close together that resembles a band in solids.
    \end{itemize}
    \item Back to the chain of \ce{H} atoms:
    \begin{itemize}
        \item The basis function on each lattice point is a \ce{H_{$1s$}} orbital.
        \item The appropriate SALC $\psi$ is
        \begin{equation*}
            \psi_k = \sum_n\e[ikna]\phi_n
        \end{equation*}
        \item In this formalism, $k$ is an index labeling irreducible representations of the translation group. $\psi$ transforms just like $a$, $e_1$, and $e_2$ (e.g., in the $C_5$ point symmetry group).
        \item This process of symmetry adaptation is called "forming Bloch functions."
    \end{itemize}
    \item Elementary band theory for extended solids:
    \begin{itemize}
        \item Energy bands in solids arise from overlapping atomic orbitals, which become the \textbf{crystal orbitals} that make up the bands.
        \item Recipe: Use LCAO (tight binding) approach.
        \item A crystal is a regular periodic array with translational symmetry.
        \item Periodic boundary conditions require $\psi(x+Na)=\psi(x)$, i.e., each wavefunction must be symmetry equivalent to the one in the neighboring cells.
        \item For a 1D solid with lattice constant $a$ and atom index $n$, \textbf{Bloch's theorem} tells us that the above SALC $\psi_k$ is a solution to the Schr\"{o}dinger equation.
    \end{itemize}
    \item If we calculate $\psi_0$ and $\psi_{\pi/a}$, we get the most and least bonding states possible, respectively (the least bonding state is the most antibonding state and has the highest energy).
    \begin{figure}[h!]
        \centering
        \begin{subfigure}[b]{0.35\linewidth}
            \centering
            \begin{tikzpicture}
                \draw (-2,0) -- (2,0);
                \foreach \x in {-1.5,-0.5,0.5,1.5} {
                    \filldraw [semithick,draw,fill=grt] (\x,0) circle (2.5mm);
                }
            \end{tikzpicture}
            \caption{$k=0$.}
            \label{fig:bandBonding-sa}
        \end{subfigure}
        \begin{subfigure}[b]{0.35\linewidth}
            \centering
            \begin{tikzpicture}
                \draw (-2,0) -- (2,0);
                \foreach \x in {-1.5,0.5} {
                    \filldraw [semithick,draw,fill=grt] (\x,0) circle (2.5mm);
                    \filldraw [semithick,draw,fill=white] ({\x+1},0) circle (2.5mm);
                }
            \end{tikzpicture}
            \caption{$k=\frac{\pi}{a}$.}
            \label{fig:bandBonding-sb}
        \end{subfigure}
        \caption{$s$ orbital bonding states.}
        \label{fig:bandBonding-s}
    \end{figure}
    \begin{align*}
        \psi_0 &= \phi_0+\phi_1+\phi_2+\phi_3+\cdots\\
        \psi_{\pi/a} &= \phi_0-\phi_1+\phi_2-\phi_3+\cdots
    \end{align*}
    \item At this point, we can construct a band between these two states.
    \emph{picture}
    \begin{itemize}
        \item The band is \emph{almost} infinite; it's on the order of Avogadro's number.
        \item We have as many $k$ values as translations in the crystal or as many unit cells in a crystal.
    \end{itemize}
    ...
    \item \textbf{First Brillouin zone}: The region that covers all possible energy states that the crystal can have.
    \begin{itemize}
        \item It is $-\frac{\pi}{a}<k<\frac{\pi}{a}$; which is the range of all possible values that the sine function will give.
    \end{itemize}
    \item There is one energy level for each value of $k$, but $E(k)=E(-k)$.
    \item The energy is proportional to the electron momentum.
    \item Calculation of 1D band structure:
    \begin{itemize}
        \item We have $N$ atoms such that $\psi_k=\sum_{n=0}^N\e[inka]\phi_n$.
        \item The crystal Schr\"{o}dinger equation is $\hat{H}\Psi(k)=E(k)\Psi(k)$.
        \item Thus, the electron energies are given by
        \begin{equation*}
            E(k) = \frac{\ev{\hat{H}}{\psi}}{\braket{\psi}}
        \end{equation*}
        \item Recall that in Dirac's bra-ket notation, $\ev{\hat{H}}{\psi}\equiv\int\psi^*\hat{H}\psi\dd{\tau}$; for normalized atomic orbitals and ignoring overlap integrals:
        \begin{equation*}
            \braket{\phi_m}{\phi_n} =
            \begin{cases}
                1 & m=n\\
                0 & m\neq n
            \end{cases}
        \end{equation*}
        \item Also recall that
        \begin{equation*}
            \braket{\psi} = \sum_{m,n}\e[i(n-m)ka]\braket{\phi_m}{\phi_n} = N
        \end{equation*}
        \item Thus, we can calculate for on-site ($m=n$):
        \begin{equation*}
            \ev{\hat{H}}{\psi(k)} = \sum_n\ev{\hat{H}}{\phi_n} = N\alpha
        \end{equation*}
        And for resonance ($m\neq n$), where we need only consider the two nearest neighbors:
        \begin{equation*}
            \mel{\e[-inka]\phi_n}{\hat{H}}{\e[i(n\pm 1)ka]\phi_{n\pm 1}} = \beta\e[\pm ika]
        \end{equation*}
        \item Putting everything together, we have
        \begin{equation*}
            E(k) = \frac{\ev{\hat{H}}{\psi}}{\braket{\psi}} = \frac{N\alpha+N\beta(\e[ika]+\e[-ika])}{N} = \alpha+2\beta\cos(ka)
        \end{equation*}
    \end{itemize}
    \item \textbf{Zone center}: The ? where all atomic orbitals are in phase (all bonding $\sigma$). \emph{Also known as} $\bm{\Gamma}$.
    \item \textbf{Zone border}: The ? where all atomic orbitals are out of phase (all antibonding $\sigma^*$). \emph{Also known as} $\bm{X}$.
    \item Large numbers of MOs form bands of states.
    \item \textbf{Band structure}: The plot of $E$ as a function of $k$.
    \begin{itemize}
        \item The one we've derived so far is an s-shape curve.
    \end{itemize}
    \item The $p$-orbitals are opposite --- they form a bonding state with inverted phases.
    \begin{figure}[h!]
        \centering
        \begin{subfigure}[b]{0.35\linewidth}
            \centering
            \begin{tikzpicture}
                \foreach \x in {-1,0,1} {
                    \draw [densely dashed] (\x,-0.3) -- ++(0,0.6);
                }
                \foreach \x in {-1.5,-0.5,0.5,1.5} {
                    \filldraw [semithick,draw,fill=grt] (\x,0)
                        to [out=90,in=90,out looseness=0.5] ({\x+0.4},0)
                        to [out=-90,in=-90,in looseness=0.5] cycle
                    ;
                    \draw [semithick] (\x,0)
                        to [out=90,in=90,out looseness=0.5] ({\x-0.4},0)
                        to [out=-90,in=-90,in looseness=0.5] cycle
                    ;
                }
            \end{tikzpicture}
            \caption{$k=0$.}
            \label{fig:bandBonding-pa}
        \end{subfigure}
        \begin{subfigure}[b]{0.35\linewidth}
            \centering
            \begin{tikzpicture}
                \path (0,-0.3) -- ++(0,0.6);
                \foreach \x in {-1.5,0.5} {
                    \filldraw [semithick,draw,fill=grt] (\x,0)
                        to [out=90,in=90,out looseness=0.5] ({\x+0.4},0)
                        to [out=-90,in=-90,in looseness=0.5] cycle
                    ;
                    \draw [semithick] (\x,0)
                        to [out=90,in=90,out looseness=0.5] ({\x-0.4},0)
                        to [out=-90,in=-90,in looseness=0.5] cycle
                    ;
                }
                \foreach \x in {-0.5,1.5} {
                    \draw [semithick] (\x,0)
                        to [out=90,in=90,out looseness=0.5] ({\x+0.4},0)
                        to [out=-90,in=-90,in looseness=0.5] cycle
                    ;
                    \filldraw [semithick,draw,fill=grt] (\x,0)
                        to [out=90,in=90,out looseness=0.5] ({\x-0.4},0)
                        to [out=-90,in=-90,in looseness=0.5] cycle
                    ;
                }
            \end{tikzpicture}
            \caption{$k=\frac{\pi}{a}$.}
            \label{fig:bandBonding-pb}
        \end{subfigure}
        \caption{$p$ orbital bonding states.}
        \label{fig:bandBonding-p}
    \end{figure}
    \item The analysis of Figures \ref{fig:bandBonding-s} and \ref{fig:bandBonding-p} can be done for many more types of orbitals, including $p_z$, $d_{z^2}$, and $d_{xz}$.
    \item Bonding orbital bands run uphill (concave upwards $E(k)$) at $k=0$ and antibonding orbital bands run downhill (concave downwards $E(k)$) at $k=0$.
    \item Energy bands run from $\alpha+2\beta$ to $\alpha-2\beta$ since $\beta$ is negative for $s$ orbitals.
    \item \textbf{Density of states}: The number of energy levels in the energy interval $\Delta E$. \emph{Also known as} \textbf{DOS}.
    \begin{figure}[h!]
        \centering
        \begin{tikzpicture}
            \footnotesize
            \begin{scope}[xshift=-4.85cm]
                \foreach \y in {0.4,0.5,...,3} {
                    \draw [ultra thick] (0,\y) -- ++(0.5,0);
                }
                \node at (1.15,1.7) {$\equiv$};
            \end{scope}
            \begin{scope}[xshift=-3cm]
                \draw [thick] (0,0.4) rectangle (0.5,3);
                \node at (1.15,1.7) {$\equiv$};
            \end{scope}
            \begin{scope}
                \draw (4,0) node[below]{$\pi/a$} -- node[below=2mm]{\small$k\longrightarrow$} (0,0) node[below]{$0$} -- node[left=2mm,align=center]{\small$\uparrow$\\$E$} (0,3.3);
                \draw (0.1,0.5) -- ++(-0.2,0) node[left]{$\alpha+2\beta$};
                \draw (0.1,2.9) -- ++(-0.2,0) node[left]{$\alpha-2\beta$};
    
                \draw [grx,thick] (0,0.5) cos (2,1.7) node[above left,black]{$E(k)$} sin (4,2.9);
            \end{scope}
            \begin{scope}[xshift=6cm]
                \draw (3,0) -- node[below=2mm]{\small$\text{DOS}\longrightarrow$} (0,0) node[below]{$0$} -- node[left=2mm,align=center]{\small$\uparrow$\\$E$} (0,3.3);
    
                \draw [gry,thick,rotate=90] (0.5,0) -- (0.5,-2) sin (1.7,-0.5) cos (2.9,-2) -- (2.9,0);
            \end{scope}
        \end{tikzpicture}
        \caption{Density of states.}
        \label{fig:densityOfStates}
    \end{figure}
    \begin{itemize}
        \item Proportional to the inverse slope of the band; steep bands with large overlap yield a small DOS, and vice versa for flat bands.
        \item Reality check: PES for a long-chain alkane (\ce{C36H74}) shows this inverse DOS relationship for a little while.
    \end{itemize}
\end{itemize}



\section{Module 22: Electronic Structure of Solids (2D and 3D solids)}
\begin{itemize}
    \item 2D band structure:
    \begin{itemize}
        \item Simple H\"{u}ckel: A two-dimensional square net ($s$ orbitals only (or $p_z$)).
        \begin{equation*}
            \psi(k) = \sum_{m,n}\e[ik_xma+ik_yna]\cdot\phi_{m,n}
        \end{equation*}
        \item Consider the \textbf{crystal orbitals} at special $k$ points (high symmetry).
        \item The \textbf{Brillouin zone} is 2D here (we have a \textbf{wave vector}).
        \begin{figure}[h!]
            \centering
            \begin{tikzpicture}[
                every node/.append style={black}
            ]
                \footnotesize
                \path (-2,0) -- (2,0);
                \draw (-1,-1) rectangle (1,1);

                \draw [gry,thick,-stealth] (0,0) -- (1.5,0) node[right]{$\vec{b}_1$};
                \draw [gry,thick,-stealth] (0,0) -- (0,1.5) node[above]{$\vec{b}_2$};

                \fill [grx] (0,0) circle (2pt) node[below left]{$\Gamma$};
                \fill [grx] (0,1) circle (2pt) node[above left]{$X$};
                \fill [grx] (1,1) circle (2pt) node[above right]{$M$};
                \fill [grx] (1,0) circle (2pt) node[above right]{$X$};
            \end{tikzpicture}
            \caption{2D Brillouin zone.}
            \label{fig:brillouinZone-2D}
        \end{figure}
        \begin{figure}[h!]
            \centering
            \begin{subfigure}[b]{0.24\linewidth}
                \centering
                \begin{tikzpicture}[scale=0.8]
                    \draw (0,0) grid (3,3);
                    \foreach \x in {0,...,3} {
                        \foreach \y in {0,...,3} {
                            \filldraw [semithick,draw,fill=grt] (\x,\y) circle (3.125mm);
                        }
                    }
                \end{tikzpicture}
                \caption{$\Gamma$: $k_x,k_y=0$.}
                \label{fig:specialKPointsa}
            \end{subfigure}
            \begin{subfigure}[b]{0.24\linewidth}
                \centering
                \begin{tikzpicture}[scale=0.8]
                    \draw (0,0) grid (3,3);
                    \foreach \x in {0,2} {
                        \foreach \y in {0,...,3} {
                            \filldraw [semithick,draw,fill=grt] (\x,\y) circle (3.125mm);
                        }
                    }
                    \foreach \x in {1,3} {
                        \foreach \y in {0,...,3} {
                            \filldraw [semithick,draw,fill=white] (\x,\y) circle (3.125mm);
                        }
                    }
                \end{tikzpicture}
                \caption{$X$: $k_x=\pi/a,k_y=0$.}
                \label{fig:specialKPointsb}
            \end{subfigure}
            \begin{subfigure}[b]{0.24\linewidth}
                \centering
                \begin{tikzpicture}[scale=0.8]
                    \draw (0,0) grid (3,3);
                    \foreach \x in {0,2} {
                        \foreach \y in {1,3} {
                            \filldraw [semithick,draw,fill=grt] (\x,\y) circle (3.125mm);
                        }
                        \foreach \y in {0,2} {
                            \filldraw [semithick,draw,fill=grt] ({\x+1},\y) circle (3.125mm);
                        }
                    }
                    \foreach \x in {0,2} {
                        \foreach \y in {0,2} {
                            \filldraw [semithick,draw,fill=white] (\x,\y) circle (3.125mm);
                        }
                        \foreach \y in {1,3} {
                            \filldraw [semithick,draw,fill=white] ({\x+1},\y) circle (3.125mm);
                        }
                    }
                \end{tikzpicture}
                \caption{$M$: $k_x,k_y=\pi/a$.}
                \label{fig:specialKPointsc}
            \end{subfigure}
            \begin{subfigure}[b]{0.24\linewidth}
                \centering
                \begin{tikzpicture}[scale=0.8]
                    \draw (0,0) grid (3,3);
                    \foreach \x in {0,...,3} {
                        \foreach \y in {1,3} {
                            \filldraw [semithick,draw,fill=grt] (\x,\y) circle (3.125mm);
                        }
                    }
                    \foreach \x in {0,...,3} {
                        \foreach \y in {0,2} {
                            \filldraw [semithick,draw,fill=white] (\x,\y) circle (3.125mm);
                        }
                    }
                \end{tikzpicture}
                \caption{$X$: $k_x=0,k_y=\pi/a$.}
                \label{fig:specialKPointsd}
            \end{subfigure}
            \caption{Special $k$ points.}
            \label{fig:specialKPoints}
        \end{figure}
        \begin{itemize}
            \item The center is the $\Gamma$ point ($k_x=k_y=0$). The midpoint of the lines are called $X$ points ($k_x=\frac{\pi}{a},k_y=0$, and vice versa). The maximum point is the $M$ point ($k_x=k_y=\frac{\pi}{a}$).
        \end{itemize}
        \item Calculating $E(k)$ in two dimensions.
        \begin{equation*}
            E(k) = \alpha+2\beta(\cos(k_xa)+\cos(k_ya))
        \end{equation*}
        \begin{figure}[H]
            \centering
            \begin{tikzpicture}[scale=0.75]
                \footnotesize
                \draw (3.41,0) -- node[below=3mm]{\small$k\longrightarrow$} (0,0) node[below]{$\Gamma$} -- node[left=1mm,align=center]{\small$\uparrow$\\$E$} (0,6);
                \draw [densely dashed,|-] (1,6) -- ++(0,-6) node[below]{$X$};
                \draw [densely dashed,|-] (2,6) -- ++(0,-6) node[below]{$M$};
                \draw [densely dashed,|-] (3.41,6) -- ++(0,-6) node[below]{$\Gamma$};

                \draw [grx,thick] (0,1.5) cos (0.5,2.5) sin (1,3.5) cos (1.5,4.5) sin (2,5.5) cos (2.71,3.5) sin (3.41,1.5);
            \end{tikzpicture}
            \caption{Schematic band structure (2D).}
            \label{fig:schematicBands-2D}
        \end{figure}
        \item Our schematic band structure (Figure \ref{fig:schematicBands-2D}) traces a 2D path $\Gamma\to X\to M\to\Gamma$.
    \end{itemize}
\end{itemize}




\end{document}