\documentclass[../notes.tex]{subfiles}

\pagestyle{main}
\renewcommand{\chaptermark}[1]{\markboth{\chaptername\ \thechapter\ (#1)}{}}
\renewcommand{\thechapter}{\Roman{chapter}}
\setcounter{chapter}{7}

\begin{document}




\chapter{Electronic Spectra of Coordination Compounds}
\section{Module 40: Electronic Transitions}
\begin{itemize}
    \item \marginnote{3/1:}Suggested reading: Chapter 11.1.
    \item Transition metal complexes are known to show rich photophysics and optical properties.
    \begin{itemize}
        \item For example, \ce{[Ni(NH3)6]^2+} has peaks in the infrared, visible, and UV spectra.
    \end{itemize}
    \item How electronic transitions occur:
    \begin{itemize}
        \item Take a solid or aqueous sample and illuminate it with photons of a particular power $P_0$ through $l\si{\centi\meter}$ of it.
        \item Some will be absorbed and some will pass through. Measure the power $P$ that comes out on the other side.
    \end{itemize}
    \item Transmittance $T=P/P_0$.
    \item Absorbance $A=-\log T=\log\frac{P_0}{P}$.
    \item The Beer-Lambert Law: $A=\varepsilon Cl$, where $C$ is the concentration of the sample in solution, $l$ is the path length (how large the cuvette is), and $\varepsilon$ is the molar absorption coefficient.
    \item Plotting the wavelength of the impinging photons vs. $\varepsilon$ gives us a graph with peaks, where each peak corresponds to an electron transition.
    \item Spectral features:
    \begin{itemize}
        \item Number of transitions.
        \item Energy of the transitions.
        \item Intensity of the transitions.
        \item Shape of the transition.
    \end{itemize}
    \item \textbf{Transition probability}: The probability of a particular transition taking place.
    \item The transition probability depends on:
    \begin{itemize}
        \item Energy of the transition vs. incident light.
        \item Orientation of the molecule/material.
        \item Symmetry of the initial and final states.
        \item Angular momentum (spin).
    \end{itemize}
    \item The absorption spectra of various hexaaqua complexes of the first-row transition metals give us a zoo of spectra.
    \item We usually have $\varepsilon<10$, which means faint colors.
    \item Types of molecular transitions:
    \begin{itemize}
        \item Metal Centered (MC): Transitions between the $d$-orbitals on the metal center.
        \item Ligand to Metal Charge Transfer (LMCT): For example, \ce{MnO4-} has $\varepsilon\approx\num{10000}$.
        \item Metal to Ligand Charge Transfer (MLCT) and Metal to Metal Charge Transfer (MMCT), too.
    \end{itemize}
    \item The transition probability of one molecule from one state $\Psi_1$ to another state $\Psi_2$ is given by $|\vec{M}_{21}|$, the transition dipole moment or transition moment from $\Psi_1$ to $\Psi_2$.
    \begin{itemize}
        \item The transition matrix element $\vec{M}_{21}=\int\Psi_2\vec{\mu}\Psi_1\dd{\tau}$, where $\vec{\mu}$ is the electric dipole moment operator $\vec{\mu}=\sum_nQ_n\vec{x}_n$, where $Q_n$ is charge and $\vec{x}_n$ is the position vector operator.
        \item Derived with time-dependent perturbation theory.
        \item For an electronic transition to be allowed, the transition moment integral must be nonzero.
        \item Note that $\varepsilon\approx\vec{M}_{21}$.
    \end{itemize}
    \item How the HOMO moves about the molecule depends on the type of incoming light.
    \begin{itemize}
        \item If $\vec{M}_{21}=0$, then the transition probability is 0 and the transition from $\Psi_1$ to $\Psi_2$ is forbidden or electric-dipole forbidden ($\varepsilon=0$).
        \item If $\vec{M}_{21}\neq 0$, then the transition probability is not 0 and the transition from $\Psi_1$ to $\Psi_2$ is not forbidden ($\varepsilon\geq 0$).
        \begin{itemize}
            \item If $\vec{M}_{21}\neq 0$, we do not definitively know that there will be an electron transition or know how intense it will be; we just know that it is not electric-dipole forbidden.
        \end{itemize}
    \end{itemize}
    \item Calculating $\vec{M}_{21}$:
    \begin{itemize}
        \item Use the same procedure with $\Gamma_2\otimes\Gamma_\mu\otimes\Gamma_1$ as in Module 12.
        \item If the direct product does not contain the totally symmetric representation, then the transition is forbidden by symmetry arguments.
        \item If the direct product does contain the totally symmetric representation, then the transition is allowed by symmetry arguments.
    \end{itemize}
    \begin{table}[h!]
        \centering
        \small
        \renewcommand{\arraystretch}{1.4}
        \begin{tabular}{|l|lll|}
            \hline
            \normalsize$C_{3v}$ & $\bm{A_1}$ & $\bm{A_2}$ & $\bm{E}$\\
            \hline
            $\bm{A_1}$ & $A_1$ & $A_2$ & $E$\\
            $\bm{A_2}$ &       & $A_1$ & $E$\\
            $\bm{E}$   &       &       & $A_1+A_2+E$\\
            \hline
        \end{tabular}
        \caption{Direct product table for the $C_{3v}$ point group.}
        \label{tab:directProductTable-C3v}
    \end{table}
    \item Be aware of direct product tables, such as the above example, which we may readily obtain from Table \ref{tab:characterTable-C3v}.
    \item Example: In a $D_{2h}$ complex, can we excite a $d_{z^2}$ electron to the $p_z$ orbital?
    \begin{itemize}
        \item From the $D_{2h}$ character table, we have that $\Gamma_1=A_g$ and $\Gamma_2=B_{1u}$. We also have that $\Gamma_\mu$ for an $x$-, $y$-, and $z$-basis is $B_{3u}$, $B_{2u}$, and $B_{1u}$, respectively.
        \item Taking direct products under each basis gives us
        \begin{align*}
            B_{1u}\otimes B_{3u}\otimes A_g &= B_{2g}\tag{$x$-basis}\\
            B_{1u}\otimes B_{2u}\otimes A_g &= B_{3g}\tag{$y$-basis}\\
            B_{1u}\otimes B_{1u}\otimes A_g &= A_g\tag{$z$-basis}
        \end{align*}
        \item Thus, the $x$- and $y$-components are forbidden while the $z$ one is not.
        \item What this means is that $z$-plane polarized light will be able to cause the desired electron transition, but $x$- and $y$-plane polarized light will not.
    \end{itemize}
    \item We can use the same procedure to prove that we can never promote an electron from $d_{xy}$ to $p_z$.
    \item We can use the same procedure for octahedral complexes, except the calculations of the direct products are just a bit more difficult.
    \begin{itemize}
        \item For a $d^1$ complex, we calculate $E_g\otimes T_{1u}\otimes T_{2g}$.
        \item For a $d^6$ complex, we calculate $(T_{2g}\otimes E_g)\otimes T_{1u}\otimes A_{1g}=(T_{1g}+T_{2g})\otimes T_{1u}\otimes A_{1g}$.
        \begin{itemize}
            \item A low spin $d^6$ complex has $A_{1g}$ symmetry by taking the direct product of $T_{2g}$ times itself six times.
            \item The excited state has $T_{2g}$ times itself five times, and then times $E_g$.
            \item Basically, we take the direct product of the orbital that each electron occupies.
        \end{itemize}
    \end{itemize}
\end{itemize}



\section{Module 41: Many Electron States}
\begin{itemize}
    \item Suggested reading: Chapter 11.2.
    \item For octahedral $d^3$, we have multiple excited states (six, to be exact).
    \begin{itemize}
        \item Fortunately, there is an easier way to describe transitions between states (we will talk about this next time).
    \end{itemize}
    \item A single electron is completely described by the principal quantum number $n$, its angular momentum $\ell$, its magnetic quantum number $m_\ell$, and its spin $m_s$.
    \item Multielectron states are described by \textbf{Russell-Sounders coupling}, \emph{also known as} \textbf{LS coupling}, \textbf{L-S coupling}.
    \item For example, consider the $d^2$ configured \ce{V^3+} ion.
    \begin{itemize}
        \item There are 45 different possible microstates. Some will have the same energy, some will not.
        \item There are five states (denoted by \textbf{term symbols}) with distinct energy in total.
    \end{itemize}
    \item To find the term symbol, we need:
    \begin{itemize}
        \item $L=\text{total orbital angular momentum}=\sum m_\ell$.
        \item $S=\text{total spin angular momentum}=\sum m_s$.
    \end{itemize}
    \item Term symbols then are of the form
    \begin{equation*}
        {}^{2S+1}L_J
    \end{equation*}
    where $2S+1$ is the spin multiplicity, $L$ is the subshell letter corresponding to the angular momentum quantum number ($0\mapsto s$, $1\mapsto p$, $2\mapsto d$, $3\mapsto f$, \dots), and $J$ is the total angular momentum ($J=L+S,L+S-1,L+S-2,\dots,|L-S|$, the spin orbit coupling).
    \item Some examples:
    \begin{itemize}
        \item \tikz[xscale=1.5,every node/.append style={black},baseline={(0,0)}]{\footnotesize\draw [gry,ultra thick] (0,0) -- node{\Large$\upharpoonleft$} node[below=2mm]{$+1/2$} ++(0.5,0) ++(0.1,0) -- ++(0.5,0) ++(0.1,0) -- ++(0.5,0) ++(0.1,0) -- ++(0.5,0) ++(0.1,0) -- ++(0.5,0)}: $S=\frac{1}{2}$, so our term symbol will be of the form ${}^2L_J$.
        \item \tikz[xscale=1.5,every node/.append style={black},baseline={(0,0)}]{\footnotesize\draw [gry,ultra thick] (0,0) -- node{\Large$\upharpoonleft$} node[below=2mm]{$+1/2$} ++(0.5,0) ++(0.1,0) -- node{\Large$\upharpoonleft$} node[below=2mm]{$+1/2$} ++(0.5,0) ++(0.1,0) -- node{\Large$\downharpoonright$} node[below=2mm]{$-1/2$} ++(0.5,0) ++(0.1,0) -- node{\Large$\upharpoonleft$} node[below=2mm]{$+1/2$} ++(0.5,0) ++(0.1,0) -- ++(0.5,0)}: $S=1$, so our term symbol will be of the form ${}^3L_J$.
        \item \tikz[xscale=1.5,every node/.append style={black},baseline={(0,0)}]{\footnotesize\draw [gry,ultra thick] (0,0) -- node{\Large$\upharpoonleft$} node[below=2mm]{$+2$} ++(0.5,0) ++(0.1,0) -- node{\Large$\upharpoonleft$} node[below=2mm]{$+1$} ++(0.5,0) ++(0.1,0) -- node[below=2mm]{$0$} ++(0.5,0) ++(0.1,0) -- node[below=2mm]{$-1$} ++(0.5,0) ++(0.1,0) -- node[below=2mm]{$-2$} ++(0.5,0)}: $S=1$ and $L=3$, so our term symbol will be of the form ${}^3F_J$.
        \item \tikz[xscale=1.5,every node/.append style={black},baseline={(0,0)}]{\footnotesize\draw [gry,ultra thick] (0,0) -- node{\Large$\upharpoonleft$} node[below=2mm]{$+2$} ++(0.5,0) ++(0.1,0) -- node[below=2mm]{$+1$} ++(0.5,0) ++(0.1,0) -- node[below=2mm]{$0$} ++(0.5,0) ++(0.1,0) -- node[below=2mm]{$-1$} ++(0.5,0) ++(0.1,0) -- node{\Large$\upharpoonleft$} node[below=2mm]{$-2$} ++(0.5,0)}: $S=1$ and $L=0$, so our term symbol will be of the form ${}^3S_J$.
    \end{itemize}
    \item Spin-orbit coupling is weak for a carbon atom (we can essentially just disregard it).
    \begin{itemize}
        \item For lanthanides, it becomes very significant.
    \end{itemize}
\end{itemize}




\end{document}