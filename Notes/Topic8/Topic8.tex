\documentclass[../notes.tex]{subfiles}

\pagestyle{main}
\renewcommand{\chaptermark}[1]{\markboth{\chaptername\ \thechapter\ (#1)}{}}
\renewcommand{\thechapter}{\Roman{chapter}}
\setcounter{chapter}{7}

\begin{document}




\chapter{Electronic Spectra of Coordination Compounds}
\section{Module 40: Electronic Transitions}
\begin{itemize}
    \item \marginnote{3/1:}Suggested reading: Chapter 11.1.
    \item Transition metal complexes are known to show rich photophysics and optical properties.
    \begin{itemize}
        \item For example, \ce{[Ni(NH3)6]^2+} has peaks in the infrared, visible, and UV spectra.
    \end{itemize}
    \item How electronic transitions occur:
    \begin{itemize}
        \item Take a solid or aqueous sample and illuminate it with photons of a particular power/intensity $P_0$ through $l\si{\centi\meter}$ of it.
        \item Some will be absorbed and some will pass through. Measure the power $P$ that comes out on the other side.
    \end{itemize}
    \item Transmittance $T=P/P_0$.
    \item Absorbance $A=-\log T=\log\frac{P_0}{P}$.
    \item The Beer-Lambert Law: $A=\varepsilon Cl$, where $C$ is the concentration of the sample in solution, $l$ is the path length (the length through solution), and $\varepsilon$ is the molar absorption coefficient.
    \item Plotting the wavelength of the impinging photons vs. $\varepsilon$ gives us a graph with peaks, where each peak corresponds to an electron transition.
    \item Spectral features:
    \begin{itemize}
        \item Number of transitions.
        \item Energy of the transitions.
        \item Intensity of the transitions.
        \item Shape of the transition.
    \end{itemize}
    \item \textbf{Transition probability}: The probability of a particular transition taking place.
    \item The transition probability depends on:
    \begin{itemize}
        \item Energy of the transition vs. incident light.
        \item Orientation of the molecule/material.
        \item Symmetry of the initial and final states.
        \item Angular momentum (spin).
    \end{itemize}
    \item The absorption spectra of various hexaaqua complexes of the first-row transition metals give us a zoo of spectra.
    \item We usually have $\varepsilon<10$, which means faint colors.
    \item Types of molecular transitions:
    \begin{itemize}
        \item Metal Centered (MC): Transitions between the $d$-orbitals on the metal center.
        \item Ligand to Metal Charge Transfer (LMCT): For example, \ce{MnO4-} has $\varepsilon\approx\num{10000}$.
        \item Metal to Ligand Charge Transfer (MLCT) and Metal to Metal Charge Transfer (MMCT), too.
    \end{itemize}
    \item The transition probability of one molecule from one state $\Psi_1$ to another state $\Psi_2$ is given by $|\vec{M}_{21}|$, the transition dipole moment or transition moment from $\Psi_1$ to $\Psi_2$.
    \begin{itemize}
        \item The transition matrix element $\vec{M}_{21}=\int\Psi_2\vec{\mu}\Psi_1\dd{\tau}$, where $\vec{\mu}$ is the electric dipole moment operator $\vec{\mu}=\sum_nQ_n\vec{x}_n$, where $Q_n$ is charge and $\vec{x}_n$ is the position vector operator.
        \item Derived with time-dependent perturbation theory.
        \item For an electronic transition to be allowed, the transition moment integral must be nonzero.
        \item Note that $\varepsilon\approx\vec{M}_{21}$.
    \end{itemize}
    \item How the HOMO moves about the molecule depends on the type of incoming light.
    \begin{itemize}
        \item If $\vec{M}_{21}=0$, then the transition probability is 0 and the transition from $\Psi_1$ to $\Psi_2$ is forbidden or electric-dipole forbidden ($\varepsilon=0$).
        \item If $\vec{M}_{21}\neq 0$, then the transition probability is not 0 and the transition from $\Psi_1$ to $\Psi_2$ is not forbidden ($\varepsilon\geq 0$).
        \begin{itemize}
            \item If $\vec{M}_{21}\neq 0$, we do not definitively know that there will be an electron transition or know how intense it will be; we just know that it is not electric-dipole forbidden.
        \end{itemize}
    \end{itemize}
    \item Calculating $\vec{M}_{21}$:
    \begin{itemize}
        \item Use the same procedure with $\Gamma_2\otimes\Gamma_\mu\otimes\Gamma_1$ as in Module 12.
        \item If the direct product does not contain the totally symmetric representation, then the transition is forbidden by symmetry arguments.
        \item If the direct product does contain the totally symmetric representation, then the transition is allowed by symmetry arguments.
    \end{itemize}
    \begin{table}[h!]
        \centering
        \small
        \renewcommand{\arraystretch}{1.4}
        \begin{tabular}{|c|lll|}
            \hline
            \normalsize$C_{3v}$ & $\bm{A_1}$ & $\bm{A_2}$ & \multicolumn{1}{c|}{$\bm{E}$}\\
            \hline
            $\bm{A_1}$ & $A_1$ & $A_2$ & $E$\\
            $\bm{A_2}$ &       & $A_1$ & $E$\\
            $\bm{E}$   &       &       & $A_1+[A_2]+E$\\
            \hline
        \end{tabular}
        \caption{Direct product table for the $C_{3v}$ point group.}
        \label{tab:directProductTable-C3v}
    \end{table}
    \item Be aware of direct product tables, such as the above example, which we may readily obtain from Table \ref{tab:characterTable-C3v}.
    \item Example: In a $D_{2h}$ complex, can we excite a $d_{z^2}$ electron to the $p_z$ orbital?
    \begin{itemize}
        \item From the $D_{2h}$ character table, we have that $\Gamma_1=A_g$ and $\Gamma_2=B_{1u}$. We also have that $\Gamma_\mu$ for an $x$-, $y$-, and $z$-basis is $B_{3u}$, $B_{2u}$, and $B_{1u}$, respectively.
        \item Taking direct products under each basis gives us
        \begin{align*}
            B_{1u}\otimes B_{3u}\otimes A_g &= B_{2g}\tag{$x$-basis}\\
            B_{1u}\otimes B_{2u}\otimes A_g &= B_{3g}\tag{$y$-basis}\\
            B_{1u}\otimes B_{1u}\otimes A_g &= A_g\tag{$z$-basis}
        \end{align*}
        \item Thus, the $x$- and $y$-components are forbidden while the $z$ one is not.
        \item What this means is that $z$-plane polarized light will be able to cause the desired electron transition, but $x$- and $y$-plane polarized light will not.
    \end{itemize}
    \item We can use the same procedure to prove that we can never promote an electron from $d_{xy}$ to $p_z$.
    \item We can use the same procedure for octahedral complexes, except the calculations of the direct products are just a bit more difficult.
    \begin{itemize}
        \item For a $d^1$ complex, we calculate $E_g\otimes T_{1u}\otimes T_{2g}$.
        \item For a $d^6$ complex, we calculate $(T_{2g}\otimes E_g)\otimes T_{1u}\otimes A_{1g}=(T_{1g}+T_{2g})\otimes T_{1u}\otimes A_{1g}$.
        \begin{itemize}
            \item A low spin $d^6$ complex has $A_{1g}$ symmetry by taking the direct product of $T_{2g}$ times itself six times.
            \item The excited state has $T_{2g}$ times itself five times, and then times $E_g$.
            \item Basically, we take the direct product of the orbital that each electron occupies.
        \end{itemize}
    \end{itemize}
\end{itemize}



\section{Module 41: Many Electron States}
\begin{itemize}
    \item Suggested reading: Chapter 11.2.
    \item For octahedral $d^3$, we have multiple excited states (six, to be exact).
    \begin{itemize}
        \item Fortunately, there is an easier way to describe transitions between states (we will talk about this next time).
    \end{itemize}
    \item A single electron is completely described by the principal quantum number $n$, its angular momentum $\ell$, its magnetic quantum number $m_\ell$, and its spin $m_s$.
    \item Multielectron states are described by \textbf{Russell-Saunders coupling}, \emph{also known as} \textbf{LS coupling}, \textbf{L-S coupling}.
    \item For example, consider the $d^2$ configured \ce{V^3+} ion.
    \begin{itemize}
        \item There are 45 different possible microstates. Some will have the same energy, some will not.
        \item There are five states (denoted by \textbf{term symbols}) with distinct energy in total.
    \end{itemize}
    \item To find the term symbol, we need:
    \begin{itemize}
        \item $L=\text{total orbital angular momentum}=\sum m_\ell$.
        \item $S=\text{total spin angular momentum}=\sum m_s$.
    \end{itemize}
    \item Term symbols then are of the form
    \begin{equation*}
        {}^{2S+1}L_J
    \end{equation*}
    where $2S+1$ is the spin multiplicity, $L$ is the subshell letter corresponding to the angular momentum quantum number ($0\mapsto s$, $1\mapsto p$, $2\mapsto d$, $3\mapsto f$, \dots), and $J$ is the total angular momentum ($J=L+S,L+S-1,L+S-2,\dots,|L-S|$, the spin-orbit coupling).
    \item Some examples:
    \begin{itemize}
        \item \tikz[xscale=1.5,every node/.append style={black},baseline={(0,0)}]{\footnotesize\draw [gry,ultra thick] (0,0) -- node{\Large$\upharpoonleft$} node[below=2mm]{$+1/2$} ++(0.5,0) ++(0.1,0) -- ++(0.5,0) ++(0.1,0) -- ++(0.5,0) ++(0.1,0) -- ++(0.5,0) ++(0.1,0) -- ++(0.5,0)}: $S=\frac{1}{2}$, so our term symbol will be of the form ${}^2L_J$\footnote{Pronounced "doublet state."}.
        \item \tikz[xscale=1.5,every node/.append style={black},baseline={(0,0)}]{\footnotesize\draw [gry,ultra thick] (0,0) -- node{\Large$\upharpoonleft$} node[below=2mm]{$+1/2$} ++(0.5,0) ++(0.1,0) -- node{\Large$\upharpoonleft$} node[below=2mm]{$+1/2$} ++(0.5,0) ++(0.1,0) -- node{\Large$\downharpoonright$} node[below=2mm]{$-1/2$} ++(0.5,0) ++(0.1,0) -- node{\Large$\upharpoonleft$} node[below=2mm]{$+1/2$} ++(0.5,0) ++(0.1,0) -- ++(0.5,0)}: $S=1$, so our term symbol will be of the form ${}^3L_J$.
        \item \tikz[xscale=1.5,every node/.append style={black},baseline={(0,0)}]{\footnotesize\draw [gry,ultra thick] (0,0) -- node{\Large$\upharpoonleft$} node[below=2mm]{$+2$} ++(0.5,0) ++(0.1,0) -- node{\Large$\upharpoonleft$} node[below=2mm]{$+1$} ++(0.5,0) ++(0.1,0) -- node[below=2mm]{$0$} ++(0.5,0) ++(0.1,0) -- node[below=2mm]{$-1$} ++(0.5,0) ++(0.1,0) -- node[below=2mm]{$-2$} ++(0.5,0)}: $S=1$ and $L=3$, so our term symbol will be of the form ${}^3F_J$\footnote{Pronounced "triplet eff state."}.
        \item \tikz[xscale=1.5,every node/.append style={black},baseline={(0,0)}]{\footnotesize\draw [gry,ultra thick] (0,0) -- node{\Large$\upharpoonleft$} node[below=2mm]{$+2$} ++(0.5,0) ++(0.1,0) -- node[below=2mm]{$+1$} ++(0.5,0) ++(0.1,0) -- node[below=2mm]{$0$} ++(0.5,0) ++(0.1,0) -- node[below=2mm]{$-1$} ++(0.5,0) ++(0.1,0) -- node{\Large$\upharpoonleft$} node[below=2mm]{$-2$} ++(0.5,0)}: $S=1$ and $L=0$, so our term symbol will be of the form ${}^3S_J$\footnote{Pronounced "triplet ess state."}.
    \end{itemize}
    \item Spin-orbit coupling is weak for a carbon atom (we can essentially just disregard it).
    \begin{itemize}
        \item For lanthanides, it becomes very significant.
    \end{itemize}
    \item \marginnote{3/3:}In term symbols, $J$ is typically a small correction to the energy for light elements.
    \begin{itemize}
        \item It becomes significant with the lanthanides and actinides.
    \end{itemize}
    \item We group electrons into terms because each state has a characteristic energy in the absence of external electric and magnetic fields.
    \begin{itemize}
        \item If we do apply an external magnetic fields, states will split into $2S+1$ substates.
        \item Each term includes multiple microstates, which are atomic states produced by interactions of the atom with a magnetic field.
    \end{itemize}
    \item Consider the examples from above. The third and fourth examples are very similar but give rise to two different terms. This is because if we have ions with these states, we will observe differing energy levels spectroscopically.
    \item What if we have one electron in the orbital with $m_\ell=-2$? Then how do we create the term symbol?
    \begin{itemize}
        \item It is part of one of the microstates of the ${}^1D$ term.
        \item We fill electrons starting with the lowest energy states, and the lowest energy state has the greatest spin multiplicity (e.g., $+2$; this is one of Hund's rules).
    \end{itemize}
    \item Correspondences between $d$ electron configurations and term symbols:
    \begin{align*}
        d^1 &= {}^2D&
            d^6 &= {}^5D\\
        d^2 &= {}^3F&
            d^7 &= {}^4F\\
        d^3 &= {}^4F&
            d^8 &= {}^3F\\
        d^4 &= {}^5D&
            d^9 &= {}^2D\\
        d^5 &= {}^6S&
            d^{10} &= {}^1S\\
    \end{align*}
    \begin{itemize}
        \item Note that a fully occupied or unoccupied subshell is always ${}^1S$.
        \item Now if we take a direct product of a singlet $S$ state with any other state, we will have just that state left over. This implies that we can ignore all fully occupied shells, and the term will be determined entirely by partially filled ones.
    \end{itemize}
    \item If we want to build the full picture and see all possible states, we need a \textbf{microstate table}.
    \item Microstate table:
    \begin{itemize}
        \item A microstate table contains all possible combinations of $m_\ell$ and $m_s$.
        \item Each microstate represents a possible electron configuration.
        \item It includes both ground and excited states.
        \item It must obey the Pauli exclusion principle.
    \end{itemize}
    \item Example ($p^2$ electron configuration):
    \begin{itemize}
        \item Microstate notation:
        \begin{itemize}
            \item For each of the $n$ electrons in the configuration, list a special symbol in an $n$-tuple.
            \item The special symbol will be a number ($m_\ell$) with either a $+$ or $-$ sign as an exponent (for positive and negative spin, respectively).
        \end{itemize}
        \item There are three ground state configurations.
        \begin{itemize}
            \item \tikz[xscale=1.5,every node/.append style={black},baseline={(0,0)}]{\footnotesize\draw [gry,ultra thick] (0,0) -- node{\Large$\upharpoonleft$} node[below=2mm]{$+1$} ++(0.5,0) ++(0.1,0) -- node{\Large$\upharpoonleft$} node[below=2mm]{$0$} ++(0.5,0) ++(0.1,0) -- node[below=2mm]{$-1$} ++(0.5,0)}: The microstate is $(1^+,0^+)$.
            \item \tikz[xscale=1.5,every node/.append style={black},baseline={(0,0)}]{\footnotesize\draw [gry,ultra thick] (0,0) -- node[below=2mm]{$+1$} ++(0.5,0) ++(0.1,0) -- node{\Large$\upharpoonleft$} node[below=2mm]{$0$} ++(0.5,0) ++(0.1,0) -- node{\Large$\upharpoonleft$} node[below=2mm]{$-1$} ++(0.5,0)}: The microstate is $(0^+,-1^+)$.
            \item \tikz[xscale=1.5,every node/.append style={black},baseline={(0,0)}]{\footnotesize\draw [gry,ultra thick] (0,0) -- node{\Large$\upharpoonleft$} node[below=2mm]{$+1$} ++(0.5,0) ++(0.1,0) -- node[below=2mm]{$0$} ++(0.5,0) ++(0.1,0) -- node{\Large$\upharpoonleft$} node[below=2mm]{$-1$} ++(0.5,0)}: The microstate is $(1^+,-1^+)$.
        \end{itemize}
        \item We will not show all excited state configurations, but we will show a few.
        \begin{itemize}
            \item \tikz[xscale=1.5,every node/.append style={black},baseline={(0,0)}]{\footnotesize\draw [gry,ultra thick] (0,0) -- node{\Large$\upharpoonleft$\hspace{-1mm}$\downharpoonright$} node[below=2mm]{$+1$} ++(0.5,0) ++(0.1,0) -- node[below=2mm]{$0$} ++(0.5,0) ++(0.1,0) -- node[below=2mm]{$-1$} ++(0.5,0)}: The microstate is $(1^+,1^-)$.
            \item \tikz[xscale=1.5,every node/.append style={black},baseline={(0,0)}]{\footnotesize\draw [gry,ultra thick] (0,0) -- node[below=2mm]{$+1$} ++(0.5,0) ++(0.1,0) -- node{\Large$\upharpoonleft$\hspace{-1mm}$\downharpoonright$} node[below=2mm]{$0$} ++(0.5,0) ++(0.1,0) -- node[below=2mm]{$-1$} ++(0.5,0)}: The microstate is $(0^+,0^-)$.
            \item \tikz[xscale=1.5,every node/.append style={black},baseline={(0,0)}]{\footnotesize\draw [gry,ultra thick] (0,0) -- node[below=2mm]{$+1$} ++(0.5,0) ++(0.1,0) -- node[below=2mm]{$0$} ++(0.5,0) ++(0.1,0) -- node{\Large$\upharpoonleft$\hspace{-1mm}$\downharpoonright$} node[below=2mm]{$-1$} ++(0.5,0)}: The microstate is $(-1^+,-1^-)$.
        \end{itemize}
        \item We can now generate the microstate table.
        \begin{table}[h!]
            \centering
            \small
            \renewcommand{\arraystretch}{1.4}
            \begin{tabular}{c|c|c|c|c|}
                \multicolumn{2}{c}{} & \multicolumn{3}{c}{$M_S$}\\
                \cline{3-5}
                \multicolumn{2}{c|}{} & $-1$ & $0$ & $+1$\\
                \cline{2-5}
                 & $+2$ & & $1^+1^-$ & \\
                \cline{2-5}
                 & $+1$ & $1^-0^-$ & \vertcell{$1^+0^-$\\[-3pt]$1^-0^+$} & $1^+0^+$\\
                \cline{2-5}
                $M_L$ & $0$ & $-1^-1^-$ & \vertcell{$-1^+1^-$\\[-3pt]$0^+0^-$\\[-3pt]$-1^-1^+$} & $-1^+1^+$\\
                \cline{2-5}
                 & $-1$ & $-1^-0^-$ & \vertcell{$-1^+0^-$\\[-3pt]$-1^-0^+$} & $-1^+0^+$\\
                \cline{2-5}
                 & $-2$ & & $-1^+-1^-$ & \\
                \cline{2-5}
            \end{tabular}
            \caption{Microstate table for a $p^2$ electron configuration.}
            \label{tab:microstateTable-p2}
        \end{table}
        \begin{itemize}
            \item Each column represents a state with a given spin angular momentum.
            \item Each row represents the sum of the angular momentum of two electrons.
        \end{itemize}
        \item To analyze the microstate label, rename each microstate with $X$.
        \begin{itemize}
            \item Each term consists of multiple microstates of equivalent energies (similar to how we can have multiple degenerate orbitals of a given energy).
            \item We can lift degeneracy by applying an external electric (\textbf{Stark effect}) or magnetic field (\textbf{Ziemann effect}).
        \end{itemize}
        \item Now each term consists of $(2L+1)(2S+1)$ states (this is the \textbf{double multiplicity formula}).
        \item Group energetically equivalent states:
        \begin{itemize}
            \item First, focus on the term containing the states with the largest possible $L$ (i.e., those with $+2$). We know that it contains $(2(2)+1)(2(0)+1)=5$ microstates. Choose 5 microstates from the $M_S=0$ column, one per $M_L$.
            \item Next, focus on the term containing states with the next largest possible $L$ (i.e., those with $+1$ and $M_S=+1$). We know that it contains $(2(1)+1)(2(1)+1)=9$ microstates. Choose 9 microstates, one from each box in the square bounded by $(-1,-1)$ and $(+1,+1)$.
            \item The remaining microstate in $M_S=M_L=0$ forms its own term.
        \end{itemize}
        \item Thus, our microstate table can be decomposed into three terms (${}^1D$, ${}^3P$, and ${}^1S$).
    \end{itemize}
    \item Identifying relative energies with Hund's rules:
    \begin{itemize}
        \item For a given electron configuration, the term with the greatest spin multiplicity lies lowest in energy (Hund's rule).
        \item For a term with a given multiplicity, the greater the value of $L$, the lower the energy.
        \item Note that the rules for predicting the ground state always work, but they may fail in predicting the order of energies for excited states.
    \end{itemize}
    \item Thus, going back to our example, we have that energetically, ${}^3P<{}^1D<{}^1S$.
    \item Example ($d^2$ electron configuration):
    \begin{itemize}
        \item In this example, Hund's rules do not provide accurate energy predictions (they would predict ${}^3F<{}^3P<{}^1G<{}^1D<{}^1S$, but in reality, ${}^3F<{}^1D<{}^3P<{}^1G<{}^1S$).
    \end{itemize}
    \item \textbf{Electron-hole formalism}: The Russel-Saunders terms for $d^n$ and $d^{10-n}$ configurations are identical for $n=0,\dots,5$.
    \begin{itemize}
        \item We can rationalize this by thinking of $n$ electrons and $10-n$ holes (or positrons) as related to $10-n$ electrons and $n$ holes.
    \end{itemize}
\end{itemize}



\section{Module 42: Many Electron States and Transitions in Coordination Compounds}
\begin{itemize}
    \item Suggested reading: Chapter 11.3.
    \item Ligand field dependence ($d^1$ system):
    \begin{itemize}
        \item Degenerate symmetric field:
        \begin{itemize}
            \item Absence of ligand field.
            \item Free-ion term.
            \item All $d$-orbitals are energetically equal.
            \item If all $d$ orbitals are degenerate, then we can put a single electron in any orbital and we will have a microstate of ${}^2D$.
        \end{itemize}
        \item Infinite $O_h$ field:
        \begin{itemize}
            \item Strong ligand field.
            \item Coordination complexes.
            \item $d$-orbitals are not degenerate ($d_{z^2,x^2-y^2}$ have higher energy; $d_{xy,xz,yz}$ have lower energy).
            \item In this case, it matters in which $d$ orbital we put the electron.
        \end{itemize}
        \item Real molecules:
        \begin{itemize}
            \item We use a correlation diagram or Orgel diagram.
            \item If ligand field strength is zero, we have the degenerate symmetric field. If it is at maximum strength, we have two distinct states ($t_{2g}$ and $e_g$). Anywhere in between, the states are in between in energy, too (as we apply the ligand field, we split the state).
        \end{itemize}
    \end{itemize}
    \item Ligand field dependence ($d^2$ system):
    \begin{figure}[h!]
        \centering
        \includegraphics[width=0.45\linewidth]{../ExtFiles/correlationDiagram-d2.png}
        \caption{Correlation diagram for a $d^2$ system.}
        \label{fig:correlationDiagram-d2}
    \end{figure}
    \begin{itemize}
        \item Under a weak ligand field, the multielectron states split.
        \begin{itemize}
            \item From the $O_h$ character table, we can determine how each term transforms similarly to determining how each orbital transforms (however, we may need a character table with cubic, quartic, and beyond functions).
            \item For example, $S$ terms transform as $A_{1g}$, $P$ as $T_{1g}$, $D$ as $T_{2g}+E_g$, $F$ as $T_{1g}+T_{2g}+A_{2g}$, and $G$ as $A_{1g}+E_g+T_{1g}+T_{2g}$.
        \end{itemize}
        \item Under an infinitely strong field, we will return to our $t_{2g}$ / $e_g$ formalism with a low energy ${t_{2g}}^2$ state, a midlevel $t_{2g}e_g$ state (with one of the $d^2$ electrons in each level), and a high level ${e_g}^2$ state.
        \begin{itemize}
            \item From the $O_h$ group multiplication table, we can determine how each state splits under strong but finite ligand fields.
            \item For example, $E_g\otimes E_g=A_{1g}+E_g+A_{2g}$, so thats why the $e_g^2$ state coalesces from these three states.
        \end{itemize}
        \item The bold lines in Figure \ref{fig:correlationDiagram-d2} correspond to triplet states, and the dashed lines to singlet states.
    \end{itemize}
    \item The direct product of two irreducible representations of dimension 2 or higher is reducible to a sum of symmetric and antisymmetric irreducible representations, so named because their respective symmetry-adapted basis functions are either symmetric or antisymmetric with respect to exchange of the components. It is important to understand these properties when dealing with the electronic states produced by incompletely filled degenerate orbitals.
    \begin{itemize}
        \item The following expressions may be applied to determine the characters ($\chi$) of the symmetric and antisymmetric components of a direct product. The characters of the symmetric irreducible representation(s) ($\chi^+$) are given by
        \begin{equation*}
            \chi^+(R) = \frac{1}{2}\left( [\chi(R)]^2+\chi(R^2) \right)
        \end{equation*}
        The characters of the antisymmetric irreducible representation(s) are given by
        \begin{equation*}
            \chi^-(R) = \frac{1}{2}\left( [\chi(R)]^2-\chi(R^2) \right)
        \end{equation*}
        In these expressions, $\chi(R)$ is the character under symmetry operation $R$ and $\chi(R^2)$ is the character associated with the operation $R^2$.
        \begin{itemize}
            \item We do not need to know these formulas since deriving them is outside the scope of this course.
        \end{itemize}
        \item For example, in the $C_{4v}$ point group, the direct product $E\otimes E=A_1+A_2+B_1+B_2$.
        \item These results are typically written showing the antisymmetric component in square brackets, i.e., $E\otimes E=A_1+[A_2]+B_1+B_2$.
        \item The electron wave function, which is a product of the orbital and spin wave function components, must be antisymmetric (see Pauli exclusion principle). Therefore, if the orbital component is symmetric, the spin one should be antisymmetric, i.e., singlet state. And vice versa.
    \end{itemize}
    \item Back to our $d^2$ example:
    \begin{itemize}
        \item Symmetric components have electrons with opposite spins; antisymmetric components have electrons with like spin.
        \begin{itemize}
            \item This demonstrates how $T_2\otimes T_2=A_1+E+[T_1]+T_2$ rationalizes the existence of three singlet states and one triplet state under ${t_{2g}}^2$.
        \end{itemize}
    \end{itemize}
    \item Correlation diagrams are immensely useful, and we can tediously construct them or find them in the textbook. However, we can nicely simplify them into Tanabe-Sugano diagrams:
\end{itemize}
\begin{tchart}{1.4}{Correlation Diagram}{Tanabe-Sugano Diagram}
    Number of states. & Number of states.\\
    General sense of field effects. & Field effects.\\
    Only qualitative. & Quantitative.
\end{tchart}
\begin{itemize}
    \item Tanabe-Sugano diagrams are designed to intuitively interpret optical spectra and electron transitions in transition metal complexes.
    \item In a Tanabe-Sugano diagram, we make the lowest line in the corresponding correlation diagram the $x$-axis/ground state and calculate energies of every state above with respect to the gap between the ground state and the upper state.
    \item Tanabe-Sugano diagrams show:
    \begin{figure}[h!]
        \centering
        \includegraphics[width=0.35\linewidth]{../ExtFiles/tanabeSugano-d2.png}
        \caption{Tanabe-Sugano diagram for a $d^2$ system.}
        \label{fig:tanabeSugano-d2}
    \end{figure}
    \begin{itemize}
        \item Relative energies of the states vs. ligand field strength.
        \item Electronic states with the same symmetry never cross (\textbf{non-crossing rule}).
        \item Curvature (${}^1E$ and ${}^1E$).
        \item Ground state on the $x$-axis; all other states are excited.
        \item Transitions between states.
        \item A Tanabe-Sugano diagram is a graph which plots the energy of different spectroscopic terms (on the $y$-axis) against the strength of the ligand field (on the $x$-axis). The units used for each are given in terms of the \textbf{Racah parameter} $B$.
        \item The choice of unit means that the diagram takes account of electron-electron repulsion effects.
        \item The lowest energy state is usually placed along the $x$-axis.
        \item The electrostatic repulsion between electrons varies from atom to atom, depending upon the number and spin of the electrons and the orbitals they occupy. The total repulsion can be expressed in terms of three parameters $A$, $B$, and $C$, which are known as the Racah parameters. They are generally obtained empirically from gas-phase spectroscopic studies of atoms.
    \end{itemize}
    \item Each particular coordination compound will be a vertical slice through its electron configuration's Tanabe-Sugano diagram.
\end{itemize}



\section{Module 43: Using Tanabe-Sugano Diagrams}
\begin{itemize}
    \item \marginnote{3/5:}Suggested reading: Chapter 11.3.
    \item Extra Tanabe-Sugano facts:
    \begin{itemize}
        \item States of the complex have the same spin multiplicity as the free ion states from which they originate.
        \item States that are the only ones of their type have energies that depend linearly on the crystal field strength, whereas when there are two or more states of identical designation, their lines will in general show curvature. This is because such states interact with one another.
    \end{itemize}
    \item Selection rules determine the probability (intensity) of the transition.
    \item \textbf{Symmetry selection rule}: The initial and final wavefunctions must change in parity. Parity is related to the orbital angular momentum summation over all electrons $\sum {m_l}_i$, which can be even or odd; only even ($g$) $\leftrightarrow$ odd ($u$) transitions are allowed. Transitions between the orbitals of the same subshell are forbidden. \emph{Also known as} \textbf{Laporte selection rule}, \textbf{Parity selection rule}.
    \begin{itemize}
        \item For example, $g\to g$ and $u\to u$ are forbidden, but $g\to u$ and $u\to g$ are allowed.
        \item The weaker of the two selection rules.
        \item This is related back to Fermi's golden rule and $\vec{M}_{21}$.
        \item For $O_h$ complexes, $\Gamma_{\mu_{xyz}}=T_{1u}$.
        \item Direct product rules:
        \begin{align*}
            g\otimes u &= u&
            g\otimes g &= g&
            u\otimes u &= g
        \end{align*}
        \item For example, if we try a $g\to g$ transition in an $O_h$ complex, we would have $\vec{M}_{21}=g\otimes u\otimes g=u$ is unsymmetric to inversion, meaning that there is no totally symmetric representation in $\vec{M}_{21}$.
    \end{itemize}
    \item Octahedral vs. tetrahedral absorption spectra:
    \begin{itemize}
        \item A tetrahedron has no center of symmetry, and so orbitals in this point group cannot be gerade. Hence, the $d$-levels in a tetrahedral complex are $e$ and $t_2$, with no "$g$" for gerade. This largely overcomes the Laporte selection rule, making tetrahedral complexes very intense in color.
        \item This is why a solution of \ce{[Co(H2O)6]^2+} is pale pink, but \ce{[CoCl4]^2-} is a very intense blue.
    \end{itemize}
    \item \textbf{Spin selection rule}: There must be no change in the spin multiplicity ($\Delta S=0$) during the transition. i.e., the spin of the electron must not change during the transition.
    \begin{itemize}
        \item For example, ${}^1T_1\to{}^1T_2$ is allowed, but ${}^1T_1\to{}^3T_1$ and ${}^3T_1\to{}^1A_2$ are forbidden.
        \item The stronger of the two selection rules.
    \end{itemize}
    \item All $d$-$d$ transitions are symmetry (Laporte) forbidden.
    \begin{itemize}
        \item Thus, it makes more sense to disregard the complete Tanabe-Sugano diagram and focus on the spin-only one.
    \end{itemize}
    \item Example: $d^8$ Tanabe-Sugano Diagram and \ce{[Ni(H2O)6]^2+} absorption spectrum.
    \begin{figure}[h!]
        \centering
        \begin{subfigure}[b]{0.45\linewidth}
            \centering
            \includegraphics[width=0.8\linewidth]{../ExtFiles/NiH2O6a.png}
            \caption{The $d^8$ Tanabe-Sugano diagram.}
            \label{fig:NiH2O6a}
        \end{subfigure}
        \begin{subfigure}[b]{0.45\linewidth}
            \centering
            \includegraphics[width=0.95\linewidth]{../ExtFiles/NiH2O6b.png}
            \caption{The absorption spectrum.}
            \label{fig:NiH2O6b}
        \end{subfigure}
        \caption{\ce{[Ni(H2O)6]^2+}: Relating a Tanabe-Sugano diagram and an absorption spectrum.}
        \label{fig:NiH2O6}
    \end{figure}
    \begin{itemize}
        \item The two higher energy peaks involve differences in the magnetic quantum numbers of the $d$ orbitals, and are labeled as ${}^3A_{2g}\to{}^3T_{1g}(F)$ and ${}^3A_{2g}\to{}^3T_{1g}(P)$ to reflect this.
        \item Note that there is a shoulder near the second band corresponding to a nominally forbidden transition.
    \end{itemize}
    \item $d^1$ and $d^9$ complexes have only one electron transition and, thus, one peak in the absorption spectrum.
    \begin{itemize}
        \item However, the peak is not symmetric; this is a result of the Jahn-Teller distortion of the excited state.
    \end{itemize}
    \item Analyzes a ruby's Tanabe-Sugano diagram and spectrum.
    \item The $d^6$ Tanabe-Sugano diagram:
    \begin{itemize}
        \item The curves are not everywhere differentiable, the ground state changes, there is a vertical line, etc.
        \item This is because at the line, the complex switches from high-spin to low spin (recall that increasing ligand field strength increases $\Delta$; increasing $\Delta$ enough will make it larger than the spin-pairing energy).
    \end{itemize}
    \item The colors of the electron transitions tell us where the compound falls along the $x$-axis in the Tanabe-Sugano diagram.
    \item In the $d^5$ high-spin configuration, any transition involves spin pairing and does not change parity. Thus, there is a very low probability of transitions occurring.
    \item In $d^n$ and $d^{10-n}$ complexes, the splitting order in the Tanabe-Sugano diagrams (as we increase ligand field strength) is inverted.
    \begin{itemize}
        \item For example, in $d^2$, ${}^3F$ splits into (from lowest to highest energy) ${}^3T_1(F)$, ${}^3T_2$, and ${}^3A_2$. In $d^8$, ${}^3F$ splits into (from lowest to highest energy) ${}^3A_2$, ${}^3T_2$, and ${}^3T_1(F)$.
    \end{itemize}
    \item Other factors that can influence whether or not electrons are promoted:
    \begin{itemize}
        \item In $d^2$ complexes, there are three allowed transitions. However, one of them would involve promotion of two electrons from $t_{2g}$ to $e_g$ simultaneously upon the absorption of one photon, and this is highly improbable.
    \end{itemize}
    \item By comparing the energies of multiple transitions, we can look for a place in the Tanabe-Sugano diagram that would produce such results.
    \begin{itemize}
        \item For example, if one known transition has 1.5 times the energy of another, we can look for a place where the distance between it and the ground state is 1.5 times that of the other.
    \end{itemize}
    \item We can also use experimental values to calculate $B$, the Racah parameter.
    \begin{itemize}
        \item With the Racah parameter, we can pretty much calculate anything we want.
    \end{itemize}
\end{itemize}



\section{Module 44: How Can We See "Forbidden" Transitions?}
\begin{itemize}
    \item \textbf{Oscillator strength}: A quantity proportional to the integral of $\varepsilon$ as measured in the absorption spectrum. \emph{Also known as} $\bm{f}$.
    \begin{itemize}
        \item $f$ relates $\varepsilon$ to the matrix element $\vec{M}_{21}$:
        \begin{equation*}
            \frac{4m_e\pi v}{3e^2\hbar}\cdot|\vec{M}_{21}|^2 = f \propto \num{4.3e-9}\int\varepsilon\dd{\bar{\nu}}
        \end{equation*}
        \item Thus, the probability of light absorption is related to $f$.
        \item Strong absorption occurs when $f$ is around 1.
        \item The rate constant $k^0_e$ for emission is related to $\varepsilon$ by
        \begin{equation*}
            k^0_e \propto \num{4.3e-9}\bar{\nu}^{-2}_0\int\varepsilon\dd{\bar{\nu}} = \bar{\nu}^{-2}_0f
        \end{equation*}
        \begin{itemize}
            \item Good emitters are also good absorbers.
        \end{itemize}
    \end{itemize}
    \item Allowed transitions:
    \begin{itemize}
        \item Allowedness is measured by $f$ which can be dissected into
        \begin{equation*}
            f = (f_e\times f_v\times f_s)f_\text{max}
        \end{equation*}
        where $f_e$ is related to electronic factors, $f_v$ is related to Franck-Condon factors, and $f_s$ is related to spin-orbit factors.
        \item A perfectly allowed transition has $f=1$.
        \item $f_s$ factors:
        \begin{itemize}
            \item A spin-allowed transition has $f_s=1$.
            \item For a spin-forbidden transition, $f_s$ depends on spin-orbit coupling.
        \end{itemize}
        \item $f_e$ factors: \textbf{overlap forbiddenness} and \textbf{orbital forbiddenness}.
    \end{itemize}
    \item \textbf{Overlap forbiddenness}: Poor spatial overlap of orbitals involved in electronic transitions.
    \item \textbf{Orbital forbiddenness}: Wavefunctions which overlap in space but cancel because of symmetry.
    \item Mechanisms that make forbidden electronic transitions allowed: \textbf{vibronic coupling}, \textbf{spin-orbit coupling}, and \textbf{mixing of states}.
    \item \textbf{Vibronic coupling}: Electronic states coupled to vibrational states help overcome the Laporte selection rule.
    \begin{figure}[h!]
        \centering
        \includegraphics[width=0.3\linewidth]{../ExtFiles/vibronicCoupling.png}
        \caption{Vibronic coupling.}
        \label{fig:vibronicCoupling}
    \end{figure}
    \begin{itemize}
        \item For an octahedral complex, there are 15 vibrational normal modes with irreducible representations: $\Gamma_\text{vibs}=a_{1g}+e_g+2t_{1u}+t_{2g}+t_{2u}$.
        \item Vibrational transitions couple with electronic transition: $\vec{M}_{21}=\int\psi_v^*\psi_e^*\vec{\mu}\psi_e\psi_v\dd{\tau}$. Indeed, the wavefunction of the complex can be represented by this product (this is called the \textbf{Franck-Condon principle}), which will give fully symmetric irreducible representations.
        \item Essentially, $T_{1u}$ and $T_{2u}$ vibrations can couple with the electronic transition to form the allowed vibronic transition.
        \item The band one sees in the UV-visible spectrum is the sum of bands due to transitions to coupled electronic ($E$) and vibrational energy levels ($u_1$, $u_2$, $u_3$, etc.).
    \end{itemize}
    \item \textbf{Spin-orbit coupling}: Spin and orbital angular momenta can interact to make spin forbidden transitions allowed.
    \begin{itemize}
        \item Allows us to relax spin selection rules and promote an electron from an electron pair \emph{without changing its spin}.
        \item This is possible in heavy atoms because of their relativistic properties.
    \end{itemize}
    \item \textbf{Mixing of states}: $\pi$-acceptor and $\pi$-donor ligands can mix with the $d$-orbitals; transitions are no longer purely $d$-$d$.
    \begin{itemize}
        \item The Tanabe-Sugano diagram assumes pure $d$-$d$ transitions.
        \item However, spin-forbidden transitions can borrow intensity from nearby spin-allowed transitions by mixing of states.
    \end{itemize}
    \item $\varepsilon_\text{max}(\si{M^{-1}.cm^{-1}})$ ranges for different transitions:
    \begin{itemize}
        \item Spin and symmetry forbidden $d$-$d$ bands: 0.02-1.
        \item Spin allowed and symmetry forbidden $d$-$d$ bands: 1-10.
        \item Spin and symmetry allowed CT bands: $\num{e3}$-$\num{5e4}$.
    \end{itemize}
\end{itemize}



\section{Module 45: Charge Transfer Transitions}
\begin{itemize}
    \item \textbf{Charge-transfer band}: A Laporte- and spin-allowed, very intense absorption peak. \emph{Also known as} \textbf{CT band}.
    \item We can have transitions from metal $d$ orbitals to $p$ orbitals ($t_{2g}\to t_{1u}$).
    \begin{figure}[h!]
        \centering
        \includegraphics[width=0.8\linewidth]{../ExtFiles/chargeTransferTransitions.png}
        \caption{Electronic transitions in an octahedral complex.}
        \label{fig:chargeTransferTransitions}
    \end{figure}
    \item There exist \ce{M -> L} and \ce{L -> M} charge transfer transitions.
\end{itemize}



\section{Chapter 11: Coordination Chemistry III (Electronic Spectra)}
\emph{From \textcite{bib:MiesslerFischerTarr}.}
\begin{itemize}
    \item \marginnote{3/9:}We will consider the energy levels of $d$ electron configurations, how electrons in such atomic orbitals can interact with each other, and how the electronic absorption spectrum provides a convenient method (via finding $\Delta_o$) for determining the magnitude of the effect of ligands on the metal $d$ orbitals.
    \item \textbf{Complementary colors}: Two colors on opposite sides of the color wheel.
    \item In an absorption spectrum with one band, the color of the substance will be the complementary color of the wavelength absorbed.
    \item Beer-Lambert law:
    \begin{equation*}
        \log\frac{I_0}{I} = A = \varepsilon lc
    \end{equation*}
    \begin{figure}[h!]
        \centering
        \includegraphics[width=0.4\linewidth]{../ExtFiles/absorption.png}
        \caption{Absorption of light by solution.}
        \label{fig:absorption}
    \end{figure}
    \begin{itemize}
        \item Note that $\varepsilon$ is the molar absorptivity or \textbf{molar extinction coefficient}, measured in units of $\si{\liter\per\mole\per\centi\meter}$.
    \end{itemize}
    \item Absorbance is a dimensionless quantity.
    \begin{itemize}
        \item An absorbance of 1.0 corresponds to 90\% absorption since
        \begin{align*}
            1 &= \log\frac{I_0}{I}\\
            10^1 &= \frac{I_0}{I}\\
            I &= 0.10I_0
        \end{align*}
        so 10\% of light is transmitted, meaning that 90\% is absorbed.
    \end{itemize}
    \item \textbf{Spectrophotometer}: A device that obtains spectra as plots of absorbance versus wavelength.
    \item Consider a carbon atom, with its $p^2$ electron configuration.
    \begin{itemize}
        \item We might expect the $p$ electrons to be degenerate; however, we experimentally observe three major energy levels, one of which splits into three smaller energy levels. Let's find out why.
        \item Independently, each of the $2p$ electrons could have any of six possible $m_l,m_s$ combinations since $m_l=+1,0,-1$ and $m_s=+\frac{1}{2},-\frac{1}{2}$.
        \item However, the electrons are not independent of each other. Indeed, their angular momenta and spin angular moment interact via Russell-Saunders coupling.
        \begin{itemize}
            \item In oversimplified terms, imagine each electron as a magnet. The magnetic fields interact, producing microstates that can be described by the new quantum numbers $M_L$ and $M_S$.
        \end{itemize}
        \item Note that electrons in filled orbitals do not interact with the $p^2$ electrons since their net spin and orbital angular momenta are both zero.
    \end{itemize}
    \item \textbf{Microstate}: A set of possible quantum numbers that communicates a unique possible coupling of magnetic fields of the electrons.
    \item An alternate formulation for the number of microstates is
    \begin{equation*}
        \frac{i!}{j!(i-j)!}
    \end{equation*}
    where $i$ is the number of $m_l,m_s$ combinations (e.g., six for $p^2$, as discussed above) and $j$ is the number of electrons.
    \item Quantum numbers $M_L$ and $M_S$ in turn give quantum numbers $L$, $S$, and $J$.
    \begin{itemize}
        \item "Quantum numbers $L$ and $S$ describe collections of microstates, whereas $M_L$ and $M_S$ describe the microstates themselves. $L$ and $S$ are the largest possible values of $M_L$ and $M_S$" \parencite[408]{bib:MiesslerFischerTarr}.
        \item Just as $m_l=0,\pm 1,\dots,\pm l$, we have that $M_L=0,\pm 1,\dots,\pm L$. Similarly, like $m_s=+\frac{1}{2},-\frac{1}{2}$, we have that $M_S=S,S-1,\dots,-S$.
        \item $m_l$ describes the $z$-component of the magnetic field due to an electron's orbital motion; $M_L$ describes the $z$-component of the magnetic field associated with a microstate.
        \item $m_s$ describes an electron's magnetic spin; $M_S$ describes the analogous component of the magnetic field produced by electron spin for a microstate.
    \end{itemize}
    \item "States having spin multiplicities of 1, 2, 3, and 4 are described as singlet, double, triplet, and quartet states" \parencite[408]{bib:MiesslerFischerTarr}.
    \item \textbf{Free-ion term}: An atomic state characterized by $S$ and $L$. \emph{Also known as} \textbf{Russell-Saunders terms}.
    \begin{itemize}
        \item So named because they describe individual atoms and ions, i.e., ones that are free of ligands.
    \end{itemize}
    \item \textbf{Term symbol}: A label corresponding to a free-ion term and consisting of a letter relating to the value of $L$ and a left superscript for the spin multiplicity.
    \item Note that \textbf{term} and \textbf{state} are often used interchangeably, but they do technically have different meanings.
    \item \textbf{Term}: The preferred label for the results of Russell-Saunders coupling.
    \item \textbf{State}: The preferred label for the results of spin-orbit coupling, which includes the quantum number $J$.
    \item Spin multiplicity is equal to the number possible values of $M_S$, hence the number of columns in the microstate table.
    \item Returning to our $p^2$ carbon atom example:
    \begin{figure}[h!]
        \centering
        \includegraphics[width=0.8\linewidth]{../ExtFiles/microstateTable-p2-reduced.png}
        \caption{Reduced $p^2$ microstate table.}
        \label{fig:microstateTable-p2-reduced}
    \end{figure}
    \begin{itemize}
        \item Designate each microstate in Table \ref{tab:microstateTable-p2} with an $x$ for the sake of convenience.
        \item "To reduce the $p^2$ microstate table into its terms, all that is necessary is to find the rectangular arrays" \parencite[410]{bib:MiesslerFischerTarr}.
        \item Notice how the spin multiplicity in each term symbol is the same as the number of columns in each reduced microstate table in Figure \ref{fig:microstateTable-p2-reduced}.
        \item The ${}^3P$, ${}^1D$, and ${}^1S$ terms have three distinct energies (the three major energy levels observed experimentally).
    \end{itemize}
    \item \textbf{Spin-orbit coupling}: A phenomenon in which the spin and orbital angular momenta (or the magnetic fields associated with them) couple with each other.
    \item Returning to our $p^2$ carbon atom example, the ${}^1D$ term and ${}^1S$ term would only have one $J$ (namely, $2+0=0$ and $0+0=0$, respectively). However, the ${}^3P$ term subdivides into ${}^3P_2$, ${}^3P_1$, and ${}^3P_0$ since $1+1=2$, $1+1-1=1$, and $1+1-2=0$.
    \begin{figure}[h!]
        \centering
        \includegraphics[width=0.6\linewidth]{../ExtFiles/multielectronStates-p2.png}
        \caption{Multielectron energy levels in the $p^2$ configuration.}
        \label{fig:multielectronStates-p2}
    \end{figure}
    \begin{itemize}
        \item This subdivision explains the experimental observation of three smaller energy levels within one of the major energy levels (see Figure \ref{fig:multielectronStates-p2}).
        \item Notice how when we calculate $J$ values, we iterate down from $L+S$ to $|L-S|$. In ${}^3P$ for example, $1+1=2$ and $|1-1|=0$, so the possible $J$ values are 2, 1, and 0.
    \end{itemize}
    \item \textbf{Hund's third rule}: For subshells that are less than half filled, the state having the lowest $J$ value has the lowest energy; for subshells that are more than half filled, the state having the highest $J$ value has the lowest energy. Half-filled subshells have only one possible $J$ value.
    \item Spin-orbit coupling plays a significant role in heavy metals ($\text{atomic number}>40$).
    \item We now connect electron-electron interactions back to absorption spectra.
    \item \textcite{bib:MiesslerFischerTarr} list free-ion terms for $d^n$ configurations, since they are tedious to determine by hand.
    \item A method for identifying the lowest-energy term:
    \begin{enumerate}
        \item Sketch the energy levels, showing the $d$ electrons.
        \item Spin multiplicity of lowest-energy state is the number of unpaired electrons plus 1.
        \item Determine the maximum possible value of $M_L$ (sum of $m_l$ values) for the configuration as shown. This determines the type of free-ion term (e.g., $S$, $P$, $D$).
        \item Combine results of Steps 2 and 3 to get the ground term.
    \end{enumerate}
    \item Example: Applying the above method to $d^3$ octahedral symmetry.
    \begin{itemize}
        \item There will be three electrons of parallel spin, one in each of the three degenerate $t_{2g}$ orbitals.
        \item $2S+1=3+1=4$.
        \item $2+1+0=3$ (recall that we use Hund's rules to know that it is the $+2,+1,0$ orbitals filled first); therefore, $F$.
        \item Result: ${}^4F$.
    \end{itemize}
    \item \marginnote{3/11:}Vibronic coupling permits $d$-$d$  transitions having molar absorptivities in the range of approximately $5$-$\SI[per-mode=reciprocal]{50}{L.mol^{-1}.cm^{-1}}$.
    \item More on mixing of states: "Tetrahedral complexes often absorb more strongly than octahedral complexes of the same metal in the same oxidation state. Metal-ligand $\sigma$ bonding in transition-metal complexes of $T_d$ symmetry can be described as involving a combination of $sp^3$ and $sd^3$ hybridization of the metal orbitals; both types of hybridization are consistent with the symmetry. The mixing of $p$-orbital character (of $u$ symmetry)  with $d$-orbital character provides a second way of relaxing the [Laporte] selection rule" \parencite[414]{bib:MiesslerFischerTarr}.
    \item On correlation diagrams (considering a $d^2$ configuration, as earlier in the discussion surrounding Figure \ref{fig:correlationDiagram-d2}):
    \begin{itemize}
        \item Free ions have five energy levels: ${}^3F<{}^1D<{}^3P<{}^1G<{}^1S$.
        \item In a strong $O_h$ ligand field, there are three possible $d^2$ electron configurations:
        \begin{figure}[h!]
            \centering
            \includegraphics[width=0.6\linewidth]{../ExtFiles/strongFieldConfigs-d2.png}
            \caption{Strong ligand field $d^2$ electron configurations.}
            \label{fig:strongFieldConfigs-d2}
        \end{figure}
        \begin{itemize}
            \item Clearly, ${t_{2g}}^2$ is the ground state and the other two are excited states.
        \end{itemize}
        \item The above states are the \textbf{strong-field limit}, since no compound perfectly overrides LS coupling except one in an infinitely strong ligand field.
        \item Do we need to be able to reconstruct the symmetry splitting and pairing between the weak-field and strong-field states in Figure \ref{fig:correlationDiagram-d2}?
        \item Free-ion terms have symmetry characteristics that enable them to be reduced to their component irreducible representations. For the $O_h$ point group:
        \begin{table}[h!]
            \centering
            \small
            \renewcommand{\arraystretch}{1.4}
            \begin{tabular}{ll}
                \rowcolor{grx}
                \textcolor{white}{\textbf{Term}} & \textcolor{white}{\textbf{Irreducible Representations}}\\

                $S$ & $A_{1g}$\\
                \rowcolor{grz}
                $P$ & $T_{1g}$\\
                $D$ & $E_g+T_{2g}$\\
                \rowcolor{grz}
                $F$ & $A_{2g}+T_{1g}+T_{2g}$\\
                $G$ & $A_{1g}+E_g+T_{1g}+T_{2g}$\\
                \rowcolor{grz}
                $H$ & $E_g+2T_{1g}+T_{2g}$\\
                $I$ & $A_{1g}+A_{2g}+E_g+T_{1g}+2T_{2g}$\\
                \noalign{\global\arrayrulewidth=1pt}\arrayrulecolor{grx}\hline
                \noalign{\global\arrayrulewidth=0.4pt}
            \end{tabular}
            \caption{Splitting of free-ion terms in the $O_h$ point group.}
            \label{tab:freeIonOhSplitting}
        \end{table}
        \item Strong-field limit states can also be split into irreducible representations.
        \item "Each free-ion irreducible representation is matched with (correlates with) a strong-field irreducible representation having the same symmetry" \parencite[417]{bib:MiesslerFischerTarr}.
        \item The bolded lines in Figure \ref{fig:correlationDiagram-d2} are those with the same spin multiplicity as the ground state, i.e., the states available for spin-allowed transitions.
        \item The non-crossing rule applies to correlation diagrams, too.
        \item \textcite{bib:MiesslerFischerTarr} lists a reference with correlation diagrams for other $d$-electron configurations.
    \end{itemize}
    \item Note that the above revelations about the practicality of $d$-orbital splitting implies that "all the energy level diagrams in Chapter 10 are based in this strong-field limit. Although this perspective is useful to rationalize the bonding in transition metal complexes, understanding the spectra of these complexes requires additional considerations that result from electron-electron interactions" \parencite[415]{bib:MiesslerFischerTarr}.
    \item On Tanabe-Sugano diagrams (again of a $d^2$ configuration, as with Figure \ref{fig:tanabeSugano-d2}):
    \begin{itemize}
        \item The horizontal ground state does not mean it has now energy changes under increasing ligand field strength; we just do this because it's useful from a spectroscopic perspective to emphasize energy \emph{differences}.
        \item The letter in parentheses for ${}^3T_{1g}(F)$ and ${}^3T_{1g}(P)$ distinguishes the ${}^3T_{1g}$ terms arising from the ${}^3F$ and ${}^3P$ states, respectively.
        \item Axis units:
        \begin{itemize}
            \item Horizontal axis: $\Delta_o/B$, where $\Delta_o$ is the octahedral ligand field splitting parameter and $B$ is the \textbf{Racah parameter}.
            \item Vertical axis: $E/B$, where $E$ is the energy (of the excited states) above the ground states and $B$ is the Racah parameter.
        \end{itemize}
        \item Notice how in the ground state changes of $d^4$ to $d^7$ complexes, there is a spin multiplicity change reflecting the change in the number of unpaired electrons.
    \end{itemize}
    \item \textbf{Racah parameter}: A measure of the repulsion between terms of the same multiplicity.
    \begin{itemize}
        \item For $d^2$ for example, the energy difference between ${}^3F$ and ${}^3P$ is $15B$.
        \item Generally greater for a free ion than that same ion coordinated (this is because the Racah parameter depends on the amount of volume the valence electrons can access; electrons interact less if they occupy more volume).
        \begin{itemize}
            \item This means that the difference between $B$ in a free ion and a complex can be used to assess the degree of covalency in the metal-ligand bonds.
        \end{itemize}
    \end{itemize}
    \item \ce{[Mn(H2O)6]^2+} ($d^6$ configuration) is very pale pink because its ground state (${}^6A_{1g}$) is unique in its spin multiplicity. Indeed, all transitions are spin-forbidden.
    \begin{itemize}
        \item There are many such excited states, however, hence a more complicated spectrum when you zoom in.
    \end{itemize}
    \item The unsymmetric, Jahn-Teller affected peak of $d^{1,9}$ complexes:
    \begin{itemize}
        \item First, note that we can write electron configurations in terms of degenerate MOs. For example, a complex with a $d^9$ configuration also has a ${t_{2g}}^6{e_g}^3$ configuration. Note that this is analogous to writing a complex with a $p^4$ configuration as having a ${p_x}^2{p_y}^1{p_z}^1$ configuration.
        \item Electron configurations have symmetry labels that match their degeneracies (see Table \ref{tab:configurationLabels}).
        \begin{table}[h!]
            \centering
            \small
            \renewcommand{\arraystretch}{1.4}
            \begin{tabular}{cp{4cm}c}
                \rowcolor{grx}
                 & & \textcolor{white}{\textbf{Examples}}\\
                
                
                $T$ & Designates a triply degenerate asymmetrically occupied state. & \tikz[baseline={(0,0.95)}]{
                    \footnotesize
                    \begin{scope}
                        \draw [thick]
                            (0.35,0.75) -- ++(0.6,0) ++(0.1,0) -- ++(0.6,0)
                            (0,0) -- node[above]{$\bullet$} ++(0.6,0) ++(0.1,0) -- ++(0.6,0) ++(0.1,0) -- ++(0.6,0)
                        ;
                    \end{scope}
                    \begin{scope}[xshift=3cm]
                        \draw [thick]
                            (0.35,0.75) -- node[above]{$\bullet$} ++(0.6,0) ++(0.1,0) -- node[above]{$\bullet$} ++(0.6,0)
                            (0,0) -- node[above]{$\bullet\,\bullet$} ++(0.6,0) ++(0.1,0) -- node[above]{$\bullet\,\bullet$} ++(0.6,0) ++(0.1,0) -- node[above]{$\bullet$} ++(0.6,0)
                        ;
                    \end{scope}
                }\\[1cm]
        
                \rowcolor{grt}
                $E$ & Designates a doubly degenerate asymmetrically occupied state. & \tikz[baseline={(0,0.95)}]{
                    \footnotesize
                    \begin{scope}
                        \draw [thick]
                            (0.35,0.75) -- node[above]{$\bullet$} ++(0.6,0) ++(0.1,0) -- ++(0.6,0)
                            (0,0) -- node[above]{$\bullet$} ++(0.6,0) ++(0.1,0) -- node[above]{$\bullet$} ++(0.6,0) ++(0.1,0) -- node[above]{$\bullet$} ++(0.6,0)
                        ;
                    \end{scope}
                    \begin{scope}[xshift=3cm]
                        \draw [thick]
                            (0.35,0.75) -- node[above]{$\bullet\,\bullet$} ++(0.6,0) ++(0.1,0) -- node[above]{$\bullet$} ++(0.6,0)
                            (0,0) -- node[above]{$\bullet\,\bullet$} ++(0.6,0) ++(0.1,0) -- node[above]{$\bullet\,\bullet$} ++(0.6,0) ++(0.1,0) -- node[above]{$\bullet\,\bullet$} ++(0.6,0)
                        ;
                    \end{scope}
                }\\[1cm]
        
                $A$ or $B$ & Designates a nondegenerate state. Each set of levels in an $A$ or $B$ state is symmetrically occupied. & \tikz[baseline={(0,0.95)}]{
                    \footnotesize
                    \begin{scope}
                        \draw [thick]
                            (0.35,0.75) -- ++(0.6,0) ++(0.1,0) -- ++(0.6,0)
                            (0,0) -- node[above]{$\bullet$} ++(0.6,0) ++(0.1,0) -- node[above]{$\bullet$} ++(0.6,0) ++(0.1,0) -- node[above]{$\bullet$} ++(0.6,0)
                        ;
                    \end{scope}
                    \begin{scope}[xshift=3cm]
                        \draw [thick]
                            (0.35,0.75) -- node[above]{$\bullet$} ++(0.6,0) ++(0.1,0) -- node[above]{$\bullet$} ++(0.6,0)
                            (0,0) -- node[above]{$\bullet$} ++(0.6,0) ++(0.1,0) -- node[above]{$\bullet$} ++(0.6,0) ++(0.1,0) -- node[above]{$\bullet$} ++(0.6,0)
                        ;
                    \end{scope}
                }\\[1cm]
                \noalign{\global\arrayrulewidth=1pt}\arrayrulecolor{grx}\hline
                \noalign{\global\arrayrulewidth=0.4pt}
            \end{tabular}
            \caption{Symmetry labels for split $d$-orbital electron configurations.}
            \label{tab:configurationLabels}
        \end{table}
        \begin{itemize}
            \item To clarify, the definition of $T$ specifies that the triply degenerate state is asymmetrically occupied, the definition of $E$ specifies that the doubly degenerate state is asymmetrically occupied, and the definition $A$ and $B$ specifies that each set of degenerate states is symmetrically occupied.
        \end{itemize}
        \item "The labels of the states resulting from the free-ion term\dots are in reverse order to the labels on the orbitals" \parencite[424]{bib:MiesslerFischerTarr}.
        \begin{itemize}
            \item For example, in the $D_{4h}$ orbitals, the $b_{1g}$ atomic orbital is of highest energy, but the $B_{1g}$ state is of lowest energy.
        \end{itemize}
        \item We see slight distortion of the $d^1$ absorption peak, too, because J-T distortion of the excited state splits the degeneracy of the $e_g$ orbitals.
    \end{itemize}
    \item An important limitation of Tanabe-Sugano diagrams is that they assume $O_h$ symmetry in excited states as well as ground states.
    \begin{itemize}
        \item This can account for differences between their predictions and observed absorption spectra.
    \end{itemize}
    \item Determining $\Delta_o$ from spectra:
    \begin{itemize}
        \item $d^1$, $d^4$ (high spin), $d^6$ (high spin), and $d^9$ configurations:
        \begin{itemize}
            \item There is only a single absorption peak, corresponding to the promotion of an electron from the $t_{2g}$ orbitals to the $e_g$ orbitals.
            \item Thus, $\Delta_o$ is equal to the energy of the absorbed light.
        \end{itemize}
        \item $d^3$ and $d^8$ configurations:
        \begin{itemize}
            \item The ground-state $F$ term splits into three terms: $A_{2g}$, $T_{2g}$, and $T_{1g}$. The difference in energy between the two lowest-energy terms ($A_{2g}$ and $T_{2g}$) is equal to $\Delta_o$.
            \item Thus, $\Delta_o$ is equal to the energy of the light absorbed in the $A_{2g}\to T_{2g}$ transition, which is typically the lowest energy peak in the spectrum.
        \end{itemize}
        \item \textcite{bib:MiesslerFischerTarr} continues for more configurations.
        % Come back here after watching the 3/10 lecture.
    \end{itemize}
    \item States of like symmetry in correlation/Tanabe-Sugano diagrams can mix just like MOs.
    \item On CT bands:
    \begin{itemize}
        \item \ce{MnO4-} is dark purple because of "a strong absorption involving charge transfer from orbitals derived primarily from the filled oxygen $p$ orbitals to empty orbitals derived primarily from the manganese(VII)" \parencite[430]{bib:MiesslerFischerTarr}.
    \end{itemize}
    \item \textbf{Metal  to ligand charge transfer}: Empty $\pi^*$ orbitals on the ligands become the acceptor orbitals on absorption of light. \emph{Also known as} \textbf{MLCT}, \textbf{charge transfer to ligand}, \textbf{CTTL}.
    \item \textbf{Intraligand band}: An absorption band caused by a ligand with a chromophore.
    \item \textcite{bib:MiesslerFischerTarr} discusses energy applications (e.g., solar cells) of charge-transfer bands.
\end{itemize}




\end{document}