\documentclass[../notes.tex]{subfiles}

\pagestyle{main}
\renewcommand{\chaptermark}[1]{\markboth{\chaptername\ \thechapter\ (#1)}{}}
\renewcommand{\thechapter}{\Roman{chapter}}
\setcounter{chapter}{8}

\begin{document}




\chapter{Reactions and Mechanisms}
\section{Module 46: Ligand Substitution in Octahedral Complexes}
\begin{itemize}
    \item \marginnote{3/8:}Suggested reading: Chapter 12.
    \item Having learned about structure, isomerism, and electronic structure, let's now talk about reactivity.
    \item What makes transition metals unique is the reactions of the metal center (i.e., the first coordination sphere).
    \item \textbf{Substitution reaction}: A reaction where a ligand in the first coordination sphere is exchanged.
    \item Factors that affect the rate of substitution:
    \begin{itemize}
        \item Role of the entering group.
        \item Role of the leaving group.
        \item Nature of the other ligands in the complex.
        \item Effect of the metal center.
    \end{itemize}
    \item Stable/unstable are thermodynamic terms, and labile/inert are kinetic terms.
    \begin{itemize}
        \item \ce{[Co(NH3)6]^3+} is unstable but inert wrt. aquation (large $K_\text{eq}$, but slow to react).
        \item \ce{[Ni(CN)4]^2-} is stable but labile wrt. exchange (small $K_\text{eq}$, but equilibrium is established quickly).
    \end{itemize}
    \item \textbf{Potential energy landscape}: A thermodynamic representation of a reaction by a (potentially multidimensional) potential energy surface along its reaction coordinate.
    \item Thermodynamics and kinetics are two different things, but they can be related.
    \item Potential energy landscapes with small $\Delta G$ and large $\Delta E_A$ are slow to react.
    \begin{itemize}
        \item Increasing $\Delta G$ (perhaps by destabilizing the reactant) can speed up the reaction.
        \item Increasing $\Delta G$ often occurs at the expense of $\Delta E_A$ (the activation energy literally decreases as the free energy change increases).
    \end{itemize}
    \item Consider the reaction \ce{[Co(NH3)5X]^2+ + H2O -> [Co(NH3)5(H2O)]^3+ + X-}.
    \begin{itemize}
        \item The more bulky/less electronegative the X-type ligand, the higher the equilibrium constant and reaction rate.
    \end{itemize}
    \item But how do we adequately compare different reactions when so many parameters are entangled together?
    \item One good way to compare kinetics of transition metals is with water exchange rate constants (light/heavy water being exchanged in metal coordination spheres), since this eliminates thermodynamic consternation.
    \begin{itemize}
        \item There is massive variation among the metals.
        \item The fastest (the alkali metals) run up against the \textbf{diffusion limit}.
        \item Metals with larger atomic radii react more quickly (the alkali and alkaline earth metals follow this pattern nicely, but the transition metals are all over the place).
        \begin{itemize}
            \item Group 1A: As we go down the group, the cations are getting larger and the charge density decreases, so the \ce{M+-OH2} bond is getting weaker and more easily broken.
            \item Group 2A: The charge density is larger (doubly charged) so the strength of the bond is greater so the rate of exchange is slower.
        \end{itemize}
        \item We define such a reaction as having a labile half life if the half life is less than 1 minute; otherwise, it is inert.
    \end{itemize}
    \item \textbf{Diffusion limit}: At most, one molecular collision can result in one successful reaction; thus, reaction rate is bounded above by how many molecular collisions can physically occur in a given period of time at a given temperature.
    \item You cannot prove mechanisms; you can only disprove other plausible scenarios. \emph{Also known as} \textbf{diffusion limitation}.
    \item Possible mechanisms for \ce{ML5X + Y -> ML5Y + X} where $\ce{X}=\ce{H2O}$ and $\ce{Y}=\text{anion}$ or vice versa:
    \item Associative ($A$): Via a 7-coordinate intermediate.
    \begin{gather*}
        \ce{ML5X + Y <=>[$k_1$][$k_{-1}$] ML5XY}\tag{slow}\\
        \ce{ML5XY ->[$k_2$] ML5Y + X}\tag{fast}
    \end{gather*}
    \begin{itemize}
        \item Corresponds to organic S\textsubscript{N}2.
        \item First step (RDS) is slow since the incoming \ce{Y} causes steric hindrance.
        \item The transition state is either a monocapped octahedron or pentagonal bipyramidal, depending on how the \ce{Y} attacks.
        \item Applying a steady-state approximation for \ce{[ML5XY]}, we have
        \begin{equation*}
            \dv{\ce{[ML5XY]}}{t} = 0 = k_1\ce{[ML5X][Y]}-k_{-1}\ce{[ML5XY]}-k_2\ce{[ML5XY]}
        \end{equation*}
        \item We can solve the above equation for \ce{[ML5XY]}:
        \begin{equation*}
            \ce{[ML5XY]} = \frac{k_1\ce{[ML5X][Y]}}{k_{-1}+k_2}
        \end{equation*}
        \item Substituting the above into the rate constant equation for the fast step will give us the rate law in terms of the reactants (where $k=\frac{k_1k_2}{k_{-1}+k_2}$):
        \begin{align*}
            \text{Rate} &= k_2\ce{[ML5XY]}\\
            &= \frac{k_1k_2\ce{[ML5X][Y]}}{k_{-1}+k_2}\\
            &= k\ce{[ML5X][Y]}
        \end{align*}
        \item Thus, the S\textsubscript{N}2 mechanism is second-order overall, but first-order in both reactants.
    \end{itemize}
    \item Dissociative ($D$): Via a 5-coordinate intermediate.
    \begin{gather*}
        \ce{ML5X <=>[$k_1$][$k_{-1}$] ML5 + X}\tag{Slow}\\
        \ce{ML5 + Y ->[$k_2$] ML5Y}\tag{Fast}
    \end{gather*}
    \begin{itemize}
        \item Corresponds to organic S\textsubscript{N}1.
        \item First step (RDS) is slow since it's spontaneous elimination of a ligand.
        \item The transition state is generally square pyramidal, but if it is sufficiently long-lived, it can reorganize itself into the trigonal bipyramidal state.
        \item The dissociative mechanism predicts that the rate of the overall substitution reaction depends on only the concentration of  the original complex \ce{[ML5X]}, and is independent of the concentration of the incoming ligand \ce{[Y]}.
        \item Thus, the overall rate law is
        \begin{equation*}
            \text{Rate} = k_1\ce{[ML5X]}
        \end{equation*}
        \item We can also derive this with an analogous kinetic analysis to that used for the $A$ mechanism, the only difference being that we simplify $\frac{k_1k_2\ce{[ML5X][Y]}}{k_{-1}\ce{[X]}+k_2\ce{[Y]}}$ to the rate law by noting that $k_2>>k_{-1}$ or $\ce{[Y]}>>\ce{[X]}$.
    \end{itemize}
    \item Interchange ($I$): As \ce{Y} begins to bond, \ce{X} begins to leave, i.e., the bond making to \ce{Y} and bond breaking to \ce{X} occur simultaneously.
    \begin{gather*}
        \ce{ML5X + Y <=>[$k_1$][$k_{-1}$] ML5X*Y}\tag{Slow}\\
        \ce{ML5X*Y ->[$k_2$] ML5Y + X}\tag{Fast}
    \end{gather*}
    \begin{itemize}
        \item Corresponds to organic S\textsubscript{N}2.
        \item This is how the majority of ligand substitutions occur.
        \item It is too simplistic to assume that a first-order rate law implies $D$ and a second-order rate law implies $A$. Indeed, most substitution reactions probably involve a mechanism like this, i.e., one that is intermediate between these two extremes.
        \item In an interchange mechanism, the intermediate involves an association between the original \ce{ML5X} complex and the attacking \ce{Y} ligand. The \ce{Y} ligand remains outside the coordination sphere of \ce{ML5X}, unlike the S\textsubscript{N}2 mechanism, so the intermediate is not seven coordinate. However, it can help weaken the \ce{M-X} bond.
        \item Assuming high $\ce{[Y]}\approx\ce{[Y]_0}$, it can be shown that the rate is given by
        \begin{equation*}
            \text{Rate} = \frac{k_2K_1\ce{[M]_0[Y]_0}}{1+K_1\ce{[Y]_0}}
        \end{equation*}
        where \ce{[M]_0=[ML5X]_0 + [ML5X*Y]} and \ce{[Y]_0} are initial conditions and $K_1=\frac{k_1}{k_{-1}}$ is the equilibrium constant for the RDS reaction.
        \begin{itemize}
            \item At high \ce{[Y]} and $K_1\ce{[Y]_0}>>1$, the rate is first-order in $\ce{[M]_0}\approx\ce{[ML5X]}$.
            \item At lower \ce{[Y]}, the rate is second order.
        \end{itemize}
    \end{itemize}
    \item At some point, the kinetic analysis becomes essentially worthless. Indeed, although we speak generally about associative and disassociative reaction mechanisms, the terms $A$ and $D$ are reserved for situations where 7- and 5-coordinate intermediates have actually been isolated and positively identified. If no intermediates have been isolated or identified, the designations $I_d$ and $I_a$ are more appropriate.
    \begin{itemize}
        \item Two minor variations on the $I$ mechanism are $I_d$ (dissociative interchange) and $I_a$ (associative interchange).
        \item If breaking the \ce{M-X} bond is more important, the mechanism is $I_d$.
        \item If bond formation between \ce{ML5X} and \ce{Y} is significant, the mechanism is $I_a$.
        \item The difference between $I_d$ and $I_a$ is subtle and does not necessarily correspond to whether the observed rate law is first or second order.
        \item If the rates of a series of comparable substitution reactions are most sensitive to the identity of \ce{X}, the leaving ligand, then the mechanism is more probably $I_d$, and vice versa for $I_a$.
        \item For example, the rate constants for the anation of \ce{[Cr(NH3)5(H2O)]^3+} by various ligands vary very little. Thus, it is probably $I_d$. However, the rate constants for the anation of \ce{[Cr(H2O)6]^3+} by various ligands vary by three orders of magnitude. Thus, it is probably $I_a$.
    \end{itemize}
    \item Solvent (e.g., water) effects in substitution reactions:
    \begin{itemize}
        \item Many substitution reactions occurring in solvent water may have first-order kinetics regardless of whether their initial steps are primarily $D$ or $A$.
        \item For example, this occurs if aquation is a precursor RDS.
        \item If $D$, $\text{Rate}=k\ce{[ML5X]}$.
        \item If $A$, $\text{Rate}=k\ce{[ML5X][H2O]}=k'\ce{[ML5X]}$.
        \begin{itemize}
            \item The latter equality is valid since water has constant concentration\footnote{Is this what we meant when we said in \textcite{bib:APChemNotes} that liquids and solids don't have active mass? In other words, the species should be included in the mass-action expression; they just simply get lumped in with the rate constant generally.}.
        \end{itemize}
        \item Both mechanisms lead to apparent first-order kinetics.
    \end{itemize}
    \item Steric factors favoring $D$ or $I_d$:
    \begin{itemize}
        \item For most octahedral complexes, steric factors inhibit formation of a CN7 intermediate, which suggests a dissociative mechanism ($D$ or $I_d$) is more plausible.
        \begin{itemize}
            \item Even cases showing second-order kinetics may not be $A$ for this reason.
        \end{itemize}
        \item For example, aquation of ammine-halides is second order with a first-order dependence on \ce{[OH-]}.
        \item If $k_2>>k_1$, the rate is approximately $\text{Rate}=\ce{k_2[Co(NH3)5X]^2+[OH-]}$.
        \item Calls for an alternate mechanism called S\textsubscript{N}1CB, where ligands with lower energy bond as \underline{c}onjugate \underline{b}ases and those with higher $\Delta_0$ (hence CB).
    \end{itemize}
    \item Kinetically analyzes S\textsubscript{N}1CB.
\end{itemize}



\section{Module 47: Substitution and Ligand Field Stabilization Energy}
\begin{itemize}
    \item \marginnote{3/10:}Factors that affect the rate of substitution:
    \begin{enumerate}
        \item Role of the entering group.
        \item Role of the leaving group.
        \item The nature of the other ligands in the complex.
        \item Effect of the metal center.
    \end{enumerate}
    \item Six factors that support a dissociative mechanism:
    \begin{enumerate}
        \item The rate of reaction only slightly (within a factor of 10) changes on the incoming ligand.
        \begin{itemize}
            \item Means that the ligand is not significantly involved in the RDS.
        \end{itemize}
        \item Making the charge on complex more positive decreases the rate of substitution.
        \begin{itemize}
            \item Increasing the charge increases the bond strength, making it harder for one ligand to spontaneously break away.
        \end{itemize}
        \item Steric crowding increases reaction rate.
        \item The volume of activation is positive and reaction rate decreases at high pressure.
        \begin{itemize}
            \item Especially if the reactant is a gas, for example.
        \end{itemize}
        \item Reaction rate correlates with Ligand Field Activation Energy (LFAE) predictions.
        \item Stereochemistry can give additional hints.
        \begin{itemize}
            \item A messy subject unless you use very well crafted multidentate ligands.
        \end{itemize}
    \end{enumerate}
    \item Simple changes in oxidation state can have massive effects on water exchange rate constants\footnote{We pronounce the units of rate constants, "reciprocal seconds."} (e.g., \ce{Cr^2+} to \ce{Cr^3+} results in a shift of 15 orders of magnitude).
    \item Variations in complex reactivity:
    \begin{itemize}
        \item Most first row transition metals are labile\footnote{Werner was only able to make his conclusions about coordination chemistry because he happened to be working with the relatively inert \ce{Co^3+} complexes.}, but \ce{Cr^3+} ($d^3$, $S=\frac{3}{2}$) and low-spin \ce{Co^3+} ($d^6$, $S=0$) are usually inert.
        \item $d^7$-$d^{10}$, with filling of $e_g^*$ levels, are labile.
        \begin{itemize}
            \item These configurations tend to have large Jahn-Teller distortions and/or low CFSEs.
            \item $d^7$, $d^9$, and $d^{10}$ cases are more labile than $d^8$.
            \item $d^8$ has a ${}^3A_{2g}$ ground state, which is immune to Jahn-Teller distortion.
            \item With strong-field ligands, $d^8$ may be square planar, often being inert.
        \end{itemize}
        \item Inert octahedral complexes tend to be those with high CFSE; viz., $d^3$ low-spin $d^4$-$d^6$.
        \begin{itemize}
            \item \ce{ML6} complexes of both $d^3$ (${}^4A_{2g}$) and low-spin $d^6$ (${}^1A_{1g}$) are immune from Jahn-Teller distortions and therefore can be perfect $O_h$.
            \item \ce{ML6} $d^3$ has $\text{CFSE}=-\frac{6}{5}\Delta_o$ and low-spin $d^6$ has $\text{CFSE}=-\frac{12}{5}\Delta_o+3P$.
        \end{itemize}
    \end{itemize}
    \item The most inert complexes among the first row transition metals:
    \begin{itemize}
        \item \ce{Co^3+} is primarily dissociative, but with dependence on the incoming ligand $I_d$.
        \item \ce{Cr^3+} has a dimorphism where it can be $I_d$ or $I_a$!
        \begin{itemize}
            \item For example, it is $I_d$ in the complex \ce{[Cr(NH3)5(H2O)]^3+} but $I_a$ in the complex \ce{[Cr(H2O)6]^3+}.
        \end{itemize}
    \end{itemize}
    \item Water exchange rate constants are greatest for Jahn-Teller distorted species.
    \begin{itemize}
        \item They are smallest for $d^{3,5,8}$.
    \end{itemize}
    \item Variations in complex reality: Explained by the potential energy landscape for a substitution reaction \ce{ML6 -> ML5Y}:
    \begin{itemize}
        \item $O_h$ to $C_{4v}$ to "$O_h$."
        \item The activated complex is square pyramidal.
        \item We can now use the AOM to calculate the LFSE in both the initial $O_h$ state and the transition $C_{4v}$ state.
        \begin{itemize}
            \item The difference between them will be the \textbf{ligand field activation energy} or \textbf{LFAE}. The LFAE is not particularly useful because it predicts negative activation energies, which are not a thing, but it does allow us to rationalize reactivity trends.
            \item If $\text{LFAE}\leq 0$, the reaction is labile; otherwise, it is moderate or slow.
        \end{itemize}
    \end{itemize}
\end{itemize}



\section{Module 48: Ligand Substitution in Square Planar Complexes}
\begin{itemize}
    \item Although steric factors favor $D$-type mechanisms for octahedral complexes, square planar \ce{ML4} complexes are not so inhibited.
    \begin{itemize}
        \item For square planar \ce{ML4} complexes, an associative ($A$) mechanism, in which a coordination number 5 (CN5) intermediate is formed, is plausible.
    \end{itemize}
    \item The $d^8$ metals are especially likely to form square planar complexes.
    \item Substitution of square planar complexes, such as \ce{PtLX3}, leads to \emph{trans} and \emph{cis} isomers.
    \item The rate law is
    \begin{equation*}
        \text{Rate} = k_1\ce{[PtLX3]}+k_2\ce{[PtLX3][Y^-]}
    \end{equation*}
    \begin{itemize}
        \item Suggests two paths, where the first term may be pseudo-first-order due to excess solvent acting as an attacking group.
    \end{itemize}
    % \item Example:
    % \begin{figure}[h!]
    %     \centering
    %     \schemestart[][west]
    %         \chemfig{Pt(-Cl)(-[2]PEt_3)(-[4]L)(-[6]PEt_3)}
    %         \arrow{0}[,0]\+{1em,1em,2em}
    %         \chemfig{[:60]N**6(------)}
    %         \arrow
    %         \chemleft{[}
    %             \chemfig{Pt(-N**6(------))(-[2]PEt_3)(-[4]L)(-[6]PEt_3)}
    %         \chemright{]^+}
    %         \+{1em,1em}
    %         \chemfig{\ce{Cl-}}
    %     \schemestop
    %     \caption{Square-planar substitution reaction.}
    %     \label{fig:squarePlanarSubstitution}
    % \end{figure}
    % \begin{itemize}
    \item If we make \ce{L} bulkier, we would expect the $D$ pathway to dominate ($k_1$ increases, $k_2$ decreases).
    \item However, both $k_1$ and $k_2$ decrease with increasingly bulky ligands, suggesting an associative mechanism:
    \begin{gather*}
        \ce{ML4 + Y <=>[$k_2$] [ML4Y]^{$\ddagger$} ->[][fast] ML3Y + L}\tag{$k_2$ term}\\
        \ce{ML4 + S ->[$k_1$] [ML4S]^{$\ddagger$} -> [ML3S] + L ->[Y][fast] ML3Y + S}\tag{$k_1$ term}
    \end{gather*}
    \begin{itemize}
        \item \ce{S} means the solvent.
        \item Since $\ce{[S]}>>\ce{[ML4]}$, we see pseudo-first order kinetics in the $k_1$ term in the rate law.
    \end{itemize}
    % \end{itemize}
    \item In square planar complexes, $d$-orbital splitting yields a $d_{x^2-y^2}$ LUMO.
    \begin{itemize}
        \item However, it points exactly toward the ligands, so you cannot fill it without displacing a ligand.
        \item Indeed, since ligands cannot attack it, they instead attack the higher-lying metal $p_z$ orbital.
    \end{itemize}
    \item Note that the stereochemistry stays the same under these reactions.
    \item An example of the trans effect:
    \begin{gather*}
        \ce{[Pt(NH3)4]^2+ ->[2Cl-] \emph{trans}-[Pt(NH3)2Cl2]}\\
        \ce{[Pt(Cl)4]^2- ->[2NH3] \emph{cis}-[Pt(NH3)2Cl2]}
    \end{gather*}
    \begin{itemize}
        \item The first ligand takes any of the four spots, but the second substitutes opposite the stronger \emph{trans}-directing ligand.
        \item In the above example, chloride is a stronger \emph{trans}-directing ligand than amine groups.
    \end{itemize}
    \item The trans effect:
    \begin{itemize}
        \item The ratio of \emph{trans} and \emph{cis} isomers is found to vary with the ability of \ce{L} to act as a \emph{trans}-directing ligand.
        \item The increasing order of \emph{trans}-directing ability is
        \begin{equation*}
            \ce{H2O}<\ce{OH-}<\text{py}\approx\ce{NH3}<\ce{Cl-}<\ce{Br-}<\ce{I-}<\ce{NO2-}<\ce{PR3}\approx\ce{SH2}<<\ce{CO}\approx\ce{C2H4}\approx\ce{CN-}
        \end{equation*}
        \begin{itemize}
            \item This ranking comes from a combination of the following two factors.
        \end{itemize}
        \item The effect is kinetic rather than thermodynamic.
    \end{itemize}
    \item Two factors explain the trans effect (both aim at lowering $\Delta G^\ddagger$):
    \begin{enumerate}
        \item Weakening of the \ce{Pt-X} bond \emph{trans} to the directing ligand ($\sigma$-donor effects).
        \begin{itemize}
            \item Destabilizes the ground state of the reactants.
            \item \ce{Pt-A} is influenced by \emph{trans}-\ce{Pt-T} bond because both share \ce{Pt} $p_x$ and $d_{x^2-y^2}$ orbitals. When the \ce{Pt-T} bond is strong, electron density on those orbitals is shifted away from the \ce{Pt-A} bond.
            \item The more polarizable the \ce{L} ligand, the better \emph{trans}-director it is; e.g., $\ce{I-}>\ce{Br-}>\ce{Cl-}$.
            \begin{itemize}
                \item The \emph{trans}-directing ligand polarizes the metal ion, inducing a slight repulsion with the negative electron density on the leaving ligand in the \emph{trans}-position.
            \end{itemize}
            \item For example, the \ce{Pt-Cl} bond is longer in in \emph{cis}-\ce{[Pt(PMe3)2(Cl)2]} than in the \emph{trans} form because the stronger $\sigma$-donor ability of \ce{PMe3} weakens the opposite \ce{Pt-Cl} bond in the \emph{cis} form.
        \end{itemize}
        \item Stabilization of the presumed CN5 intermediate ($\pi$-acceptor effects).
        \begin{itemize}
            \item Stabilizes the activated complex.
            \item The strongest \emph{trans}-directors are good $\pi$-acceptor ligands.
            \item Assuming an $A$ mechanism, substitution involves a trigonal bipyramidal transition state.
            \item The \emph{trans} intermediate (activated complex) is more favorable for $\pi$-acceptor ligands because it permits $\pi$-delocalization in the trigonal plane.
            \begin{itemize}
                \item When the \ce{T}-ligand engages in a strong $\pi$-backbonding, charge is removed from \ce{Pt}, making the metal center more electrophilic and stabilizing the TBP intermediate.
            \end{itemize}
        \end{itemize}
    \end{enumerate}
    \item The trans effect has applications to synthesis (synthesizing \emph{cis} versus \emph{trans} square planar compounds).
    \begin{itemize}
        \item For example, \emph{cis}-\ce{[Pt(NH3)2(Cl)2]} is cisplatin, the first anti-cancer drug.
    \end{itemize}
\end{itemize}



\section{Module 49: Redox Reactions and Marcus Theory}
\begin{itemize}
    \item \marginnote{3/12:}Considers electron transfer (REDOX reaction) of \ce{[Cr(H2O)5Cl]^3+ + Cr^2+ -> Cr^3+ + [Cr(H2O)5Cl]^2+}.
    \item Changing the ligands in the coordination sphere affects the rate constant, sometimes dramatically.
    \item \textbf{Self-exchange reaction}: A REDOX reaction of a compound with itself.
    \begin{itemize}
        \item These are good model systems, eliminating other thermodynamic factors the same way water exchange reactions do.
    \end{itemize}
    \item If we run a self-exchange reaction, radiolabel the coordination centers of one oxidation state, and precipitate the centers of one of the oxidation states after some amount of time, we can perform analyses on how much of what center precipitated (the ratio of labeled \ce{M^n+} to \ce{M^m+}).
    \item Conclusions (wrt. \ce{{}^*Fe^{III}(ClO4)3 + Fe^{II}(ClO4)2 -> {}^*Fe^{II}(ClO4)2 + Fe^{III}(ClO4)3}\footnote{The starred iron is radiolabeled, i.e., a heavy, radioactive isomer.}):
    \begin{enumerate}
        \item Reaction half life is approximately 20 seconds.
        \item Reaction order is 2.
        \item The free energy of activation is approximately $\SI{33}{\kilo\joule\per\mole}$.
        \begin{itemize}
            \item This is derived by recording the rate constant at different temperatures and extracting $\Delta G^\ddagger$ by running the data through the Arrhenius equation.
        \end{itemize}
    \end{enumerate}
    \item Rationalizing the results:
    \begin{itemize}
        \item \ce{Fe^2+} and \ce{Fe^3+} are different in size and bond length.
        \item Electron transfer relaxes bond length but costs energy, making the transition state much greater than zero.
        \begin{itemize}
            \item Essentially, the electron transfer (assumed instantaneous) will destabilize the system, so the solvent will have to reorganize around the coordination complex to lower the energy.
            \item Solvent molecules will become random around the reduced complex and oriented around the oxidized one.
        \end{itemize}
    \end{itemize}
    \item \textbf{Reorganization energy}: The key parameter to understanding electron transfer kinetics. \emph{Also known as} $\bm{\lambda}$.
    \begin{equation*}
        \lambda = \lambda_\text{inner}+\lambda_\text{outer}
    \end{equation*}
    \begin{itemize}
        \item $\lambda_\text{inner}$ is the reorganization of bond lengths, angles, the spin state, etc. within the coordination compound.
        \item $\lambda_\text{outer}$ is the reorganization of the solvent molecules.
        \item $\lambda$ is used in many areas of chemistry (it's a foundational concept).
    \end{itemize}
    \item An example of $\lambda_\text{inner}$:
    \begin{itemize}
        \item \ce{Co^2+ -> Co^3+} converts from high spin $d^7$ to low spin $d^6$; since this is significant reorganization of the metal center, self-exchange is slow.
        \item \ce{Ru^2+ -> Ru^3+} converts from low spin $d^7$ to low spin $d^6$; since this is minor reorganization of the metal center, self-exchange is fast.
    \end{itemize}
\end{itemize}



\section{Module 50: Outer- and Inner-Sphere Electron Transfers}
\begin{itemize}
    \item Henry Taube (UChicago) built the foundation for inner-sphere electron transfers.
    \item Rudy Markus (CalTech) built the foundation for outer-sphere electron transfers.
    \item The activation energy for a redox reaction is $\Delta G^\ddagger=\Delta G^\ddagger_t+\Delta G^\ddagger_i+\Delta G^\ddagger_s$, where
    \begin{itemize}
        \item $\Delta G^\ddagger_t$ relates to bringing molecules close;
        \item $\Delta G^\ddagger_i$ relates to vibrational energy (reorganization);
        \item $\Delta G^\ddagger_s$ relates to solvent reorganization.
    \end{itemize}
    \item Markus studies these reactions in terms of weak electronic interactions, where there is no mixing of the reactant and product states.
    \begin{itemize}
        \item No mixing, molecular orbitals, etc.; instead, electrons transfer from one well-defined state in one molecule to another in another molecule.
    \end{itemize}
    \item Under these terms, electron transport (to other molecules) can be formulated in terms of Fermi's Golden Rule.
    \begin{equation*}
        k_\text{et} \propto \rho[\mel{\Psi_1}{P_{1\to 2}}{\Psi_2}]^2
    \end{equation*}
    \begin{itemize}
        \item Relating back to the Arrhenius equation, $\rho$ describes the free energy of activation using the \textbf{Franck-Condon principle}.
    \item In the Arrhenius equation, $A$ is a pre-exponential factor that combines attempt frequency (approximately $\SI{e13}{\per\second}$ if no reorganization occurs) and $x$-transmission coefficient (the probability that the system crosses the activation barrier as it approaches the top).
    \end{itemize}
    \item \textbf{Franck-Condon principle}: If we have two levels near in energy, they will randomly fluctuate and match up in energy every once in a while; electron transfers occur when they match up.
    \item We now consider the kinetics of the transfer from the reaction coordinate perspective.
    \item Potential energy curves of reactants and products by parabolas.
    \begin{figure}[H]
        \centering
        \begin{tikzpicture}
            \small
            \draw [<->] (8,0) -- node[below]{Reaction coordinate} (0,0) -- node[left]{$\Delta G^\circ$} (0,7);
    
    
            \footnotesize
            \draw [grx,thick]
                (0.5,6.5) parabola bend (3,2) (5.5,6.5)
                (2.5,6.5) parabola bend (5,2) (7.5,6.5)
            ;
            \node at (0.7,5) {R};
            \node at (7.3,5) {P};
    
            \draw [semithick,-stealth] (3,2) -- node[left]{$\lambda$} (3,4.88);
            
            \draw [semithick,-latex] (3.15,4.88) to[out=-72,in=117] (3.8,3.3);
            \draw [semithick,-latex] (3.1,2.15) to[out=10,in=-130] (3.6,2.46);
            \draw [semithick,-latex] (3.7,2.55) to[out=45,in=180,in looseness=0.7] (4,2.92) to[out=0,in=135,out looseness=0.7] (4.3,2.55);
            \draw [semithick,-stealth] (4.4,2.46) to[out=-50,in=170] (4.9,2.15);
            \node at (5.3,3.5) {Markus} edge [semithick,-stealth] (4.2,2.8);
            \node [above=2mm] at (4,2.72) {$\ddagger$};
    
            \begin{scope}[on background layer]
                \draw [very thin]
                    (4,2.72) -- (7.5,2.72)
                    (3,2) -- (7.5,2)
                ;
                \draw [semithick,stealth-stealth] (7,2) -- node[right]{$\Delta G^\ddagger$} (7,2.72);
    
                \draw [very thin]
                    (3,1.6) -- node[left]{Reorganization} ++(0,0.2)
                    (4,1.6) -- ++(0,0.2)
                    (4,0.8) -- ++(0,0.2)
                    (5,0.8) -- node[right]{Relaxation} ++(0,0.2)
                ;
                \draw [semithick,stealth-stealth] (3,1.7) -- (4,1.7);
                \draw [semithick,-stealth] (4,1.5) -- node[right]{$\kappa$} (4,1.1);
                \draw [semithick,stealth-stealth] (4,0.9) -- (5,0.9);
            \end{scope}
        \end{tikzpicture}
        \caption{Potential energy description of an electron-transfer reaction with $\Delta G^\circ=0$.}
        \label{fig:potentialEnergyParabolas}
    \end{figure}
    \begin{itemize}
        \item Parabolas come from the representation of potential energy as a function of separation for a harmonic oscillator (if everything's oscillating, classical mechanics predicts a parabolic energy curve $\frac{1}{2}kx^2$).
        \item The $x$-axis (reaction coordinate) represents the change in the geometry of the reactants and products.
    \end{itemize}
    \item Lieche predicts that the activation energy will be the vertical jump $\lambda$ from the reactants parabola to the products parabola, and then the energy will fall back down.
    \begin{itemize}
        \item Conversely, Markus asserts that the reactants will reorganize first to the optimal transfer point (where both parabolas process). This reduces $\Delta G^\ddagger$ to $\frac{1}{4}\lambda$.
    \end{itemize}
    \item Important parameters needed to describe $k_\text{et}$.
    \begin{enumerate}
        \item Reorganization energy $\lambda$ corresponds to the vertical transition from parabola $R$ to parabola $P$.
        \item Free energy of activation $\Delta G^\ddagger$.
        \item Free energy of reaction $\Delta G^\circ$.
    \end{enumerate}
    \item Bounds on $\lambda$:
    \begin{itemize}
        \item $\lambda$ is usually on the order of $1.0$-$\SI{1.5}{eV}$ for isoenergetic reactions.
        \item $\Delta G^\ddagger=\frac{\lambda}{4}$.
    \end{itemize}
    \item If we have a thermodynamic driving factor for an electron-transfer reaction, we simply either raise the R parabola or lower the P parabola (see Figure \ref{fig:potentialEnergyParabolas}).
    \item Predictions for $\Delta G^\ddagger$ based on $\Delta G^\circ$ and the parabola model (see Figure \ref{fig:parabolas-droppingP}):
    \begin{figure}[h!]
        \centering
        \begin{subfigure}[b]{0.24\linewidth}
            \centering
            \begin{tikzpicture}[scale=0.5]
                \path (0,-6) -- (0,2);
    
                \draw [grx,thick,name path=R] plot[domain=-2.41:0.41,smooth] (\x,{(\x+1)^2});
                \draw [grx,thick,name path=P] plot[domain=-0.41:2.41,smooth] (\x,{(\x-1)^2});
                \fill [name intersections={of=R and P}] (intersection-1) circle (5pt);
            \end{tikzpicture}
            \caption{$\Delta G^\circ=0$.}
            \label{fig:parabolas-droppingPa}
        \end{subfigure}
        \begin{subfigure}[b]{0.24\linewidth}
            \centering
            \begin{tikzpicture}[scale=0.5]
                \path (0,-6) -- (0,2);
    
                \draw [grx,thick,name path=R] plot[domain=-2.41:0.41,smooth] (\x,{(\x+1)^2});
                \draw [grx,thick,name path=P] plot[domain=-1:3,smooth] (\x,{(\x-1)^2-2});
                \fill [name intersections={of=R and P}] (intersection-1) circle (5pt);
            \end{tikzpicture}
            \caption{$-\lambda<\Delta G^\circ<0$.}
            \label{fig:parabolas-droppingPb}
        \end{subfigure}
        \begin{subfigure}[b]{0.24\linewidth}
            \centering
            \begin{tikzpicture}[scale=0.5]
                \path (0,-6) -- (0,2);
    
                \draw [grx,thick,name path=R] plot[domain=-2.41:0.41,smooth] (\x,{(\x+1)^2});
                \draw [grx,thick,name path=P] plot[domain=-1.45:3.45,smooth] (\x,{(\x-1)^2-4});
                \fill [name intersections={of=R and P}] (intersection-1) circle (5pt);
            \end{tikzpicture}
            \caption{$-\lambda=\Delta G^\circ$.}
            \label{fig:parabolas-droppingPc}
        \end{subfigure}
        \begin{subfigure}[b]{0.24\linewidth}
            \centering
            \begin{tikzpicture}[scale=0.5]
                \path (0,-6) -- (0,2);
    
                \draw [grx,thick,name path=R] plot[domain=-2.41:0.41,smooth] (\x,{(\x+1)^2});
                \draw [grx,thick,name path=P] plot[domain=-1.83:3.83,smooth] (\x,{(\x-1)^2-6});
                \fill [name intersections={of=R and P}] (intersection-1) circle (5pt);
            \end{tikzpicture}
            \caption{$\Delta G^\circ<-\lambda$.}
            \label{fig:parabolas-droppingPd}
        \end{subfigure}
        \caption{Effect of $\Delta G^\circ$ and $\lambda$ on $\Delta G^\ddagger$.}
        \label{fig:parabolas-droppingP}
    \end{figure}
    \begin{itemize}
        \item $\Delta G^\circ=0$ implies $\Delta G^\ddagger=\frac{\lambda}{4}$.
        \item $-\lambda<\Delta G^\circ<0$ implies $\Delta G^\ddagger$ is decreasing (this can be rationalized by normal intuition; it makes sense that as $\Delta G^\circ$ decreases, $\Delta G^\ddagger$ would decrease, too).
        \item $-\lambda=\Delta G^\circ$ implies $\Delta G^\ddagger=0$.
        \item $\Delta G^\circ<-\lambda$ implies $\Delta G^\ddagger>0$.
    \end{itemize}
    \item Under the assumption that the shapes of the curves do not change, we have
    \begin{equation*}
        \Delta G^\ddagger = \frac{(\Delta G^\circ+\lambda)^2}{4\lambda}
    \end{equation*}
    \item The above expression can be plugged into the formula for the kinetic rate constant,
    \begin{equation*}
        k_\text{et} = Ax\exp\left( \frac{-(\Delta G^\circ+\lambda)^2/(4\lambda)}{RT} \right)
    \end{equation*}
    where
    \begin{itemize}
        \item $A$ is the attempt frequency (the Arrhenius coefficient).
        \item $x$ is the transmission coefficient.
    \end{itemize}
    \item Special cases of $\Delta G^\circ$:
    \begin{itemize}
        \item Self-exchange: $\Delta G^\circ=0$.
        \item Barrierless region: $\Delta G^\circ=-\lambda$.
        \item "Inverted" region: $\Delta G^\circ<-\lambda$.
    \end{itemize}
    \item Experimental verification of Markus theory:
    \begin{itemize}
        \item This very counterintuitive implication (that in the inverted region, increases in $\Delta G^\circ$ actually slow the reaction down) needed verification.
        \item Early attempts faced difficulties because of the diffusion limitation (increasing exothermicity causes $\log K$ to hit a ceiling).
        \item Class and Miller (UChicago) eventually found a reaction (intramolecular electron transfer) that could verify Markus theory (later observed in many other photo and electrochemical systems).
    \end{itemize}
    \item Back to outer- vs. inner-sphere electron transfer:
    \begin{itemize}
        \item Markus theory discusses outer-sphere electron transfer, and cannot explain why in some cases Redox reaction rates vary so greatly.
        \item Taube and Halpern (UChicago) posit that some ligands (i.e., ones with multiple lone pairs or low lying antibonding orbitals) can form a bridge between two metal centers, enabling inner-spherre electron transfer.
    \end{itemize}
    \item Example: \ce{[Co(NH3)5X]^n+} / \ce{Cr^2+} in aqueous medium:
    \begin{itemize}
        \item Since chromium is $d^4$ (hence labile), \ce{[Cr(H2O)6]^2+ <=>[][fast] [Cr(H2O)5]^2+ + H2O}.
        \item The undercoordinated chromium ligand can engage in an inner-sphere electron transfer with the chloro ligand: \ce{[(H3N)5Co-Cl-Cr(H2O)5]^4+ ->[e.t.] [(H3N)5Co]^2+ + [ClCr(H2O)5]^3+}.
        \item Lastly, we hydrolize cobalt: \ce{[Co(NH3)5]^2+ -> [Co(H2O)6]^2+ + NH3}.
    \end{itemize}
    \item In the transition state with two metals bridged by a ligand, the reactant and product states mix, forming two \textbf{adiabatic} states along the reaction coordinate.
    \item \textbf{Adiabatic} (states): Two state of equal energy.
    \item State mixing ($H_{ab}=\mel{\Psi_a}{\hat{H}}{\Psi_b}>0$):
    \begin{itemize}
        \item With weak electron transfer, we have R and P parabolas.
        \item As electron transfer increases in strength, the "inner loop" in Figure \ref{fig:potentialEnergyParabolas} separates and rises above the bottom loop.
        \item The bottom loop is stabilizing and bonding; vice versa for the upper one.
    \end{itemize}
    \item There exist compounds that can capture the charge-delocalized state.
\end{itemize}



\section{Office Hours (Talapin)}
\begin{itemize}
    \item \marginnote{3/16:}In Problem Set 6 problem IIId, the answer key lists chromium in \ce{Cr(CO)5} as having a $d^6$ electron configuration. Why? When exactly do the $s$ electrons get used in bonding?
    \begin{itemize}
        \item Because when chromium is chemically bound, it's energetically unfavorable to have a large $4s$ orbital pushing the bounds of the atom.
        \item Any transition metal in a compound with "remaining" $s$ electrons drops those electrons to the $(n-1)d$ orbitals.
    \end{itemize}
    \item Did we ever use similarity transformations (Notes pg. 32)?
    \item What do we need to know about quantum mechanics, the Huckel approximation, LCAO theory, Fermi's golden rule, etc. from Modules 11-12 (Notes pg. 48-49)? Calculations with overlap integrals and energy things and bra-ket notation?
    \item LCAO vs. SALC?
    \begin{itemize}
        \item SALCs are a subset of all possible linear combinations.
        \item To be a solution to the Schr\"{o}dinger equation, it must be a SALC.
    \end{itemize}
    \item Choosing subgroups when constructing MOs for molecules like \ce{HF}? You told us you'd specify later. Is it always just $C_{2v}$ or $D_{2h}$?
    \item Do we need to know how to create Walsh diagrams?
    \item Is it not a bit circular to assign the point group $C_{3v}$ and then later use Walsh diagrams to determine that it has a $C_{3v}$ structure?
    \item Can you go over the angular overlap model?
    \begin{itemize}
        \item The main idea is you account for each ligand's energy contribution individually.
        \item Summing columns with values for appropriate ligands allows you to get the relative $d$-orbital energies.
        \item Summing a row allows you to get the energy for a ligand in a particular position.
        \item Once we order the energies, we can use the appropriate character table to assign Mulliken symbols.
        \item To be more quantitative, we can use the tabulated angular overlap parameters (wavenumbers).
        \item $d$ orbitals are stabilized by $\pi$-accepting ligands, and destabilized by $\pi$-donating ligands?
        \item Square pyramidal has $\pi$-donor ligands in Module 36 slide 15.
        \item \textbf{Lifting degeneracy}: Splitting degenerate orbitals.
    \end{itemize}
    \item What is $k$ and $k$-space in band theory?
    \begin{itemize}
        \item $k$ has many definitions.
        \item $k$ loses its ability to describe momentum past the first Brillouin zone.
        \item Essentially, it's either a label of a crystal orbital or it's correlated with momentum.
    \end{itemize}
    \item Can you explain Fermi-Dirac statistics?
    \item Module 42: What do the brackets in decompositions of direct products mean?
    \item What is reducing a microstate table?
    \begin{itemize}
        \item Some different microstates will have the same energy.
        \item Degenerate states split into multiple states.
        \item We have to cross out one $x$ in the appropriate cells, but not the specific ones he did.
    \end{itemize}
    \item What kind of rate law/kinetics stuff do we need?
    \item Hydrolyze:
    \begin{itemize}
        \item An ion breaking water apart; Hard \ce{Al^3+} maximally hydrolyzes with hard \ce{OH-}.
    \end{itemize}
    \item They ask how many of what type of orbitals?
    \begin{itemize}
        \item It can't hurt to draw them out.
        \item Also, don't draw molecular orbitals as molecular orbitals; draw atomic orbitals and \emph{maybe} lines between corresponding lobes.
    \end{itemize}
    \item Vibronic coupling:
    \begin{itemize}
        \item Be aware of rotational coupling.
    \end{itemize}
    \item When drawing an MO diagram, draw the \emph{full} diagram.
    \item Matching colors to transitions:
    \begin{itemize}
        \item Pale $\Rightarrow$ spin-forbidden.
        \item $\pi$-acceptor ligand $\Rightarrow$ \ce{M -> L}.
        \item $d^0$ $\Rightarrow$ \ce{L -> M}.
        \item Misc: $d$-$d$ transition.
    \end{itemize}
    \item Low spin $d^3$ and $d^6$ are extra inert because their $t_{2g}$ sets exhibit half-filled stability.
    \item To synthesize \emph{cis}-\ce{[PtI2(py)CN)]-} given that the trans effect order is $\ce{CN-}>\ce{I-}>\text{py}$, add two measures of py to create a cis compound and then one measure of cyanide to substitute one of the newly added py ligands.
    \begin{itemize}
        \item This gives us a higher yield than just one measure of each by cutting down on production of the \emph{trans} product.
    \end{itemize}
    \item Associative mechanisms can be faster than dissociative.
    \item Cisplatin is a carcinogenic anti-cancer drug that works by binding to amine groups on the DNA double helix.
\end{itemize}



\section{Chapter 12: Coordination Chemistry IV (Reactions and Mechanisms)}
\emph{From \textcite{bib:MiesslerFischerTarr}.}
\begin{itemize}
    \item \marginnote{3/11:}\textbf{Transition-state theory}: A theory describing chemical reactions as moving from one energy minimum (the reactants) through higher energy structures (transition states, intermediates) to another energy minimum (the products).
    \item \textbf{Principle of microscopic reversibility}: The lowest energy pathway going in one direction must also be the lowest energy pathway going in the opposite direction.
    \begin{itemize}
        \item "Although the complexity of reaction coordinate diagrams can vary widely, the adopted path between the reactants and the products is always the lowest energy pathway available and must be the same regardless of the direction of the reaction" \parencite[437-38]{bib:MiesslerFischerTarr}.
    \end{itemize}
    \item \textbf{Steady-state approximation}: The concentration of the intermediate is assumed to be extremely small and essentially unchanging during much of the reaction.
    \begin{itemize}
        \item Allowed by the presence of undetectable intermediates.
    \end{itemize}
    \item \textbf{Order} (of a reactant): The power of the reactant concentration in the differential equation that describes how its concentration changes with time, which indicates how the reaction rate is tied to a change in that reactant's concentration.
    \item \textbf{Rate constant}: A proportionality constant that relates the reaction rate to the concentration of the reactants, which is temperature dependent.
    \item The \textbf{free energy of activation} can be divided into two components: \textbf{enthalpy of activation} and \textbf{entropy of activation}.
    \item \textbf{Volume of activation}: A quantity that offers insight into whether the transition state is larger or smaller than the reactants, derived from examinations of pressure dependence on reaction rates.
    \item The rate of reaction depends on the activation energy via the Arrhenius equation
    \begin{equation*}
        k = A\e[-\frac{E_A}{RT}]
    \end{equation*}
    \item \textcite{bib:MiesslerFischerTarr} introduces substitution reactions of \ce{[M(H2O)_m]^n+} and color-based identification of species.
    \begin{itemize}
        \item Note that \ce{[V(H2O)6]^3+} has a higher water exchange rate constant than \ce{[V(H2O)6]^2+}, despite the fact that we might expect it to hold onto its ligands more tightly because of its higher oxidation state.
    \end{itemize}
    \item \textbf{Labile} (compound): Compounds which react rapidly, essentially exchanging one ligand for another within the time of mixing the reactants. \emph{Also known as} \textbf{kinetically labile}.
    \begin{itemize}
        \item Very low activation energy for ligand substitution.
        \item Examples include $d^1$, $d^2$, and high-spin $d^4$ through $d^6$ compounds, as well as $d^7$, $d^9$, and $d^{10}$ ones.
    \end{itemize}
    \item \textbf{Inert} (compound): A compound that does not resist ligand substitution but is simply slower to react. \emph{Also known as} \textbf{kinetically inert}.
    \begin{itemize}
        \item "Inert octahedral complexes are generally those with high ligand field stabilization energies\dots specifically those with $d^3$ or low-spin $d^4$ through $d^6$ electronic structures" \parencite[440-41]{bib:MiesslerFischerTarr}.
        \item Strong ligand-field $d^8$ complexes often form inert square-planar complexes.
    \end{itemize}
    \item \textbf{Stoichiometric mechanism}: Any one of the substitution reaction categories ($D$, $A$, or $I$).
    \item \textbf{Intimate mechanism}: The distinction between activation processes that are associative and dissociative.
    \item A word on notation: In this chapter's reactions, "\ce{X} will indicate the ligand that is leaving a complex, \ce{Y} the ligand that is entering, and \ce{L} any ligands that are unchanged during the reaction. In cases of solvent exchange, the \ce{X}, \ce{Y}, and \ce{L} may be the same species. Charges will be omitted when using \ce{X}, \ce{Y}, and \ce{L}, but the species may be ions" \parencite[441]{bib:MiesslerFischerTarr}.
    \item In a rate law, Rate equals the change in the concentration of the product with respect to time, $\dv*{[\ce{P}]}{t}$.
    \item We assume the formation of an \textbf{ion pair} or \textbf{preassociation complex} in our description of an interchange reaction, instead of just assuming that the interchange occurs in one step.
    \item Ion pairs form from ionic reactions while preassociation complexes form from dipole-dipole interactions.
    \item Substitution reaction intermediates are detectable when $k_1,k_{-1}>>k_2$ since this allows the first (reversible) reaction to reach an equilibrium independent of the second.
    \begin{itemize}
        \item Determining the concentration of \ce{[ML5X*Y]} facilitates the calculation of $K_1$. Otherwise, it must be estimated theoretically.
    \end{itemize}
    \item Interchange mechanisms involving preassociation complexes can be difficult to distinguish from $D$ mechanisms when \ce{[Y]} is large since they both tend toward first-order kinetics.
    \item \marginnote{3/14:}\textbf{Ligand field activation energy}: The difference between the LFSE of the square-pyramidal transition state and the LFSE of the octahedral reactant. \emph{Also known as} \textbf{LFAE}.
    \begin{itemize}
        \item LFAEs for trigonal bipyramidal transition states are usually the same or larger than those for square pyramidal ones.
    \end{itemize}
    \item Other factors that influence reaction rate:
    \begin{itemize}
        \item Oxidation state of the central ion: Higher oxidation states lead to slower ligand exchange.
        \item Ionic radius: Smaller ions have slower exchange rates.
    \end{itemize}
    \item \textbf{Aquation}: Substitution by water.
    \item \textbf{Anation}: Substitution by an anion.
    \item \textbf{Linear free-energy relationship}: A relation between kinetic effects and thermodynamic effects. \emph{Also known as} \textbf{LFER}.
    \begin{itemize}
        \item Observed when the bond strength of a metal-ligand bond (thermodynamic parameter) plays a major role in determining the dissociation rate of a ligand (kinetic parameter).
        \item Observed when a plot of the logarithm of the rate constants for \ce{[ML5X]^n+ + Y} substitution reactions, where \ce{X} is varied but \ce{Y} is not, versus the logarithm of the equilibrium constants for \ce{[ML5X]^n+ + Y <=> [ML5Y]^m+ + X} is linear.
    \end{itemize}
    \item Stronger field entering groups augment the reaction rate, as can be seen in Table \ref{tab:enteringGroupRate}.
    \begin{table}[h!]
        \centering
        \renewcommand{\arraystretch}{1.4}
        \setlength{\tabcolsep}{3em}
        \begin{tabular}{lcc}
            \rowcolor{grx}
             & \multicolumn{2}{c}{\textcolor{white}{\textbf{Rate Constants for Anation}}}\\
            \rowcolor{grx}
            \textcolor{white}{\textbf{Entering Ligand}} & \vertcell{\textcolor{white}{\textbf{\ce{[Cr(NH3)5(H2O)]^3+}}}\\\textcolor{white}{$\bm{k(10^{-4}\, M^{-1}s^{-1})}$}} & \vertcell{\textcolor{white}{\textbf{\ce{[Cr(H2O)6]^3+}}}\\\textcolor{white}{$\bm{k(10^{-8}\, M^{-1}s^{-1})}$}}\\
    
            \ce{NCS-}     & 4.2 & 180\\
            \rowcolor{grt}
            \ce{NO3-}     & --- & 73\\
            \ce{Cl-}      & 0.7 & 2.9\\
            \rowcolor{grt}
            \ce{Br-}      & 3.7 & 0.9\\
            \ce{I-}       & --- & 0.08\\
            \rowcolor{grt}
            \ce{CF3COOO-} & 1.4 & ---\\
            \noalign{\global\arrayrulewidth=1pt}\arrayrulecolor{grx}\hline
            \noalign{\global\arrayrulewidth=0.4pt}
        \end{tabular}
        \caption{Effects of the entering group on reaction rate.}
        \label{tab:enteringGroupRate}
    \end{table}
    \item \textbf{Conjugate base mechanism}: A ligand is deprotonated, the ligand \emph{trans} to it dissociates, and then the deprotonated ligand is reprotonated and a new ligand binds. \emph{Also known as} \textbf{S\textsubscript{N}1CB}, \textbf{substitution, nuccleophilic, unimolecular, conjugate base}.
    \begin{itemize}
        \item For example, the mechanism of \ce{[Co(NH3)5X]^2+ + OH- -> [Co(NH3)5(OH)]^2+ + X-} could be
        \begin{gather*}
            \ce{[Co(NH3)5X]^2+ + OH- <=> [Co(NH3)4(NH2)X]+ + H2O}\tag{Equilibrium}\\
            \ce{[Co(NH3)4(NH2)X]+ -> [Co(NH3)4(NH2)]^2+ + X-}\tag{Slow}\\
            \ce{[Co(NH3)4(NH2)]^2+ + H2O -> [Co(NH3)5(OH)]^2+}\tag{Fast}
        \end{gather*}
        \item This mechanism is common with octahedral cobalt (III) metal centers.
    \end{itemize}
    \item \textbf{Kinetic chelate effect}: Substitution for a chelated ligand is slower than a similar monodentate ligand.
    \item In dissociative mechanisms, products can have the same or different stereochemistry than the reactants.
    \begin{itemize}
        \item What happens depends on the stereochemistry of the reactants, the ligands (both to be removed and to be added), and occasionally the concentration of the ligand to be added.
        \item A square pyramidal intermediate leads to a retention of configuration. A trigonal bipyramidal one does not (necessarily; different points of attack along the equator lead to different products in varying ratios).
        \item "As a general rule, \emph{cis} reactants give a relatively higher percentage of substitution products that retain their \emph{cis} configuration; \emph{trans} reactants often afford a more balanced mixture of \emph{cis} and \emph{trans} substitution products" \parencite[456]{bib:MiesslerFischerTarr}.
        \item Compounds with multiple chelating rings can interconvert between stereoisomers by a dissociation $\to$ rearrangement $\to$ reattachment mechanism.
    \end{itemize}
    \item Square-planar nomenclature: \ce{T} is the ligand \emph{trans} to the departing ligand \ce{X}.
    \item Square-planar axes: Let the plane of the molecule be the $xy$-plane, with the $x$-axis collinear to the \ce{T-Pt-X} axis.
    \item Square-planar rate law:
    \begin{equation*}
        \text{Rate} = k_1\ce{[Cplx]}+k_x\ce{[Cplx][Y]}
    \end{equation*}
    \begin{itemize}
        \item The $k_2$ term derives from a standard associative mechanism.
        \item The $k_1$ term derives from a solvent-assisted $A$ mechanism (the solvent substitutes first, slowly; then the incoming ligand replaces the solvent, quickly).
        \begin{itemize}
            \item This may occasionally lead to a 6-coordinate transition state.
        \end{itemize}
    \end{itemize}
    \item Determining the effect on reaction rate of ligands other than \ce{T}:
    \begin{equation*}
        \log k_Y = s\cdot\eta_{\ce{Pt}}+\log k_S
    \end{equation*}
    \begin{itemize}
        \item $k_Y$ is the rate constant for reaction with \ce{Y}.
        \item $k_S$ is the rate constant for reaction with \ce{S}.
        \item $s$ is the \textbf{nucleophilic discrimination factor} (for the complex).
        \begin{itemize}
            \item $s:=1.00$ for \emph{trans}-\ce{[Pt(py)2Cl2]}, and other values are based off of this reference.
        \end{itemize}
        \item $\eta_{\ce{Pt}}$ is the \textbf{nucleophilic reactivity constant} (for the entering ligand).
        \begin{equation*}
            \eta_{\ce{Pt}} = \log\left( \frac{k_Y}{k_{\ce{CH3OH}}} \right)
        \end{equation*}
        \begin{itemize}
            \item Values are determined via the above equation with kinetic data from the reactions used to determine $s$.
        \end{itemize}
    \end{itemize}
    \item \textbf{\emph{trans} influence}: The ground-state, thermodynamic effect where \ce{Pt-T} $\sigma$ donation uses a larger contribution of the $p_x$ and $d_{x^2-y^2}$ orbitals, leaving less for the \ce{Pt-X} bond.
    \item \textbf{Inner-sphere reaction}: A redox reaction of transition-metal complexes where the two molecules are connected by a common ligand through which the electron is transferred.
    \item \textbf{Outer-sphere reaction}: A redox reaction of transition-metal complexes where the exchange occurs between two separate coordination spheres.
    \begin{itemize}
        \item Occur when the ligands of both reactants are tightly held.
        \item Primary change upon electron transfer is metal-ligand bond length, affected both by changes in metal oxidation state and use of $e_g^*$ electrons (adding these increases antibonding character, lengthening bonds; vice versa for removing them).
        \item Promoted by the chelate effect.
    \end{itemize}
    \item Factors that influence electron transfer rates:
    \begin{itemize}
        \item The rate of ligand substitution within the reactants.
        \item The match of the reactant orbital energies.
        \item The solvation of reactants.
        \item The nature of the ligands.
    \end{itemize}
    \item Inner sphere mechanisms depend on \textbf{quantum tunneling}.
    \begin{itemize}
        \item "Ligands with $\pi$ or $p$ orbitals that can be used in bonding provide good pathways for tunneling. Ligands like \ce{NH3}, with neither extra nonbonding pairs nor low-lying antibonding orbitals, do not provide effective tunneling pathways" \parencite[463]{bib:MiesslerFischerTarr}.
    \end{itemize}
    \item \textbf{Quantum tunneling}: A quantum mechanical property whereby electrons can pass through potential barriers that are too high to permit ordinary transfer.
    \item Three steps in an inner-sphere mechanism:
    \begin{enumerate}
        \item A substitution reaction that leaves the oxidant and reductant linked by the bridging ligand.
        \item The electron transfer, frequently accompanied by the transfer of the ligand.
        \item A separation of the products.
    \end{enumerate}
    \item Lists specific examples of molecules/ions that are prone to inner- or outer-sphere mechanisms.
    \item Factors that influence the stability of complexes with different oxidation numbers:
    \begin{itemize}
        \item LFSE.
        \item Metal-ligand bonding.
        \item Redox properties of the ligands.
        \item Hard/soft character of the ligands.
    \end{itemize}
    \item Discusses reactions of coordinated ligands.
\end{itemize}




\end{document}