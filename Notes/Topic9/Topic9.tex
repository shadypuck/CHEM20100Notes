\documentclass[../notes.tex]{subfiles}

\pagestyle{main}
\renewcommand{\chaptermark}[1]{\markboth{\chaptername\ \thechapter\ (#1)}{}}
\renewcommand{\thechapter}{\Roman{chapter}}
\setcounter{chapter}{8}

\begin{document}




\chapter{Reactions and Mechanisms}
\section{Module 46: Ligand Substitution in Octahedral Complexes}
\begin{itemize}
    \item \marginnote{3/8:}Suggested reading: Chapter 12.
    \item Having learned about structure, isomerism, and electronic structure, let's now talk about reactivity.
    \item What makes transition metals unique is the reactions of the metal center (i.e., the first coordination sphere).
    \item \textbf{Substitution reaction}: A reaction where a ligand in the first coordination sphere is exchanged.
    \item Factors that affect the rate of substitution:
    \begin{itemize}
        \item Role of the entering group.
        \item Role of the leaving group.
        \item Nature of the other ligands in the complex.
        \item Effect of the metal center.
    \end{itemize}
    \item Stable/unstable are thermodynamic terms, and labile/inert are kinetic terms.
    \begin{itemize}
        \item \ce{[Co(NH3)6]^3+} is unstable but inert wrt. aquation (large $K_\text{eq}$, but slow to react).
        \item \ce{[Ni(CN)4]^2-} is stable but labile wrt. exchange (small $K_\text{eq}$, but equilibrium is established quickly).
    \end{itemize}
    \item \textbf{Potential energy landscape}: A thermodynamic representation of a reaction by a (potentially multidimensional) potential energy surface along its reaction coordinate.
    \item Thermodynamics and kinetics are two different things, but they can be related.
    \item Potential energy landscapes with small $\Delta G$ and large $\Delta E_A$ are slow to react.
    \begin{itemize}
        \item Increasing $\Delta G$ (perhaps by destabilizing the reactant) can speed up the reaction.
        \item Increasing $\Delta G$ often occurs at the expense of $\Delta E_A$ (the activation energy literally decreases as the free energy change increases).
    \end{itemize}
    \item Consider the reaction \ce{[Co(NH3)5X]^2+ + H2O -> [Co(NH3)5(H2O)]^3+ + X-}.
    \begin{itemize}
        \item The more bulky/less electronegative the X-type ligand, the higher the equilibrium constant and reaction rate.
    \end{itemize}
    \item But how do we adequately compare different reactions when so many parameters are entangled together?
    \item One good way to compare kinetics of transition metals is with water exchange rate constants (light/heavy water being exchanged in metal coordination spheres), since this eliminates thermodynamic consternation.
    \begin{itemize}
        \item There is massive variation among the metals.
        \item The fastest (the alkali metals) run up against the \textbf{diffusion limit}.
        \item Metals with larger atomic radii react more quickly (the alkali and alkaline earth metals follow this pattern nicely, but the transition metals are all over the place).
        \begin{itemize}
            \item Group 1A: As we go down the group, the cations are getting larger and the charge density decreases, so the \ce{M+-OH2} bond is getting weaker and more easily broken.
            \item Group 2A: The charge density is larger (doubly charged) so the strength of the bond is greater so the rate of exchange is slower.
        \end{itemize}
        \item We define such a reaction as having a labile half life if the half life is less than 1 minute; otherwise, it is inert.
    \end{itemize}
    \item \textbf{Diffusion limit}: At most, one molecular collision can result in one successful reaction; thus, reaction rate is bounded above by how many molecular collisions can physically occur in a given period of time at a given temperature.
    \item You cannot prove mechanisms; you can only disprove other plausible scenarios.
    \item Possible mechanisms for \ce{ML5X + Y -> ML5Y + X} where $\ce{X}=\ce{H2O}$ and $\ce{Y}=\text{anion}$ or vice versa:
    \item Associative ($A$): Via a 7-coordinate intermediate.
    \begin{gather*}
        \ce{ML5X + Y <=>[$k_1$][$k_{-1}$] ML5XY}\tag{slow}\\
        \ce{ML5XY ->[$k_2$] ML5Y + X}\tag{fast}
    \end{gather*}
    \begin{itemize}
        \item Corresponds to organic S\textsubscript{N}2.
        \item First step (RDS) is slow since the incoming \ce{Y} causes steric hindrance.
        \item The transition state is either a monocapped octahedron or pentagonal bipyramidal, depending on how the \ce{Y} attacks.
        \item Applying a steady-state approximation for \ce{[ML5XY]}, we have
        \begin{equation*}
            \dv{\ce{[ML5XY]}}{t} = 0 = k_1\ce{[ML5X][Y]}-k_{-1}\ce{[ML5XY]}-k_2\ce{[ML5XY]}
        \end{equation*}
        \item We can solve the above equation for \ce{[ML5XY]}:
        \begin{equation*}
            \ce{[ML5XY]} = \frac{k_1\ce{[ML5X][Y]}}{k_{-1}+k_2}
        \end{equation*}
        \item Substituting the above into the rate constant equation for the fast step will give us the rate law in terms of the reactants (where $k=\frac{k_1k_2}{k_{-1}+k_2}$):
        \begin{align*}
            \text{Rate} &= k_2\ce{[ML5XY]}\\
            &= \frac{k_1k_2\ce{[ML5X][Y]}}{k_{-1}+k_2}\\
            &= k\ce{[ML5X][Y]}
        \end{align*}
        \item Thus, the S\textsubscript{N}2 mechanism is second-order overall, but first-order in both reactants.
    \end{itemize}
    \item Dissociative ($D$): Via a 5-coordiinate intermediate.
    \begin{gather*}
        \ce{ML5X <=>[$k_1$][$k_{-1}$] ML5 + X}\tag{Slow}\\
        \ce{ML5 + Y ->[$k_2$] ML5Y}\tag{Fast}
    \end{gather*}
    \begin{itemize}
        \item Corresponds to organic S\textsubscript{N}1.
        \item First step (RDS) is slow since its spontaneous elimination of a ligand.
        \item The transition state is generally square pyramidal, but if it is sufficiently long-lived, it can reorganize itself into the trigonal bipyramidal state.
        \item The dissociative mechanism predicts that the rate of the overall substitution reaction depends on only the concentration of  the original complex \ce{[ML5X]}, and is independent of the concentration of the incoming ligand \ce{[Y]}.
        \item Thus, the overall rate law is
        \begin{equation*}
            \text{Rate} = k_1\ce{[ML5X]}
        \end{equation*}
        \item We can also derive this with an analogous kinetic analysis to that used for the $A$ mechanism, the only difference being that we simplify $\frac{k_1k_2\ce{[ML5X][Y]}}{k_{-1}\ce{[X]}+k_2\ce{[Y]}}$ to the rate law by noting that $k_2>>k_{-1}$ or $\ce{[Y]}>>\ce{[X]}$.
    \end{itemize}
    \item Interchange ($I$): As \ce{Y} begins to bond, \ce{X} begins to leave, i.e., the bond making to \ce{Y} and bond breaking to \ce{X} occur simultaneously.
    \begin{gather*}
        \ce{ML5X + Y <=>[$k_1$][$k_{-1}$] ML5X*Y}\tag{Slow}\\
        \ce{ML5X*Y ->[$k_2$] ML5Y + X}\tag{Fast}
    \end{gather*}
    \begin{itemize}
        \item Corresponds to organic SN\textsubscript{2}.
        \item This is how the majority of ligand substitutions occur.
        \item It is too simplistic to assume that a first-order rate law implies $D$ and a second-order rate law implies $A$. Indeed, most substitution reactions probably involve a mechanism like this, i.e., one that is intermediate between these two extremes.
        \item In an interchange mechanism, the intermediate involves an association between the original \ce{ML5X} complex and the attacking \ce{Y} ligand. The \ce{Y} ligand remains outside the coordination sphere of \ce{ML5X}, unlike the S\textsubscript{N}2 mechanism, so the intermediate is not seven coordinate. However, it can help weaken the \ce{M-X} bond.
        \item Assuming high $\ce{[Y]}\approx\ce{[Y]_0}$, it can be shown that the rate is given by
        \begin{equation*}
            \text{Rate} = \frac{k_2K_1\ce{[M]_0[Y]_0}}{1+K_1\ce{[Y]_0}}
        \end{equation*}
        where \ce{[M]_0=[ML5X]_0 + [ML5X*Y]} and \ce{[Y]_0} are initial conditions and $K_1=\frac{k_1}{k_{-1}}$ is the equilibrium constant for the RDS reaction.
        \begin{itemize}
            \item At high \ce{[Y]} and $K_1\ce{[Y]_0}>>1$, the rate is first-order in $\ce{[M]_0}\approx\ce{[ML5X]}$.
            \item At lower \ce{[Y]}, the rate is second order.
        \end{itemize}
    \end{itemize}
    \item At some point, the kinetic analysis becomes essentially worthless. Indeed, although we speak generally about associative and disassociative reaction mechanisms, the terms $A$ and $D$ are reserved for situations where 7- and 5-coordinate intermediates have actually been isolated and positively identified. If no intermediates have been isolated or identified, the designations $I_d$ and $I_a$ are more appropriate.
    \begin{itemize}
        \item Two minor variations on the $I$ mechanism are $I_d$ (dissociative interchange) and $I_a$ (associative interchange).
        \item If breaking the \ce{M-X} bond is more important, the mechanism is $I_d$.
        \item If bond formation between \ce{ML5X} and \ce{Y} is significant, the mechanism is $I_a$.
        \item The difference between $I_d$ and $I_a$ is subtle and does not necessarily correspond to whether the observed rate law is first or second order.
        \item If the rates of a series of comparable substitution reactions are most sensitive to the identity of \ce{X}, the leaving ligand, then the mechanism is more probably $I_d$, and vice versa for $I_a$.
        \item For example, the rate constants for the anation of \ce{[Cr(NH3)5(H2O)]^3+} by various ligands vary very little. Thus, it is probably $I_d$. However, the rate constants for the anation of \ce{[Cr(H2O)6]^3+} by various ligands vary by three orders of magnitude. Thus, it is probably $I_a$.
    \end{itemize}
    \item Solvent (e.g., water) effects in substitution reactions:
    \begin{itemize}
        \item Many substitution reactions occurring in solvent water may have first-order kinetics regardless of whether their initial steps are primarily $D$ or $A$.
        \item For example, this occurs if aquation is a precursor RDS.
        \item If $D$, $\text{Rate}=k\ce{[ML5X]}$.
        \item If $A$, $\text{Rate}=k\ce{[ML5X][H2O]}=k'\ce{[ML5X]}$.
        \begin{itemize}
            \item The latter equality is valid since water has constant concentration\footnote{Is this what we meant when we said in \textcite{bib:APChemNotes} that liquids and solids don't have active mass? In other words, the species should be included in the mass-action expression; they just simply get lumped in with the rate constant generally.}.
        \end{itemize}
        \item Both mechanisms lead to apparent first-order kinetics.
    \end{itemize}
    \item Steric factors favoring $D$ or $I_d$:
    \begin{itemize}
        \item For most octahedral complexes, steric factors inhibit formation of a CN7 intermediate, which suggests a dissociative mechanism ($D$ or $I_d$) is more plausible.
        \begin{itemize}
            \item Even cases showing second-order kinetics may not be $A$ for this reason.
        \end{itemize}
        \item For example, aquation of ammine-halides is second order with a first-order dependence on \ce{[OH-]}.
        \item If $k_2>>k_1$, the rate is approximately $\text{Rate}=\ce{k_2[Co(NH3)5X]^2+[OH-]}$.
        \item Calls for an alternate mechanism called S\textsubscript{N}1CB, where ligands with lower energy bond as \underline{c}onjugate \underline{b}ases and those with higher $\Delta_0$ (hence CB).
    \end{itemize}
    \item Kinetically analyzes S\textsubscript{N}1CB.
\end{itemize}




\end{document}