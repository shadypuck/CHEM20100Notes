\documentclass[../notes.tex]{subfiles}

\pagestyle{main}
\renewcommand{\chaptermark}[1]{\markboth{\chaptername\ \thechapter\ (#1)}{}}
\renewcommand{\thechapter}{\Roman{chapter}}
\setcounter{chapter}{3}

\begin{document}




\chapter{Hard-Soft Acid-Base and Donor-Acceptor Concepts of Transition Metals}
\chaptermark{Acids and Bases}
\section{Module 24: Acid-Base Chemistry}
\begin{itemize}
    \item \marginnote{2/8:}Br{\o}nsted-Lowry Acid-Base Theory of Acids and Bases (1923).
    \begin{itemize}
        \item Acid: Any chemical species (molecule or ion) that is able to lose, or "donate," a hydrogen ion (proton).
        \item Base: Any chemical species that is able to gain, or "accept", a proton.
        \begin{itemize}
            \item A base must have a pair of electrons available to share with the proton; this is usually present as an unshared pair, but sometimes is in a $\pi$ orbital.
        \end{itemize}
        \item Acid-base reactions: The transfer of a proton from an acid to a base.
        \begin{equation*}
            \ce{A-H + $:$BH <=> A$:$ + BH2}
        \end{equation*}
        \begin{itemize}
            \item Protons do not exist free in solution but must be attached to an electron pair.
        \end{itemize}
        \item Water is amphoteric.
    \end{itemize}
    \item In the Br{\o}nsted-Lowry paradigm, we cannot separate the acids/bases from the solvent (no protons; only \emph{solvated} protons). In a non-aqueous medium such as DMSO, however, we have much broader scope of acids and bases.
    \item \textbf{Carbon acid}: Any molecule containing a \ce{C-H} bond can lose a proton forming the carbanion.
    \item Carborane (\ce{H(CHB11Cl11)}) is a superacid one million times stronger than sulfuric acid since it's conjugate basis is incredibly stable (super easy to delocalize the charge).
    \item The base dissociation constant or $\Kb$ is a measure of basicity. $\pKb$ is the negative log of $\Kb$ and related to the $\pKa$ by the simple relationship $\pKa+\pKb=14$. The larger the $\pKb$, the more basic the compound.
    \item \textbf{Superacid}: An acid with acidity greater than that of 100\% pure sulfuric acid.
    \begin{itemize}
        \item In water, the strongest acid you can have is \ce{H3O+}.
        \item The strongest superacids are prepared by the combination of two components, a strong Lewis acid and a strong Br{\o}nsted-Lowry acid.
        \item Fluoroantimonic acid \ce{HF-SbF5} is $\num{2e19}$ stronger than 100\% sulfuric acid.
        \item Olah's magic acid (\ce{FSO3H-SbF5}) can dissolve paraffin (candle wax; extremely inert), converting methane into the t-butyl carbocation.
    \end{itemize}
    \item \textbf{Hammett acidity function}: Can replace the $\pH$ in concentrated solutions. \emph{Also known as} $\bm{H_0}$.
    \begin{equation*}
        H_0 = \p{K_{\ce{BH^+}}}+\log\frac{[\ce{B}]}{[\ce{BH^+}]}
    \end{equation*}
    \begin{itemize}
        \item Let \ce{BH+} be the conjugate acid of a very weak base \ce{B}, with a very negative $\p{K_{\ce{BH+}}}$. In this way, it is rather as if the $\pH$ scale has been extended to very negative values.
        \item Hammett originally used a series of anilines with EWGs for the bases.
    \end{itemize}
    \item \textbf{Superbase}: A compound that has a high affinity for protons.
    \begin{itemize}
        \item Again, these do not exist in water.
        \item Often destroyed by water, \ce{CO2}, and \ce{O2}.
        \item A superbase has been defined as an organic compound whose basicity is greater than that of proton sponge, which has conjugate $\pKa$ of 12.1.
        \item These are valuable in organic chemistry, which abounds in very weak acids.
        \item A common superbase is lithium diisopropylamide.
    \end{itemize}
\end{itemize}




\end{document}