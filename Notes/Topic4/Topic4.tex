\documentclass[../notes.tex]{subfiles}

\pagestyle{main}
\renewcommand{\chaptermark}[1]{\markboth{\chaptername\ \thechapter\ (#1)}{}}
\renewcommand{\thechapter}{\Roman{chapter}}
\setcounter{chapter}{3}

\begin{document}




\chapter{Hard-Soft Acid-Base and Donor-Acceptor Concepts of Transition Metals}
\chaptermark{Acids and Bases}
\section{Module 24: Acid-Base Chemistry}
\begin{itemize}
    \item \marginnote{2/8:}Br{\o}nsted-Lowry Acid-Base Theory of Acids and Bases (1923).
    \begin{itemize}
        \item Acid: Any chemical species (molecule or ion) that is able to lose, or "donate," a hydrogen ion (proton).
        \item Base: Any chemical species that is able to gain, or "accept", a proton.
        \begin{itemize}
            \item A base must have a pair of electrons available to share with the proton; this is usually present as an unshared pair, but sometimes is in a $\pi$ orbital.
        \end{itemize}
        \item Acid-base reactions: The transfer of a proton from an acid to a base.
        \begin{equation*}
            \ce{A-H + $:$BH <=> A$:$ + BH2}
        \end{equation*}
        \begin{itemize}
            \item Protons do not exist free in solution but must be attached to an electron pair.
        \end{itemize}
        \item Water is amphoteric.
    \end{itemize}
    \item In the Br{\o}nsted-Lowry paradigm, we cannot separate the acids/bases from the solvent (no protons; only \emph{solvated} protons). In a non-aqueous medium such as DMSO, however, we have much broader scope of acids and bases.
    \item \textbf{Carbon acid}: Any molecule containing a \ce{C-H} bond can lose a proton forming the carbanion.
    \item Carborane (\ce{H(CHB11Cl11)}) is a superacid one million times stronger than sulfuric acid since it's conjugate basis is incredibly stable (super easy to delocalize the charge).
    \item The base dissociation constant or $\Kb$ is a measure of basicity. $\pKb$ is the negative log of $\Kb$ and related to the $\pKa$ by the simple relationship $\pKa+\pKb=14$. The larger the $\pKb$, the more basic the compound.
    \item \textbf{Superacid}: An acid with acidity greater than that of 100\% pure sulfuric acid.
    \begin{itemize}
        \item In water, the strongest acid you can have is \ce{H3O+}.
        \item The strongest superacids are prepared by the combination of two components, a strong Lewis acid and a strong Br{\o}nsted-Lowry acid.
        \item Fluoroantimonic acid \ce{HF-SbF5} is $\num{2e19}$ stronger than 100\% sulfuric acid.
        \item Olah's magic acid (\ce{FSO3H-SbF5}) can dissolve paraffin (candle wax; extremely inert), converting methane into the t-butyl carbocation.
    \end{itemize}
    \item \textbf{Hammett acidity function}: Can replace the $\pH$ in concentrated solutions. \emph{Also known as} $\bm{H_0}$.
    \begin{equation*}
        H_0 = \p{K_{\ce{BH^+}}}+\log\frac{[\ce{B}]}{[\ce{BH^+}]}
    \end{equation*}
    \begin{itemize}
        \item Let \ce{BH+} be the conjugate acid of a very weak base \ce{B}, with a very negative $\p{K_{\ce{BH+}}}$. In this way, it is rather as if the $\pH$ scale has been extended to very negative values.
        \item Hammett originally used a series of anilines with EWGs for the bases.
    \end{itemize}
    \item \textbf{Superbase}: A compound that has a high affinity for protons.
    \begin{itemize}
        \item Again, these do not exist in water.
        \item Often destroyed by water, \ce{CO2}, and \ce{O2}.
        \item A superbase has been defined as an organic compound whose basicity is greater than that of proton sponge, which has conjugate $\pKa$ of 12.1.
        \item These are valuable in organic chemistry, which abounds in very weak acids.
        \item A common superbase is lithium diisopropylamide (LDA).
    \end{itemize}
\end{itemize}



\section{Module 25: Acid-Base Chemistry: Lewis Acids and Bases}
\begin{itemize}
    \item \marginnote{2/10:}The second midterm will be Saturday, February 20.
    \item Talapin will hold office hours this Friday from 3:00-4:00 PM.
    \item \textbf{Lewis acid}: Any species with a vacant orbital.
    \begin{itemize}
        \item An atomic or molecular species that has an empty atomic or molecular orbital of low energy (LUMO) that can accommodate a pair of electrons.
    \end{itemize}
    \item \textbf{Lewis base}: A compound with an available pair of electrons, either unshared or in a $\pi$ orbital.
    \begin{itemize}
        \item An atomic or molecular species that has a lone pair of electrons in the HOMO.
    \end{itemize}
    \item Examples of Lewis acids:
    \begin{itemize}
        \item The proton, onium ions (e.g., \ce{NH4+}, \ce{H3O+}), metal cations (e.g., \ce{Li+}, \ce{Mg^2+}), trigonal planar species (e.g., \ce{BF3}, \ce{CH3+}), and electron-poor $\pi$-systems (e.g., enones, tetracyanoethylene [TCNE]).
    \end{itemize}
    \item \textbf{Lewis base adduct}: The product of a Lewis acid reaction.
    \begin{itemize}
        \item Simple Lewis acids:
        \begin{itemize}
            \item Examples include \ce{BF4-} in \ce{BF3 + F- -> BF4-} and \ce{BF3OMe2} in \ce{BF3 + OMe2 -> BF3OMe2}, which are Lewis base adducts of \ce{BF3}.
            \item In many cases, Lewis base adducts violate the octet rule.
            \item In some cases, Lewis acids can bind to two Lewis bases (e.g., \ce{SiF4 + 2F- -> SiF6^2-}).
        \end{itemize}
        \item Complex Lewis acids:
        \begin{itemize}
            \item Most compounds considered to be Lewis acids require an activation step prior to the formation of the adduct.
            \item For example, the reaction \ce{B2H6 + 2H- -> 2BH4-} goes through the intermediate \ce{B2H7-}.
        \end{itemize}
    \end{itemize}
    \item The proton (\ce{H+}) is one of the strongest but is also one of the most complicated Lewis acids. It is convention to ignore the fact that a proton is heavily solvated (bound to solvent). With this simplification in mind, acid-base reactions can be viewed as the formation of adducts.
    \begin{itemize}
        \item For example, \ce{H+ + NH3 -> NH4+} and \ce{H+ + OH- -> H2O}.
        \item In the first reaction, essentially what happens is the frontier orbitals of \ce{H+} and \ce{NH3} are very similar in energy, and thus strongly combine when they form bonding and antibonding orbitals. Two electrons previously in the HOMO of \ce{NH3} drop the bonding energy into the new lower MO of \ce{NH4+}. More rigorously, we have to account for the change in symmetry group from $C_{3v}$ to $T_d$, but the above description basically encapsulates what happens.
    \end{itemize}
    \item Examples of Lewis bases:
    \begin{itemize}
        \item \ce{NH_{$3-x$}R_{$x$}} where $\ce{R}=\text{alkyl or aryl}$.
        \item \ce{PR_{$3-x$}A_{$x$}} where $\ce{R}=\text{alkyl}$ and $\ce{A}=\text{aryl}$.
        \item Compounds of \ce{O}, \ce{S}, \ce{Se}, and \ce{Te} in oxidation state 2, including water, ethers, and ketones.
        \item Simple anions (e.g., \ce{H-}, \ce{F-}), other lone pair-containing species (e.g., \ce{H2O}, \ce{NH3}, \ce{OH-}, \ce{CH3-}), complex anions (e.g., sulfate), and electron-rich $\pi$-systems (e.g., ethyne, ethene, benzene).
    \end{itemize}
    \item To quantify the strength of Lewis acids and bases, compare the standard enthalpies of complexation in $\si[per-mode=symbol]{\kilo\joule\per\mole}$.
    \begin{itemize}
        \item For example, heats of binding of Lewis bases to \ce{BF3}.
        \item We do this because we cannot use $\pKb$ in the same way as we can in the Br{\o}nsted-Lowry description.
    \end{itemize}
    \item \textbf{Electrides}: Compounds where the anions are electrons.
    \begin{itemize}
        \item Put an alkali metal in liquid ammonia; the deep blue color comes from solvated electrons floating freely in solution surrounded by ammonia molecules.
        \item If you add a very strong complexing agent for alkali metal ions to the solution, then the electrons will not be able to react back with the alkali metal and we will have crystallized an electride.
        \item Electrides are dielectrics.
    \end{itemize}
    \item The power of the Lewis approach is that it allows us to classify many reactions as acid-base reactions.
    \begin{itemize}
        \item For example, Wilkinson's catalyst (\ce{Rh(PPh3)Cl} where \ce{Ph} is a phenyl group) is probably the first hydrogenation catalyst to be explained.
        \item The Lewis approach allows us to view every step in the mechanism\footnote{We will study this mechanism in depth in CHEM 20200.} as an acid-base reaction.
    \end{itemize}
\end{itemize}



\section{Module 26: Hard and Soft Acids and Bases (HSAB) Principle}
\begin{itemize}
    \item HSAB principle can't be derived, but it's been proven to be a very powerful tool for chemists.
    \item The affinity of hard acids and hard bases for each other is mainly ionic in nature, whereas the affinity of soft acids and bases for each other is mainly covalent in nature.
    \item Hard acids and hard bases tend to have:
    \begin{itemize}
        \item Small atomic/ionic radius.
        \item High oxidation state.
        \item Low polarizability.
        \item High electronegativity.
        \item Energy low-lying HOMO (bases) or energy high-lying LUMO (acids).
        \item Examples: \ce{H+}, alkali ions, \ce{OH-}, \ce{F-}.
        \item Small stabilization energy when bonding, so attraction is mostly electrostatic.
    \end{itemize}
    \item Soft acids and soft bases tend to have:
    \begin{itemize}
        \item Large atomic/ionic radius.
        \item Low or zero oxidation state.
        \item High polarizability.
        \item Low electronegativity.
        \item Energy high-lying HOMO (bases) or energy low-lying LUMO (acids).
        \item Examples: \ce{CCH3Hg+}, \ce{Pt^2+}, \ce{H-}, \ce{R3P}, \ce{I-}.
        \item Large stabilization energy when bonding, so attraction is mostly covalent.
    \end{itemize}
    \item Hard acids don't readily react with soft bases and vice versa.
    \item You should be able to intuitively classify acids/bases as hard or soft.
    \begin{figure}[h!]
        \centering
        \begin{tikzpicture}[node distance=-0.6pt]
            \felement{1}{H}{grt}{}
            \belement{3}{Li}{grt}{H}
            \belement{11}{Na}{grt}{Li}
            \belement{19}{K}{grt}{Na}
            \belement{37}{Rb}{grt}{K}
            \belement{55}{Cs}{grt}{Rb}
            \belement{87}{Fr}{grt}{Cs}
    
            \felement{4}{Be}{grt}{[right=of Li]}
            \belement{12}{Mg}{grt}{Be}
            \belement{20}{Ca}{grt}{Mg}
            \belement{38}{Sr}{grt}{Ca}
            \belement{56}{Ba}{grt}{Sr}
            \belement{88}{Ra}{grt}{Ba}
    
            \felement{21}{Sc}{grt}{[right=of Ca]}
            \belement{39}{Y}{grt}{Sc}
    
            \felement{22}{Ti}{grt}{[right=of Sc]}
            \belement{40}{Zr}{grt}{Ti}
            \belement{72}{Hf}{grt}{Zr}
    
            \felement{23}{V}{grt}{[right=of Ti]}
            \belement{41}{Nb}{grt}{V}
            \belement{73}{Ta}{gry}{Nb}
    
            \felement{24}{Cr}{grt}{[right=of V]}
            \belement{42}{Mo}{gry}{Cr}
            \belement{74}{W}{gry}{Mo}
    
            \felement{25}{Mn}{gry}{[right=of Cr]}
            \belement{43}{Te}{gry}{Mn}
            \belement{75}{Re}{gry}{Te}
    
            \felement{26}{Fe}{gry}{[right=of Mn]}
            \belement{44}{Ru}{gax}{Fe}
            \belement{76}{Os}{gax}{Ru}
    
            \felement{27}{Co}{gry}{[right=of Fe]}
            \belement{45}{Rh}{gax}{Co}
            \belement{77}{Ir}{gax}{Rh}
    
            \felement{28}{Ni}{gry}{[right=of Co]}
            \belement{46}{Pd}{gax}{Ni}
            \belement{78}{Pt}{gax}{Pd}
    
            \felement{29}{Cu}{gry}{[right=of Ni]}
            \belement{47}{Ag}{gax}{Cu}
            \belement{79}{Au}{gax}{Ag}
    
            \felement{30}{Zn}{gry}{[right=of Cu]}
            \belement{48}{Cd}{gax}{Zn}
            \belement{80}{Hg}{gax}{Cd}
    
            \felement{13}{Al}{grt}{[above right=of Zn]}
            \belement{31}{Ga}{grt}{Al}
            \belement{49}{In}{gry}{Ga}
            \belement{81}{Ti}{gax}{In}
    
            \felement{14}{Si}{grt}{[right=of Al]}
            \belement{32}{Ge}{grt}{Si}
            \belement{50}{Sn}{grt}{Ge}
            \belement{82}{Pb}{gry}{Sn}
    
            \felement{33}{As}{grt}{[right=of Ge]}
            \belement{51}{Sb}{grt}{As}
            \belement{83}{Bi}{gry}{Sb}
    
            \felement{57}{La}{grt}{at (0.5,-7.7)}
            \relement{58}{Ce}{grt}{La}
            \relement{59}{Pr}{grt}{Ce}
            \relement{60}{Nd}{grt}{Pr}
            \relement{61}{Pm}{grt}{Nd}
            \relement{62}{Sm}{grt}{Pm}
            \relement{63}{Eu}{grt}{Sm}
            \relement{64}{Gd}{grt}{Eu}
            \relement{65}{Tb}{grt}{Gd}
            \relement{66}{Dy}{grt}{Tb}
            \relement{67}{Ho}{grt}{Dy}
            \relement{68}{Er}{grt}{Ho}
            \relement{69}{Tm}{grt}{Er}
            \relement{70}{Yb}{grt}{Tm}
            \relement{71}{Lu}{grt}{Yb}
    
            \felement{89}{Ac}{grt}{at (0.5,-9)}
            \relement{90}{Th}{grt}{Ac}
            \relement{91}{Pa}{grt}{Th}
            \relement{92}{U}{grt}{Pa}
            \relement{93}{Np}{grt}{U}
            \relement{94}{Pu}{grt}{Np}
            \relement{95}{Am}{grt}{Pu}
            \relement{96}{Cm}{grt}{Am}
            \relement{97}{Bk}{grt}{Cm}
    
            \small
            \node (hard) [draw,semithick,fill=grt,minimum width=6mm,minimum height=6mm,label={below:hard}] at (6,0) {};
            \node (soft) [draw,semithick,fill=gax,minimum width=6mm,minimum height=6mm,label={below:soft}] [right=4mm of hard] {};
            \node (intermediate) [draw,semithick,fill=gry,minimum width=6mm,minimum height=6mm,label={[xshift=6mm]below:intermediate}] [right=4mm of soft] {};
        \end{tikzpicture}
        \caption{Hard vs. soft metal ions.}
        \label{fig:hardSoftPeriodicTbl}
    \end{figure}
    \begin{itemize}
        \item The affinity that metal ions have for ligands is controlled by size, charge, and electronegativity.
        \item This can be refined further by noting that for some metal ions, their chemistry is dominated by size and charge, while for others it is dominated by their electronegativity. These two categories of metal ions have been termed by Pearson as \textbf{hard} (metal ions) and \textbf{soft} (metal ions).
        \item Soft Lewis acids are transition metals with unusually high electronegativities (these go against typical periodic trends).
        \begin{itemize}
            \item Note that QMech can't explain this island of high electronegativities in the transition metals; we need relativistic corrections and Dirac equations.
            \item The higher electronegativity stabilizes the LUMO of the acid.
            \item These ions are a treasure trove for catalytic chemistry.
        \end{itemize}
    \end{itemize}
    \item \textbf{Chemical hardness}: One half the difference between the ionization potential $I$ and the electron affinity $A$.
    \begin{equation*}
        \eta = \frac{I-A}{2}
    \end{equation*}
    \begin{itemize}
        \item Note that the above equation cannot be applied to anions because electron affinity cannot be measured for them; the assumption is made that $\eta$ for an anion \ce{X-} is same as that for the radical \charge{0=\.}{X}\hspace{5pt}.
        \item Related to the \textbf{Mulliken electronegativity}:
        \begin{equation*}
            \chi = \frac{I+A}{2}
        \end{equation*}
    \end{itemize}
    \item \textbf{HSAB principle}:
    \begin{enumerate}[label={(\roman*)}]
        \item Hard acids prefer to bond to hard bases, and soft acids prefer to bond to soft bases.
        \begin{itemize}
            \item Note that this rule has nothing to do with acid or base strength but merely says that the product \ce{A-B} will have extra stability if \ce{A} and \ce{B} are both hard or both soft.
        \end{itemize}
        \item A soft Lewis acid and a soft Lewis base tend to form a covalent bond, while a hard acid and a hard base tend to form ionic bonds.
    \end{enumerate}
    \item Solubility: Hard solvents (e.g., \ce{HF}, \ce{H2O}, and the protic solvents) tend to solvate strong solute bases (e.g., \ce{F-} and the oxygen anions). Dipolar aprotic solvents (e.g., \ce{Me2SO} and \ce{CH3COCH3}) are soft solvents with a preference for solvating large anions and soft bases.
    \begin{itemize}
        \item For example, \ce{LiI + AgF -> LiF + AgI}. In this reaction, two mixed reactants recombine exothermically to form a hard-hard product and a soft-soft precipitate. Because of the Lewis definition, we can treat this as an acid-base reaction.
        \begin{table}[h!]
            \centering
            \small
            \renewcommand{\arraystretch}{1.4}
            \begin{tabular}{cl|cccc}
                \hline
                \multicolumn{2}{c|}{\multirow{2}{*}{\textbf{Classification}}} & Hard & Intermediate & \multicolumn{2}{c}{Soft}\\
                 & & \ce{F-} & \ce{Cl-} & \ce{Br-} & \ce{I-}\\
                \hline
                Soft & \ce{Ag+} & 0.4 & 3.3 & 4.7 & 6.6\\
                Intermediate & \ce{Pb^2+} & 1.3 & 0.9 & 1.1 & 1.3\\
                Hard & \ce{Fe^3+} & 6.0 & 1.4 & 0.5 & -\\
                \hline
            \end{tabular}
            \caption{Hard and soft formation constants.}
            \label{tab:hardSoftFormationCnst}
        \end{table}
        \item As another example, we can see in Table \ref{tab:hardSoftFormationCnst} that soft-soft and hard-hard ions have higher formation constants ($\log K_1$) than any other combination, and hard-soft ions have lower formation constants than any other combination.
    \end{itemize}
    \item Coordination chemistry: Numerous experiments have been done to determine the relative ordering of ligands and transition metal ions in terms of their hardness and softness.
    \item \ce{Au} (I) is the softest metal ion.
    \begin{itemize}
        \item Soft to the extent that compounds such as \ce{AuF} and \ce{Au2O} are unknown.
        \item Nevertheless, it forms stable compounds with soft ligands such as cyanide.
    \end{itemize}
    \item \ce{Al} (III) is a very hard metal ion.
    \begin{itemize}
        \item It has very high formation constants with both \ce{F-} and \ce{OH-}; additionally, it has virtually no affinity in solution for heavier halides such as \ce{Cl-}.
        \item Its solution chemistry is dominated by its affinity for \ce{F-} and for ligands with negative \ce{O}-donors.
    \end{itemize}
    \item \textbf{Symbiosis effect}: Taking a borderline element and reacting it with a soft Lewis base to soften it, or a hard Lewis base to harden it.
    \begin{itemize}
        \item For example, \ce{B^3+} can be reacted with \ce{3H-}, we make the soft compound \ce{BH3}. If we react it with \ce{3F-}, we make the hard compound \ce{BF3}.
        \item Now \ce{BH3} and \ce{F-} do not react, but we can further soften it in the reaction \ce{BH3 + H- -> BH4-}.
        \item Similarly, \ce{BF3} and and \ce{H-} don't react, but we can further harden \ce{BF3} by reacting it with \ce{F-} to form \ce{BF4-}.
    \end{itemize}
    \item There is a preference against combining hard and soft ligands in the first coordination sphere. Examples:
    \begin{itemize}
        \item \ce{CH3F + CF3I <=>> CH3I + CF4} since \ce{CH3+} and \ce{I-} are soft while \ce{CF3+} and \ce{F-} are hard.
        \item \ce{Co(NH3)5F} is stable since it has all hard ligands, but \ce{Co(NH3)5I} will react with \ce{H2O} to produce \ce{Co(NH3)5OH} since the former combines hard and soft and the latter does not.
        \item Thiocyanate is \textbf{ambidentate}. Thus, we have \ce{Fe-N} bonds in \ce{Fe(NCS)6^3+} (with the hard \ce{Fe} (III) ion) and \ce{Au-S} bonds in \ce{Au(SCN)2-} (with the soft \ce{Au} (I) ion).
        \begin{itemize}
            \item Note that intermediate metal ions tend to bond to thiocyanate through nitrogen; indeed, intermediate \ce{Cu} (II) forms \ce{Cu(NCS)4^2-}.
        \end{itemize}
    \end{itemize}
    \item \textbf{Ambidentate} (ligand): A polyatomic ligand that can bind through more than one of its constituent atoms.
\end{itemize}



\section{Module 27: Steric Effects in Inorganic Chemistry}
\begin{itemize}
    \item \marginnote{2/12:}Chemistry is a game of two players: Electronic and steric effects.
    \item Gallanes, gallenes, cyclogallenes, and gallynes:
    \begin{itemize}
        \item Various compounds with gallium.
        \item React gallium (III) chloride with a Grignard reagent containing mesityl (2,4,6-trimethylphenyl; Mes; \ce{Me3C6H2}): \ce{3(Me3C6H2)MgBr + GaCl3 ->[][-3MgBrCl] Ga(Me3C6H2)3}.
        \item A molecule with even more steric hindrance: a three gallium ring with a 2,6-dimesitylphenyl groups attached to each gallium.
        \item The triphenylcyclopropenium cation is aromatic but the cyclogallene dianion is metalloaromatic.
        \item In another compound, a \ce{Ga#Ga} triple bond holds two massive groups together.
        \begin{itemize}
            \item The debate over this is a good example of chemistry not being black and white.
        \end{itemize}
        \item Acetylene and gallyne both show band structure.
        \begin{itemize}
            \item Additionally, the former is linear while the latter is trans-bent with a donor-acceptor bond lacking a sigma bond.
            \item To learn more about such bonding, read \textcite{bib:PowerTransBent}.
        \end{itemize}
    \end{itemize}
    \item Steric effects are much more common in Lewis acid-base reactions in which larger acids are used.
    \begin{itemize}
        \item With a small acid, increasing size in the base (i.e., replacing hydrogens with methyl groups) correlates with increasing strength.
        \item With a large acid, decreasing size in the base correlates with increasing strength.
    \end{itemize}
    \item Frustrated Lewis pairs:
    \begin{itemize}
        \item Excessive steric bulk can prevent acids and bases from reacting to form an adduct.
        \item However, you can also have alternate reactions that yield compounds with Lewis acidic \emph{and} Lewis basic regions.
    \end{itemize}
    \item Frustrated Lewis pairs can activate (cleave) \ce{H2}.
    \begin{itemize}
        \item They can also activate \ce{CO2}; this has applications in the realm of \ce{CO2} sequestration.
    \end{itemize}
\end{itemize}



\section{Chapter 6: Acid-Base and Donor-Acceptor Chemistry}
\emph{From \textcite{bib:MiesslerFischerTarr}.}
\begin{itemize}
    \item IUPAC calls hydronium "oxonium" and uses "hydrogen ion" instead of "proton."
    \item In the Br{\o}nsted-Lowry paradigm, "the equilibrium always favors the formation of \emph{weaker} acids and bases" \parencite[171]{bib:MiesslerFischerTarr}.
    \item Common amphoteric solvents: sulfuric acid, hydrofluoric acid, water, acetic acid, methanol, ammonia, and acetonitrile.
    \item \ce{H3O+} and \ce{OH-} are the strongest acid and base, respectively, that\footnote{Errata: \textcite[173]{bib:MiesslerFischerTarr} has a typo: "than" instead of "that."} can exist in \ce{H2O}.
    \item \textbf{Leveling effect}: Acids stronger than \ce{H3O+} cannot be differentiated by their aqueous ionization.
    \item Because of leveling, the strength of the strong acids cannot be differentiated in aqueous solution.
    \begin{itemize}
        \item Thus, we use more strongly acidic solvents, such as \textbf{glacial} acetic acid.
        \item The use of this solvent allows us to determine that $\ce{HClO4}>\ce{HCl}>\ce{H2SO4}>\ce{HNO3}$.
        \item Note that basic solvents similarly permit the differentiation of strong bases.
    \end{itemize}
    \item \textbf{Glacial} (substance): A 100\% pure, concentrated sample of a substance.
    \item "Nonamphoteric solvents\dots do not limit solute acidity or basicity because the solute does not react with the solvent. In these solvents, the inherent solute acid or base strength determines the reactivity, without a leveling effect" \parencite[173]{bib:MiesslerFischerTarr}.
    \item \textbf{Superelectrophilic activation}: The result of generating small organic ions bearing a large amount of positive charge.
    \item Introduces the Hammett acidity function\footnote{Errata: The $-$ sign should be a $+$ sign.}.
    \item Protonating \ce{CH4} is of particular interest since the natural abundance of methane (from natural gas) makes it attractive as a starting point for synthesis of more complex molecules.
    \begin{itemize}
        \item \ce{CH5+} has been isolated; \ce{CH6^2+} and \ce{CH7^3+} have been proposed.
    \end{itemize}
    \item Dissolving \ce{SO3} in \ce{H2SO4} results in \textbf{fuming sulfuric acid}, which contains higher polysulfuric acids such as \ce{H2S2O7}, all of which are stronger than \ce{H2SO4}.
    \item Water is a strong base in superacid media: Adding \ce{H2O} to superacids produces hydronium salts.
    \begin{itemize}
        \item As a consequence, we can form \ce{[H3O][Ln][AsF6]3} where \ce{Ln} is a lanthanide element and \ce{AsF6} is part of a superacid.
    \end{itemize}
    \item Thermodynamic measurements:
    \begin{itemize}
        \item Hess's law is often used to express the enthalpy of weak acid reactions in terms of reactions that do go to completion; this is not perfect, but it's an ok starting point.
        \item One can also measure $\Ka$ via titration curves at different temperatures and use the van't Hoff equation
        \begin{equation*}
            \ln\Ka = \frac{-\Delta H}{R}\frac{1}{T}+\frac{\Delta S}{R}
        \end{equation*}
        \begin{itemize}
            \item According to the above equation the plot of $\ln\Ka$ vs. $\frac{1}{T}$ will be linear with a slope that allows us to determine $\Delta H$ and a $y$-intercept that allows us to determine $\Delta S$.
        \end{itemize}
        \item Br{\o}nsted basicity scale: Enthalpy changes associated with protonation by the superacid fluorosulfonic acid (\ce{HSO3F}).
    \end{itemize}
    \item The best measure of acid/base strength is gas-phase acidity/basicity, since there are no solvent effects.
    \begin{itemize}
        \item Wrt. \ce{HA(g) -> A-(g) + H+(g)}, $\Delta G=\text{Gas-Phase Acidity (GA)}$ and $\Delta H=\text{Proton Affinity (PA)}$.
        \item There exist analogous definitions for bases.
        \item Modern measurement techniques can measure these values very accurately for a select few molecules; from these molecules and Hess's law, we can build pretty good approximations of other reactions.
    \end{itemize}
    \item \textbf{Superbase}: A base with a gas-phase proton affinity greater than $\SI[per-mode=symbol]{1000}{\kilo\joule\per\mole}$.
    \begin{itemize}
        \item Examples such as Grignard and organolithium reagents are ubiquitous in organic synthesis.
    \end{itemize}
    \item Considers inductive effects.
    \item "Solvation [of amines] is dependent on the number of hydrogen atoms available to form \ce{O\cdots H-N} hydrogen bonds with water" \parencite[181]{bib:MiesslerFischerTarr}.
    \item Steric effects are less obvious --- for example, 2,6-dimethylpyridine is more basic than 2-methylpyridine, but the latter is more basic than 2-$t$-butylpyridine (see Figure \ref{fig:4Lewisbases}).
    \item Binary hydrogen compounds:
    \begin{itemize}
        \item Acidity increases left-to-right across a period and down a group (wrt. the non-hydrogen atom).
        \item The three heaviest hydrohalic acids are all equally acidic in water due to leveling.
    \end{itemize}
    \item Considers relative strengths of various chlorine-based oxyacids (e.g., \ce{HClO4} vs. \ce{HClO3}).
    \begin{itemize}
        \item As the number of oxygens increases, the electronegativity of the terminal oxygen increases. Less and less electron density binds the hydrogen, making the bond increasingly weak and susceptible to heterolytic cleavage.
    \end{itemize}
    \item In polyprotic acids, the $\pKa$ increases about 5 units with each successive proton removal.
    \item Aqueous cations get surrounded by \ce{H2O}.
    \begin{itemize}
        \item Then these complexes react in acid-base reactions.
        \item When the concentration is sufficiently high, two complexes can combine via hydroxide or oxide bridges between the metal atoms.
        \item Metal ions with charges of $4+$ or higher are the really acidic ones. They're acidic to the extent that they do not exist by themselves in solution, but are present in oxygenated forms such as permanganate (\ce{MnO4-}).
    \end{itemize}
    \item \textbf{Coordinate covalent} (bond): A bond that links a Lewis acid and base into their adduct. \emph{Also known as} \textbf{dative} (bond).
    \item \textbf{Coordination compound}: A Lewis acid-base adduct involving one or more metal ions.
    \item "In most Lewis acid-base reactions, the HOMO-LUMO combination forms new HOMO and LUMO orbitals of the product" \parencite[186]{bib:MiesslerFischerTarr}.
    \begin{itemize}
        \item When the symmetries match and the energies are close, a stable adduct forms.
    \end{itemize}
    \item Since every molecule has a HOMO and LUMO, technically any molecule can act as acid, base, oxidizing agent, or reducing agent when combined with the right reactants.
    \begin{itemize}
        \item For example, \ce{H2O} can oxidize \ce{Ca} metal, forming \ce{Ca^2+(aq) + 2OH-(aq) + H2(g)}. \ce{H2O-} is not formed because the addition of electrons to antibonding orbitals in \ce{H2O} weakens one \ce{O-H} bond.
    \end{itemize}
    \item \marginnote{2/14:}Section on spectroscopic support for frontier orbital interactions.
    \begin{itemize}
        \item Very confusing; electron transitions' colors in \ce{I2} when it acts as a Lewis acid.
    \end{itemize}
    \item \textbf{Lewis basicity}: The thermodynamic tendency of a substance to act as a Lewis base.
    \begin{itemize}
        \item It would be ideal to measure in the gas phase, but this is hard to do.
        \item Determining a good reference acid is difficult.
        \item When in solution, we want to make sure that we choose a solvent that will not react significantly as a Lewis acid with the solutes.
        \begin{itemize}
            \item In other words, we need a solvent that primarily solvates the solutes through dispersion forces (i.e., nonpolar solvents).
        \end{itemize}
        \item Once we have a solvent, we quantify the Lewis basicity of a substance by finding the $\Kb$ and $\log\Kb$ values for its complexation with a reference acid (such as \ce{I2}). The absolute basicity will vary based on the solvent, but the Lewis basicity ranking will be the same overall.
    \end{itemize}
    \item The standard scale for Lewis basicity is its \ce{BF3} affinity, or $-\Delta H^\circ$ in the following reaction, corrected for the enthalpy of \ce{BF3} dissolving in the solvent.
    \begin{equation*}
        \ce{BF3 + \text{Lewis Base} ->[CH2Cl2] \text{Lewis Base}-BF3}
    \end{equation*}
    \item \textbf{Halogen} (bond): A coordinate covalent bond formed by a halogen \ce{X2} (e.g., \ce{I2}) or interhalogen \ce{XY} (e.g., \ce{ICl}) to a Lewis base.
    \begin{itemize}
        \item Exhibit approximately $\ang{180}$ angles about the halogen donor atom, supporting use of $\sigma^*$ LUMO.
    \end{itemize}
    \item \ce{I2} affinity values are commonly determined in heptane, and while there is interest in creating an \ce{I2} affinity scale analogous to the \ce{BF3} one, as of yet, various experimental designs have not been reconciled.
    \item Since the formation of a halogen bond involves donation into the halogen $\sigma^*$ LUMO, the \ce{X-Y} bond weakens and lengthens. This also decreases the stretching frequency. Measuring changes in stretching frequency induced in \ce{I2}, \ce{ICN}, and \ce{ICl} via Raman spectroscopy has correlated reasonably well to basicity, but not perfectly since these stretching bands are affected by the presence of the Lewis base, too.
    \item Inductive effects:
    \begin{itemize}
        \item Adding more electronegative substituents decreases Lewis basicity (in $C_{3v}$ Lewis bases).
        \item Adding more alkyl substituents increases Lewis basicity (in $C_{3v}$ Lewis bases).
        \item Adding more electronegative substituents increases Lewis basicity (in $D_{3h}$ Lewis bases) since the increase in bond lengths draws $\pi$-bonding electrons farther from the central boron.
    \end{itemize}
    \item Steric effects on Lewis acidity and basicity: \textbf{front strain}, \textbf{back strain}, and \textbf{internal strain}.
    \item \textbf{Front strain}: Bulky groups interfering directly with the approach of an acid and a base to each other. \emph{Also known as} \textbf{F strain}.
    \item \textbf{Back strain}: Bulky groups interfering with each other when VSEPR effects force them to bend away from the other molecule forming the adduct. \emph{Also known as} \textbf{B strain}.
    \item \textbf{Internal strain}: Effects from electronic differences within similar molecules. \emph{Also known as} \textbf{I strain}.
    \item An example of differences caused by F strain:
    \begin{figure}[h!]
        \centering
        \begin{subfigure}[b]{0.24\linewidth}
            \centering
            \chemfig{[:-120]N**6(-(-H_3C)----(-CH_3)-)}
            \caption{2,6-dimethylpyridine.}
            \label{fig:4Lewisbasesa}
        \end{subfigure}
        \begin{subfigure}[b]{0.24\linewidth}
            \centering
            \chemfig{[:-120]N**6(-(-H_3C)-----)}
            \caption{2-methylpyridine.}
            \label{fig:4Lewisbasesb}
        \end{subfigure}
        \begin{subfigure}[b]{0.24\linewidth}
            \centering
            \chemfig{[:-120]N**6(-(-C(-[1]CH_3)(-[3]H_3C)(-[5]H_3C))-----)}
            \caption{2-$t$-butylpyridine.}
            \label{fig:4Lewisbasesc}
        \end{subfigure}
        \begin{subfigure}[b]{0.15\linewidth}
            \centering
            \chemfig{[:-120]N**6(------)}
            \caption{Pyridine.}
            \label{fig:4Lewisbasesd}
        \end{subfigure}
        \caption{Four Lewis bases.}
        \label{fig:4Lewisbases}
    \end{figure}
    \begin{itemize}
        \item When reacting with \ce{H+}, $\text{2,6-dimethylpyridine}>\text{2-methylpyridine}>\text{2-$t$-butylpyridine}>\text{pyridine}$.
        \item When reacting with \ce{BF3} or \ce{BMe3}, $\text{pyridine}>\text{2-methylpyridine}>\text{2,6-dimethylpyridine}>\text{2-$t$-butylpyridine}$.
    \end{itemize}
    \item An example of B strain: The fact that tri($t$-butyl)boron vs. \ce{H+} elicit a reversed order of the basicity of \ce{NH3}, \ce{MeNH2}, \ce{Me2NH}, and \ce{Me3N}.
    \item \textbf{Frustrated Lewis pair}: A lone pair on a Lewis base that does not form a traditional adduct with a Lewis acid because both molecules are hindered by excessive steric bulk. \emph{Also known as} \textbf{FLP}.
    \item Example (refer to Figure \ref{fig:FLPexample} throughout the following discussion):
    \begin{figure}[h!]
        \centering
        \begin{subfigure}[b]{0.4\linewidth}
            \centering
            \chemfig[atom sep=5mm]{B(-[2,1.6]*6(=(-F)-(-F)=(-F)-(-F)=(-F)-))(-[:-30,1.6]*6(=(-F)-(-F)=(-F)-(-F)=(-F)-))(-[:-150,1.6]*6(=(-F)-(-F)=(-F)-(-F)=(-F)-))}
            \caption{Tris(pentafluorophenyl)borane.}
            \label{fig:FLPexamplea}
        \end{subfigure}
        \begin{subfigure}[b]{0.4\linewidth}
            \centering
            \chemfig[atom sep=5mm]{\charge{180=\:}{P}(<H)(>:[:60,1.5]*6(=(-CH_3)-=(-CH_3)-=(-H_3C)-))(-[:-60,1.5]*6(=(-H_3C)-=(-CH_3)-=(-CH_3)-))}
            \caption{di(2,4,6-trimethylphenyl)phosphine.}
            \label{fig:FLPexampleb}
        \end{subfigure}\\[1em]
        \begin{subfigure}[b]{0.49\linewidth}
            \centering
            \chemfig[atom sep=5mm]{\charge{45=$+$}{P}(<[:-150]\ce{H2(CH3)3}C_6)(>:[:150]\ce{H2(CH3)3}C_6)(-[2]H)-*6(-(-F)=(-F)-(-\charge{135=$-$}{B}(-[2]F)(>:[:-10]C_6H_5)(<[:-70]C_6H_5))-(-F)=(-F)-)}
            \caption{Initial phosphino-borane compound.}
            \label{fig:FLPexamplec}
        \end{subfigure}
        \begin{subfigure}[b]{0.49\linewidth}
            \centering
            \chemfig[atom sep=5mm]{\charge{45=$+$}{P}(<[:-150]\ce{H2(CH3)3}C_6)(>:[:150]\ce{H2(CH3)3}C_6)(-[2]H)-*6(-(-F)=(-F)-(-\charge{135=$-$}{B}(-[2]H)(>:[:-10]C_6H_5)(<[:-70]C_6H_5))-(-F)=(-F)-)}
            \caption{Hydrogenated phosphino-borane compound.}
            \label{fig:FLPexampled}
        \end{subfigure}
        \caption{FLP phosphino-borane compound.}
        \label{fig:FLPexample}
    \end{figure}
    \begin{itemize}
        \item Although the FLP in di(2,4,6-trimethylphenyl)phosphine does not attack the boron in\\tris(pentafluorophenyl)borane, it reacts with a para carbon of the borane to create a \textbf{zwitterionic} species after fluoride migration.
        \item This fluoride can be substituted for a hydrogen, creating a species that releases hydrogen gas upon heating and reacts with hydrogen gas at ambient temperature to reform the zwitterion.
        \item The phosphino-borane is the first non-transition metal species that can reversibly activate the \ce{H-H} bond in \ce{H2}.
    \end{itemize}
    \item \textbf{Zwitterion}: A species that contains at least one formal positive and negative charge.
    \item More small molecules have been activated with FLPs, including \ce{CO2} and \ce{N2O}.
    \item \textbf{Hydrogen} (bond): An \ce{X-H\cdots B} bond formed from an attraction between an \ce{X-H} unit (where the electronegativity of \ce{X} is greater than that of \ce{H}) and a donor atom \ce{B}.
    \begin{itemize}
        \item The components may be incorportated into larger molecular fragments.
        \item Hydrogen bonds can be either intermolecular or intramolecular.
    \end{itemize}
    \item Hydrogen bonds can be described on the basis of the varying relative contributions from three components:
    \begin{itemize}
        \item Electrostatic contribution: The polarity of \ce{X-H}.
        \item Partial covalent character and charge transfer: The donor-acceptor nature of the interaction.
        \item Dispersion forces.
    \end{itemize}
    \item Experimental stipulations that must be satisfied for an interaction to be declared a hydrogen bond.
    \begin{itemize}
        \item Bond angle close to $\ang{180}$.
        \item Red-shifted IR stretching frequency for the \ce{X-H} bond (indicative of the bond weakening and, thus, donation into antibonding orbitals).
        \item High deshielding of NMR chemical shift (extremely sensitive probe for hydrogen bonding).
        \item Magnitude of $\Delta G$ for bond formation must exceed the thermal energy of the system.
    \end{itemize}
    \item \textbf{H-bond puzzle}: Predicting the strength of hydrogen bonds solely on the basis of the structures of the participating molecules.
    \begin{itemize}
        \item Based in the fact that structurally very similar hydrogen bonds can exhibit massive differences in strength.
        \item For example, the \ce{O-H\cdots O} bond between hydronium and water is roughly six times stronger than the comparable bond between two water molecules.
    \end{itemize}
    \item \textbf{$\pKa$ equalization} (paradigm): An approach to predicting the strength of hydrogen bonds in aqueous solution by envisioning \ce{H}-bond strength as associated with the 3-way proton transfer equilibrium between \ce{X} and \ce{B}.
    \begin{itemize}
        \item The better matched the $\pKa$ values, the stronger the predicted H-bond.
        \item In effect, a strong H-bond minimizes $\Delta\pKa(\ce{X-H\cdots B})=\pKa(\ce{HX})-\pKa(\ce{BH+})$.
    \end{itemize}
    \item \textbf{Receptor-guest interaction}: An interaction between molecules with extended pi systems where their pi systems interact with each other to hold the molecules or portions of the molecules together. \emph{Also known as} \textbf{receptor-substrate interaction}, \textbf{host-guest interaction}.
    \begin{itemize}
        \item One example is a double-concave hydrocarbon buckycatcher that forms a ball-and-socket structure with \ce{C60}. This is an example of an \textbf{inclusion complex}.
    \end{itemize}
    \item The hard vs. soft distinction is largely a result of polarizability, with soft acid having high polarizability and hard acids having low polarizability.
    \item Soft transition metals have $d$ electrons available for $\pi$-bonding.
    \item \textbf{Exchange reaction}: A reaction involving the exchange of a water and a base \ce{B}.
    \item Is \ce{OH-} or \ce{S^2-} more likely to form insoluble salts with $3+$ transition-metal ions? Which is more likely to form insoluble salts with $2+$ transition-metal ions?
    \begin{itemize}
        \item \ce{OH-} and $3+$ are hard, and thus will form insoluble salts.
        \item \ce{S^2-} and $2+$ are soft, and thus will form insoluble salts.
    \end{itemize}
    \item Both hardness/softness and strength must be considered to determine reactivity.
    \item MO diagrams for hard-hard and soft-soft interactions reveal that there is a lower covalent contribution in hard-hard interactions, but this is compensated for by a strong ionic contribution, so don't think that hard-hard interactions are weaker.
    \item Wrt. chemical hardness $\eta$ (Greek "eta"), ionization energy is assumed to measure the energy of the HOMO ($E_\text{HOMO}=-I$) and electron affinity the energy of the LUMO ($E_\text{LUMO}=-A$); thus, a hard species is one with a high difference in energy between its HOMO and LUMO.
    \item \textbf{Chemical softness}: The inverse $\sigma=\frac{1}{\eta}$ of hardness.
    \item Reviews a quantitative system of acid-base parameters to account for reactivity by explicitly including electrostatic and covalent factors.
\end{itemize}




\end{document}