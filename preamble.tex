\usepackage[margin=1in]{geometry}
\usepackage{fancyhdr}
\usepackage{csquotes}
\usepackage{marginnote}
\usepackage{scrextend}
\usepackage[bottom]{footmisc}
\usepackage[style=apa]{biblatex}
\usepackage{enumitem}
\usepackage{xr}
\usepackage{tocloft}
\usepackage{amsmath,amssymb,amsthm}
\usepackage{physics,mathtools,siunitx,bm}
\usepackage{mhchem,chemfig}
\usepackage{tikz,graphicx}
\usepackage{subcaption,float}
\usepackage{colortbl,multirow,tabularx}
\usepackage[hidelinks]{hyperref}

\fancypagestyle{main}{
    \fancyhf{}
    \fancyfoot[R]{Labalme \thepage}
    \fancyhead[L]{\leftmark}
    \fancyhead[R]{CHEM 20100}
}
\fancypagestyle{plain}{
    \fancyhead{}
    \renewcommand{\headrulewidth}{0pt}
}

\MakeOuterQuote{"}

\reversemarginpar

\deffootnotemark{\textsuperscript{\textup{[}\thefootnotemark\textup{]}}}
\deffootnote[2.1em]{0em}{0em}{\textsuperscript{\thefootnote}}

\addbibresource{../main.bib}
\DefineBibliographyStrings{english}{bibliography={References}}

\setitemize[3]{label={\scriptsize$\blacksquare$}}

\setchemfig{atom sep=2em,fixed length=true,bond offset=3pt,cram width=3pt}
\setcharge{extra sep=3pt}

\usetikzlibrary{decorations.pathreplacing,3d,fpu,positioning,intersections,backgrounds,arrows.meta,bending,decorations.fractals}
\colorlet{grx}{green!25!cyan!40!black!95}
\colorlet{gry}{green!25!cyan!40!black!50}
\colorlet{grz}{green!25!cyan!40!black!20}
\colorlet{grt}{green!75!cyan!40!black!15}
\colorlet{gax}{gray!25}
\tikzset{
    dashbondg/.style={draw=grt,double=black,very thick,double distance=0.4pt,dash pattern=on 2pt off 2pt},
    dashbond/.style={draw=white,double=black,very thick,double distance=0.4pt,dash pattern=on 2pt off 2pt}
}

\newcommand{\e}[1][]{\text{e}^{#1}}

\newcommand{\porbital}[6]{
    \tikz[node distance=2mm]{
        \node (p1) {${\color{#1}\uparrow}\ {\color{#2}\downarrow}$}
            (p1.south east) edge (p1.south west)
        ;
        \node (p2) [right=of p1] {${\color{#3}\uparrow}\ {\color{#4}\downarrow}$}
            (p2.south east) edge (p2.south west)
        ;
        \node (p3) [right=of p2] {${\color{#5}\uparrow}\ {\color{#6}\downarrow}$}
            (p3.south east) edge (p3.south west)
        ;
    }
}

\newenvironment{tchart}[3]
{
    \renewcommand{\arraystretch}{#1}
    \tabularx{\linewidth}{X|X}
    \multicolumn{1}{c|}{\textbf{#2}} & \multicolumn{1}{c}{\textbf{#3}}\\
    \hline
}{
    \endtabularx
    \renewcommand{\arraystretch}{1}
}

\usepackage{subfiles}