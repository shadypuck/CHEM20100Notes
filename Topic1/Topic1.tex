\documentclass[../main.tex]{subfiles}

\pagestyle{main}
\renewcommand{\chaptermark}[1]{\markboth{\chaptername\ \thechapter\ (#1)}{}}
\renewcommand{\thechapter}{\Roman{chapter}}

\begin{document}




\chapter{Review of VSEPR Theory}
\section{Module 1: Course Logistics and History}
\begin{itemize}
    \item \marginnote{1/11:}Homework questions will be similar to exam questions, so although you probably \emph{can} find answers online, you shouldn't.
    \item Submit Psets to \href{mailto:chem201hw@gmail.com}{chem201hw@gmail.com}.
    \item Watch modules before office hours and bring questions.
    \item If you have a question outside of office hours, post it on Slack.
    \item It's a difficult class, but he is open to and welcomes our feedback (via Slack, again).
    \item You do need to read from \textcite{bib:MiesslerFischerTarr}, too; his class is not a replacement for this textbook.
    \begin{itemize}
        \item He is a big fan of \textcite{bib:Cotton}.
        \item There is an extra, new textbook to look for!
    \end{itemize}
    \item Convince yourself not to be afraid of time-independent quantum mechanics (we won't go too deep, but know wave functions and the like).
    \item Exams will probably be open book/open note.
    \item Inorganic chemistry contains too much information to rationalize empirically, so we need a system (the development of this system will be the focus of this course).
    \item Reviews history of chemistry from \textcite{bib:MiesslerFischerTarr} Chapter 1.
    \item What is Nickel's electron configuration?
    \begin{itemize}
        \item When Nickel is a free atom, the $[\text{Ar}]4s^23d^8$ electron configuration is the lowest energy.
        \item When Nickel is chemically bound, the $[\text{Ar}]3d^{10}$ electron configuration is the lowest energy because it is energetically unfavorable to have a large 4s orbital pushing the bounds of the atom.
        \item What is a \textbf{term symbol}?
    \end{itemize}
    \item Homework: Refresh Chapter 2 in \textcite{bib:MiesslerFischerTarr}.
    \item \textbf{Covalent bond}: The sharing of pairs of electrons\dots?
    \item G. N. Lewis predicts in 1916 (before Rutherford) that the atom has a positive \textbf{kernel} surrounded by a shell containing up to 8 electrons.
    \begin{itemize}
        \item Also orbital penetration.
        \item He recommends that we read the full paper: \textcite{bib:Lewis}.
    \end{itemize}
\end{itemize}



\section{Module 2: Molecular Geometries and VSEPR}
\begin{itemize}
    \item The easiest way to approach a new Lewis structure:
    \begin{enumerate}
        \item Draw a valid Lewis structure for a molecule.
        \item Place electron pairs in the valence shell as far away from each other as possible. Use the $\sigma$-bond framework first.
        \item Add $\pi$-bonds to complete the molecule.
    \end{enumerate}
    \item Through the VSEPR approach, think of a molecule as arranged around a central atom $A$ by $m$ atoms or groups of atoms $X$ and $n$ lone electron pairs $E$.
    \item \textbf{Steric number}: The sum $n+m$ of groups and electron pairs around the central atom.
    \item Steric numbers correspond to geometries.
    \item VSEPR is ok but it doesn't capture reality too well.
    \item Consider trimethyl boron (\ce{BMe3}).
    \begin{itemize}
        \item Trigonal planar ($D_{3h}$).
    \end{itemize}
    \item Octahedral: $O_h$.
    \item Bent: $C_{2v}$.
    \item \textbf{Order of the repulsive forces}: lone pair - lone pair $>$ lone pair - bonding pair $>$ bonding pair - bonding pair.
    \item In \ce{SF4} (see-saw), will the lone pair be axial or equatorial?
    \begin{itemize}
        \item Equatorial --- $2\times\ang{120}$ and $2\times\ang{90}$ vs. $3\times\ang{90}$.
    \end{itemize}
    \item In \ce{BrF3}\dots
    \begin{figure}[h!]
        \centering
        \begin{subfigure}[b]{0.2\linewidth}
            \centering
            \chemfig{Br(-F)(-[2]F)(-[6]F)(>:[:160]\Charge{160=\:}{})(<[:-150]\charge{-150=\:}{})}
            \caption{$4\times\ang{90}$.}
            \label{fig:VSEPR-BrF3a}
        \end{subfigure}
        \begin{subfigure}[b]{0.2\linewidth}
            \centering
            \chemfig{Br(-F)(-[2]\Charge{90=\:}{})(-[6]\Charge{90=\:}{})(>:[:160]F)(<[:-150]F)}
            \caption{$6\times\ang{90}$.}
            \label{fig:VSEPR-BrF3b}
        \end{subfigure}
        \begin{subfigure}[b]{0.2\linewidth}
            \centering
            \chemfig{Br(-\Charge{0=\:}{})(-[2]F)(-[6]\Charge{90=\:}{})(>:[:160]F)(<[:-150]F)}
            \caption{Cis lone pairs.}
            \label{fig:VSEPR-BrF3c}
        \end{subfigure}
        \caption{VSEPR structure of \ce{BrF3}.}
        \label{fig:VSEPR-BrF3}
    \end{figure}
    \begin{itemize}
        \item T-shaped $\to$ Distorted T --- $4\times\ang{90}$ vs. $6\times\ang{90}$ or lone pairs in cis-position.
    \end{itemize}
    \item In ions such as \ce{ICl4-}, we get square planar ($D_{4h}$).
    \item With mixed substituents (such as \ce{PF2Cl3})\dots
    \begin{figure}[h!]
        \centering
        \chemfig{P(-Cl)(-[2]F)(-[6]F)(>:[:160]Cl)(<[:-150]Cl)}
        \caption{Lewis structure of \ce{PF2Cl3}.}
        \label{fig:Lewis-PF2Cl3}
    \end{figure}
    \begin{itemize}
        \item We need \textbf{Bent's rule}, which tells us that atoms share electrons from $p$- or $d$-orbitals to a greater extent than they do from $s$-orbitals.
        \item Thus, when phosphorous excites $3s^23p^3$ to $3s^13p_x^13p_y^1ep_z^13d_{z^2}^2$ and then rehybridizes to create three $sp^2$ orbitals (each composed of $s+p_x+p_y$) and two "$pd$" hybrid orbitals (each composed of $p_z+d_{z^2}$), the equatorial $sp^2$ orbitals bond to the more \textbf{electropositive} chlorines and the axial "$pd$" hybrid orbitals bond to the remaining more electronegative fluorines.
    \end{itemize}
    \item \textbf{Bent's rule}: Atomic $s$-character concentrates in orbitals directed toward electropositive substituents.
    \item \textbf{Electropositive} (species): A species that has relatively lower electronegativity than another.
    \item For molecules with multiple bonds, ignore $\pi$-bonds.
    \item Linear: $D_{\infty h}$.
    \item Problems with VSEPR:
    \begin{itemize}
        \item \ce{XeF6} with 14 bonding electrons (7 pairs) is supposed to be pentagonal bipyramidal, but is actually octahedral (a known problem for \ce{14e-} systems).
        \item Heavy main group elements with no hybridization.
        \begin{itemize}
            \item \ce{H-C#C-H} is linear, but \ce{H-Si#Si-H} is not.
            \item No $\sigma$-bond exists in the latter species --- it's all $\pi$-bonding interactions.
        \end{itemize}
    \end{itemize}
    \item You maybe don't have to watch the modules and textbook \emph{and} attend class.
\end{itemize}



\section{Chapter 3: Simple Bonding Theory}
\emph{From \textcite{bib:MiesslerFischerTarr}.}
\subsection{Notes}
\begin{itemize}
    \item \marginnote{1/14:}\textbf{Hypervalent} (central atom): A central atom that has an electron count greater than the atom's usual requirement.
    \item There are rarely more than 18 electrons around a central atom (2 for $s$, 6 for $p$, 10 for $d$). Even heavier atoms with energetically accessible $f$ orbitals usually don't have more surrounding electrons because of crowding.
    \item With \ce{BeF2}, instead of getting the predicted double-bonded lewis structure, it forms a complex network with \ce{Be} having coordination number 4.
    \begin{itemize}
        \item \ce{BeCl2} dimerizes to a 3-coordinate structure in the vapor phase.
    \end{itemize}
    \item Boron trihalides exhibit partial double bond character.
    \begin{itemize}
        \item It is also possible that the high polarity of \ce{B-X} bonds and the \textbf{ligand-close packing} (LCP) model account for the observed shorter bond length.
        \item Boron trihalides also act as Lewis acids.
    \end{itemize}
    \begin{table}[h!]\marginnote{1/14:}
        \centering
        \renewcommand{\arraystretch}{1.4}
        \small
        \begin{tabular}{lp{1.7cm}llc}
            \rowcolor{grx}
            \textcolor{white}{\textbf{Steric Number}} & \textcolor{white}{\textbf{Geometry}} & \textcolor{white}{\textbf{Examples}} & \textcolor{white}{\textbf{Calculated Bond Angles}} & \\
            2 & Linear & \ce{CO2} & $\ang{180}$ & \chemfig[atom sep=7mm]{O=C=O}\\
    
            \rowcolor{grt}
            3 & Trigonal (triangular) & \ce{SO3} & $\ang{120}$ & \tikz[node distance=3mm,baseline={(0,0)}]{
                \node (S) [circle,inner sep=2pt] {\ce{S}};
                \node (O1) at (90:0.8) {\ce{O}}
                    (O1.260) edge [dashed,dash pattern=on 1.5pt off 1.2pt] (O1.260 |- S.100)
                    (O1.280) edge (O1.280 |- S.80)
                ;
                \node (O2) [circle,inner sep=1pt] at (-150:0.8) {\ce{O}}
                    (O2.20) edge [dashed,dash pattern=on 1.5pt off 1.2pt] (S.220)
                    (O2.40) edge (S.200)
                ;
                \node (O3) [circle,inner sep=1pt] at (-30:0.8) {\ce{O}}
                    (O3.140) edge [dashed,dash pattern=on 1.5pt off 1.2pt] (S.340)
                    (O3.160) edge (S.320)
                ;
            }\rule{0pt}{1.1cm}\\[5mm]
    
            4 & Tetrahedral & \ce{CH4} & $\ang{109.5}$ & \chemfig[atom sep=2.6em,cram width=3pt]{C(-[7]H)(>:[:60]H)(<[:120]H)-[5]H}\rule{0pt}{1.1cm}\\[6mm]
    
            \rowcolor{grt}
            5 & Trigonal bipyrimidal & \ce{PCl5} & $\ang{120},\ang{90}$ & \chemfig[atom sep=2.6em,cram width=3pt]{P(-Cl)(-[2]Cl)(-[6]Cl)(>:[:150]Cl)<[:-150]Cl}\rule{0pt}{1.3cm}\\[9mm]
    
            6 & Octahedral & \ce{SF6} & $\ang{90}$ & \chemfig[atom sep=2.6em,cram width=3pt]{S(>:[:30]F)(-[2]F)(>:[:150]F)(<[:-150]F)(-[6]F)<[:-30]F}\rule{0pt}{1.3cm}\\[9mm]
    
            \rowcolor{grt}
            7 & Pentagonal bipyrimidal & \ce{IF7} & $\ang{72},\ang{90}$ & \chemfig[atom sep=3.8em,cram width=3pt]{I(-[6]F)(<[:-50,0.6]F?[a])(-[:5,,,,dashbondg]F?[b]?[a,1,dashbondg])(-[:65,0.5]F?[c]?[b,1,dashbondg])(-[:173,,,,dashbondg]F?[d]?[c,1,dashbondg])(<[:-155]F?[a,1,dashbondg]?[d,1,dashbondg])-[2,,,,draw=grt,very thick,double=black,double distance=0.4pt]F}\rule{0pt}{1.7cm}\\[1.3cm]
    
            8 & Square\par antiprismatic & \ce{[TaF8]^3-} & $\ang{70.5},\ang{9.6},\ang{109.5}$ & \chemfig[atom sep=3.9em,cram width=3pt]{Ta(-[:47,1.2,,,dashbond]F?[a])(-[:120,,,,dashbond]F?[b]?[a])(<[:155,0.9]F?[c]?[b])(<[:30,0.8]F?[a]?[c])(-[:-35,1.1]F?[A])(-[:-75,0.5,,,dashbond]F?[B]?[A])(-[:-145,1.1]F?[C]?[B])(<[:-100]F?[A]?[C])}\rule{0pt}{1.5cm}\\[1.2cm]
    
            \noalign{\global\arrayrulewidth=1pt}\arrayrulecolor{grx}\hline
        \end{tabular}
        \caption{VSEPR predictions.}
        \label{tab:VSEPR}
    \end{table}
    \item \marginnote{1/17:}The variety of structures means that one unified VSEPR theory will not likely\footnote{Eratta: "unlikely."} work.
    \item \textbf{Not stereochemically active} (lone pair): "A lone pair that appears in the Lewis-dot structure but has no apparent effect on the molecular geometry" \parencite[54]{bib:MiesslerFischerTarr}.
    \item Double and triple bonds have slightly greater repulsive effects than single bonds in the VSEPR model.
    \item \marginnote{1/17:}Multiple bonds tend to occupy the same positions as lone pairs.
    \item Electronegativity varies for a given atom based on the neighboring atom to which it is bonded.
    \item "With the exception of helium and neon, which have large calculated electronegativities and no known stable compounds, fluorine has the largest value" \parencite[59]{bib:MiesslerFischerTarr}.
    \item Although usually classified with Group 1, Hydrogen's chemistry is distinct from that of the alkali metals and actually all of the groups.
    \item Some bond angle trends can be explained by electronegativity.
    \begin{table}[h!]
        \centering
        \renewcommand{\arraystretch}{1.4}
        \small
        \begin{tabular}{lSlS}
           \rowcolor{grx}
           \textcolor{white}{\textbf{Molecule}} & \textcolor{white}{\textbf{\ce{X-P-X} Angle ($\bm{{}^\circ}$)}} & \textcolor{white}{\textbf{Molecule}} & \textcolor{white}{\textbf{Bond Angle ($\bm{{}^\circ}$)}}\\
    
           \quad\ce{PF3}  & 97.8  & \quad\ce{H2O}  & 104.5\\
    
           \rowcolor{grt}
           \quad\ce{PCl3} & 100.3 & \quad\ce{H2S}  & 92.1\\
    
           \quad\ce{PBr3} & 101.0 & \quad\ce{H2Se} & 90.6\\
           \noalign{\global\arrayrulewidth=1pt}\arrayrulecolor{grx}\hline
        \end{tabular}
        \caption{Electronegativity and bond angles.}
        \label{tab:electronegativityBondAngle}
    \end{table}
    \begin{itemize}
        \item For instance, electronegative outer atoms pull electrons away from the central atom, allowing lone pairs to further push together such atoms.
        \item Electronegative central atoms pull electrons toward the central atom, pushing bonding pairs farther apart.
    \end{itemize}
    \item Atomic size can also have effects on VSEPR predictions.
    \begin{itemize}
        \item For example, the \ce{C-N-C} angle in \ce{N(CF3)3} is larger than that of \ce{N(CH3)3} despite the prediction we'd make based on electronegativity alone. This is because \ce{F} atoms are significantly larger than \ce{H} atoms and we get some steric hindrance.
    \end{itemize}
    \item In molecules with steric number 5, axial bond length is greater than equatorial.
    \item Symmetric structures are often preferred.
    \item Groups (such as \ce{CH3} and \ce{CF3}) have the ability to attract electrons, too --- thus, they are also assigned electronegativities.
    \item \textbf{Ligand close-packing}: A model that uses the distances between outer atoms in molecules as a guide to molecular shapes. \emph{Also known as} \textbf{LCP}.
    \begin{itemize}
        \item Works off of the observation that the nonbonded distances between outer atoms are consistent across molecules with the same central atom, but the bond angles and lengths change.
        \item This stands in contrast to VSEPR theory's concern with the central atom, as opposed to the ligands.
    \end{itemize}
    \item \textbf{Dielectric constant}: "The ratio of the capacitance of a cell filled with the substance to be measured to the capacitance of the same cell with a vacuum between the electrodes" \parencite[66]{bib:MiesslerFischerTarr}.
    \begin{itemize}
        \item This is measured to experimentally determine the polarity of molecules.
    \end{itemize}
    \item \textbf{Dipole moment}: The product $Qr$ of the distance $r$ between two charges' centers and the difference $Q$ between the charges. \emph{Also known as} $\bm{\mu}$.
    \begin{itemize}
        \item This is calculated by measuring the dielectric constant at different temperatures.
        \item SI unit: Coulomb meter; $\si{\coulomb\meter}$. Common unit: Debye; $\SI{1}{D}=\SI{3.33564e-30}{\coulomb\meter}$.
    \end{itemize}
\end{itemize}




\end{document}